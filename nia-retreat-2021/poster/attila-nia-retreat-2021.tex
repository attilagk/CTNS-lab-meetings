\documentclass[17pt,a1paper]{tikzposter} %Options for format can be included here
\usepackage{multicol}
\setlength{\columnsep}{0.7cm}
\usepackage{natbib}
\bibliographystyle{plainnat}
%\bibliographystyle{unsrtnat}
\author{Attila Jones, ..., Madhav Thambisetty}
\title{Network based drug repurposing for Alzheimer's disease (AD)}

\institute{Clinical \& Translational Neuroscience Section, Laboratory of Behavioral Neuroscience}

 %Choose Layout
\usetheme{Default}

\begin{document}
\maketitle

\block{Abstract}{
\begin{multicols}{2}
To this date Alzheimer's disease (AD) is without efficient
treatment so we recently announced the Drug Repurposing for Effective
Alzheimer's Medicines (DREAM) study \citep{Desai2020}, in which we plan to
find repurposable drugs based on:

\begin{enumerate}
\item our novel proteomics study \citep{JacksonA.Roberts2021}
\item transcriptome-wide association studies, TWAS, on AD
	\citep{Baird2021,Gerring2020,Jansen2019,Kunkle2019a}
\item curated knowledge on established AD genes \citep{PletscherFrankild2015}
\item network based drug repurposing approach \citep{Guney2016,Cheng2018}
\end{enumerate}

Below we provide an overview on the approach and on the input data sets.
\end{multicols}
}

\begin{columns}
\column{0.55}
\block{The approach: principles and utility}{
\begin{multicols}{2}
Hello World!

\cite{Cheng2018},

\includegraphics[width=1.0\columnwidth]{../figures/from-others/cheng-desai-2018-Nat-comm-Fig4a.png}
\end{multicols}
}

\block{The approach: details}{

\begin{multicols}{3}
Definition of network proximity from a set of drug targets $T$ to a set of disease genes $S$:
%$d(S, T) = \frac{1}{|T|}\sum_{t \in T} \min_{s \in S} d(s, t),$

\begin{equation}
	d = \frac{1}{|T|}\sum_{t \in T} \min_{s \in S} d(s, t),
\end{equation}

where $d(s,t)$
is the shortest path length between disease gene $s$ and drug target $t$.

Interpretation: the average length of shortest paths taken from all drug
targets to the disease genes.

We present two examples in a toy network with fixed $S = \{B, D, E\}$ and
varied $T$.

\pagebreak

\begin{center}
{\Large Distal drug targets}
\end{center}
\begin{eqnarray*}
T &=& \{C, H\} \\
d &=& 3/2 \\
z &=& 2.217 \\
p &=& 0.987 \\
\end{eqnarray*}


\includegraphics[width=1.0\columnwidth]{../../../results/2021-06-14-proximity/toy-distal-arrow.png}

\pagebreak

\begin{center}
{\Large Proximal drug targets}
\end{center}
\begin{eqnarray*}
T &=& \{E, J\} \\
d &=& 1/2 \\
z &=& -0.454 \\
p &=& 0.324 \\
\end{eqnarray*}


\includegraphics[width=1.0\columnwidth]{../../../results/2021-06-14-proximity/toy-proximal-arrow.png}

\end{multicols}

%Above are two examples in a toy network with fixed $S = \{B, D, E\}$ and
%varied $T$.
%
%\begin{flushright}
%%{\Large Results}
%%
%\begin{tabular}{|c|c|c|}
%Distal targets & & Proximal targets \\
%\hline
%3/2 & proximity $d$ & 1/2 \\
%2.217 & standardized prox.~$z$ & -0.454 \\
%0.987 & $p$ value & 0.324 \\
%\hline
%\end{tabular}
%\end{flushright}
}

\column{0.45}

\block{Planned workflow}{
\includegraphics[width=0.4\columnwidth]{../figures/by-me/network-repos-flowchart/network-repos-flowchart.pdf}
}

\block{}{
\tiny
\bibliography{library}
}

\end{columns}

\end{document}
