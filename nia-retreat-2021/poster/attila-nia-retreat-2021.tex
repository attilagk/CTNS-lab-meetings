\documentclass[17pt,a1paper]{tikzposter} %Options for format can be included here
\usepackage{multicol}
\setlength{\columnsep}{1in}
\usepackage{natbib}
\bibliographystyle{plainnat}
\author{Attila Jones, ..., Madhav Thambisetty}
\title{Network based drug repurposing for Alzheimer's disease}

\institute{Clinical \& Translational Neuroscience Section, Laboratory of Behavioral Neuroscience}

 %Choose Layout
\usetheme{Default}

\begin{document}
\maketitle

\block{Abstract}{
\begin{multicols}{2}
To this date AD is without efficient
treatment so we recently announced the Drug Repurposing for Effective
Alzheimer's Medicines (DREAM) study \citep{Desai2020}, in which we plan to
find repurposable drugs based on:

\begin{itemize}
\item our novel proteomics study \citep{JacksonA.Roberts2021}
\item several TWAS (transcriptome-wide association studies)
\item curated knowledge on established AD genes \citep{PletscherFrankild2015}
\item network based drug repurposing approach \citep{Guney2016,Cheng2018}
\end{itemize}

Below we provide an overview on the approach and on the input data sets.
\end{multicols}
}

\begin{columns}
\column{0.5}
\block{The approach: principles and utility}{
Hello World!

\cite{Cheng2018},

\includegraphics[width=0.2\columnwidth]{../figures/from-others/cheng-desai-2018-Nat-comm-Fig4a.png}

}

\block{The approach: details}{
Definition of network proximity from a set of drug targets $T$ to a set of disease genes $S$:
$d(S, T) = \frac{1}{|T|}\sum_{t \in T} \min_{s \in S} d(s, t),$
%\begin{equation}
%	d(S, T) = \frac{1}{|T|}\sum_{t \in T} \min_{s \in S} d(s, t),
%\end{equation}
where $d(s,t)$
is the shortest path length between disease gene $s$ and drug target $t$.

Interpretation: the average length of shortest paths taken from all drug
targets to the disease genes.

Below are two examples in a toy network with fixed $S = \{B, D, E\}$ and
varied $T$.

\begin{multicols}{2}

\begin{center}
Distal targets:
$T = \{C, H\}$
\end{center}

\includegraphics[width=0.8\columnwidth]{../../../results/2021-06-14-proximity/toy-distal-arrow.png}

\begin{center}
Proximal drug targets:
$T = \{E, J\}$
\end{center}

\includegraphics[width=0.8\columnwidth]{../../../results/2021-06-14-proximity/toy-proximal-arrow.png}

\end{multicols}

\begin{center}
%{\Large Results}
%
\begin{tabular}{|c|c|c|}
Distal targets & & Proximal targets \\
\hline
3/2 & proximity $d$ & 1/2 \\
2.217 & standardized prox.~$z$ & -0.454 \\
0.987 & $p$ value & 0.324 \\
\hline
\end{tabular}
\end{center}
}

\column{0.5}

\block{Planned workflow}{
\includegraphics[width=0.4\columnwidth]{../figures/by-me/network-repos-flowchart/network-repos-flowchart.pdf}
}

\block{}{
\small
\bibliography{library}
}

\end{columns}

\end{document}




\begin{columns}
\column{0.5}
\block{The approach: details}{
Definition of the closest proximity $d_c$ or just $d$ as
\begin{equation}
	d(S, T) = \frac{1}{|T|}\sum_{t \in T} \min_{s \in S} d(s, t)
\end{equation}
where $S$ is the set of disease genes, $T$ the set of drug targets, $d(s,t)$
is the shortest path length between disease gene $s$ and drug target $t$.

Interpretation: the average length of shortest paths taken from all drug
targets to the disease genes.
}

\column{0.5}

\block{}{
\small
\bibliography{library}
}

% RIGHT COLUMN
\column{0.5}

\end{columns}

