\documentclass[17pt,a1paper]{tikzposter} %Options for format can be included here
\usepackage{multicol}
\setlength{\columnsep}{0.7cm}
\usepackage{natbib}
\bibliographystyle{plainnat}
%\bibliographystyle{unsrtnat}
\author{Attila Jones, Andrew Williamson, Vijay Varma, Jackson Roberts, Madhav Thambisetty}
\title{Network based drug repurposing for Alzheimer's disease (AD)}

\institute{Clinical \& Translational Neuroscience Section, Laboratory of Behavioral Neuroscience}

 %Choose Layout
\usetheme{Default}

\begin{document}
\maketitle

\block{1. Introduction}{
\begin{multicols}{2}
We recently announced the Drug Repurposing for Effective
Alzheimer's Medicines (DREAM) study \citep{Desai2020}, in which we plan to
find repurposable drugs based on:

\begin{enumerate}
\item our novel proteomics study \citep{JacksonA.Roberts2021} that describes
	the \emph{incipient AD proteomic signature}
\item transcriptome-wide association studies, TWAS, on AD
	\citep{Baird2021,Gerring2020,Jansen2019a,Kunkle2019a}
\item curated knowledge on established AD genes \citep{PletscherFrankild2015}
\item a network based drug repurposing approach \citep{Guney2016,Cheng2018}
\end{enumerate}

Below we provide an overview on the approach and on the input data sets.
\end{multicols}
}

\begin{columns}
\column{0.55}
\block{2. The approach: principles and previous applications}{
\begin{multicols}{2}
We choose the approach to drug repurposing developed by
\cite{Cheng2018} based on their earlier work~\citep{Guney2016}.

\cite{Cheng2018} showed that:
\begin{itemize}
\item cardiovascular (CV) disease genes form clusters, so-called \emph{disease
	modules}, in the human protein-protein interactome (PPI)
\item a few drugs approved for non-CV indications, like Hydroxychloroquine,
	are proximal to the CAD (coronary artery disease) module
\item Hydroxychloroquine's protective effect on CAD is supported by
	pharmaco-epidemiology and an \emph{in vitro} assay
%\item these evidence suggests Hydroxychloroquine might be repurposed for CAD
\end{itemize}

\includegraphics[width=0.8\columnwidth]{../figures/from-others/cheng-desai-2018-Nat-comm-Fig4a.png}
\end{multicols}
}

\block{3. Quantifying proximity of drug targets to disease genes}{

\begin{multicols}{3}
Definition of network proximity from a set of drug targets $T$ to a set of disease genes $S$:
%$d(S, T) = \frac{1}{|T|}\sum_{t \in T} \min_{s \in S} d(s, t),$

\begin{equation}
	d = \frac{1}{|T|}\sum_{t \in T} \min_{s \in S} d(s, t),
\end{equation}

where $d(s,t)$
is the shortest path length between disease gene $s$ and drug target $t$.

Interpretation: the average length of the shortest paths taken from all 
targets to the disease genes~\citep{Guney2016}.

We present two examples in a toy network with fixed $S = \{B, D, E\}$ and
varied $T$.

\pagebreak

\begin{center}
{\Large Distal drug targets}
\end{center}
\begin{eqnarray*}
T &=& \{C, H\} \\
d &=& 3/2 \\
z &=& 2.217 \\
p &=& 0.987 \\
\end{eqnarray*}


\includegraphics[height=4in]{../../../results/2021-06-14-proximity/toy-distal-arrow.png}

\pagebreak

\begin{center}
{\Large Proximal drug targets}
\end{center}
\begin{eqnarray*}
T &=& \{E, J\} \\
d &=& 1/2 \\
z &=& -0.454 \\
p &=& 0.324 \\
\end{eqnarray*}

\includegraphics[height=4in]{../../../results/2021-06-14-proximity/toy-proximal-arrow.png}

\end{multicols}
}

\block{}{
\tiny
\bibliography{library}
}

\column{0.45}

\block{4. Planned workflow}{
\includegraphics[width=0.375\columnwidth]{../figures/by-me/network-repos-flowchart/network-repos-flowchart.pdf}
}

\block{5. High confidence AD gene sets}{
%\begin{tabular}{rl}
%Knowledge & established AD genes in knowledge bases complied by~\cite{PletscherFrankild2015} \\
%TWAS & AD genes whose expression is inferred causal to AD by Mendelian randomization and similar techniques \\
%Incipient proteo. & the \emph{incipient AD proteomic signature} described by us~\citep{JacksonA.Roberts2021} \\
%\end{tabular}
\begin{description}
\item[Knowledge] AD genes in knowledge bases~\cite{PletscherFrankild2015} 
\item[TWAS] AD genes whose expression is causal to AD inferred by 'omics
\item[Incipient proteo.] \emph{incipient AD proteomic signature}~\citep{JacksonA.Roberts2021}
\end{description}
\begin{center}
\includegraphics[width=0.35\columnwidth]{../../../notebooks/2021-07-01-high-conf-ADgenes/named-figure/knowledge-twas-proteo-venn.pdf}
\end{center}
}

\end{columns}

\end{document}
