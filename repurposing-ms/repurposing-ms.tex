%Version 3 October 2023
% See section 11 of the User Manual for version history
%
%%%%%%%%%%%%%%%%%%%%%%%%%%%%%%%%%%%%%%%%%%%%%%%%%%%%%%%%%%%%%%%%%%%%%%
%%                                                                 %%
%% Please do not use \input{...} to include other tex files.       %%
%% Submit your LaTeX manuscript as one .tex document.              %%
%%                                                                 %%
%% All additional figures and files should be attached             %%
%% separately and not embedded in the \TeX\ document itself.       %%
%%                                                                 %%
%%%%%%%%%%%%%%%%%%%%%%%%%%%%%%%%%%%%%%%%%%%%%%%%%%%%%%%%%%%%%%%%%%%%%

%%\documentclass[referee,sn-basic]{sn-jnl}% referee option is meant for double line spacing

%%=======================================================%%
%% to print line numbers in the margin use lineno option %%
%%=======================================================%%

%%\documentclass[lineno,sn-basic]{sn-jnl}% Basic Springer Nature Reference Style/Chemistry Reference Style

%%======================================================%%
%% to compile with pdflatex/xelatex use pdflatex option %%
%%======================================================%%

%%\documentclass[pdflatex,sn-basic]{sn-jnl}% Basic Springer Nature Reference Style/Chemistry Reference Style


%%Note: the following reference styles support Namedate and Numbered referencing. By default the style follows the most common style. To switch between the options you can add or remove �Numbered� in the optional parenthesis. 
%%The option is available for: sn-basic.bst, sn-vancouver.bst, sn-chicago.bst%  
 
%%\documentclass[sn-nature]{sn-jnl}% Style for submissions to Nature Portfolio journals
%%\documentclass[sn-basic]{sn-jnl}% Basic Springer Nature Reference Style/Chemistry Reference Style
\documentclass[sn-mathphys-num]{sn-jnl}% Math and Physical Sciences Numbered Reference Style 
%%\documentclass[sn-mathphys-ay]{sn-jnl}% Math and Physical Sciences Author Year Reference Style
%%\documentclass[sn-aps]{sn-jnl}% American Physical Society (APS) Reference Style
%%\documentclass[sn-vancouver,Numbered]{sn-jnl}% Vancouver Reference Style
%%\documentclass[sn-apa]{sn-jnl}% APA Reference Style 
%%\documentclass[sn-chicago]{sn-jnl}% Chicago-based Humanities Reference Style

%%%% Standard Packages
%%<additional latex packages if required can be included here>

\usepackage{graphicx}%
\usepackage{multirow}%
\usepackage{amsmath,amssymb,amsfonts}%
\usepackage{amsthm}%
\usepackage{mathrsfs}%
\usepackage[title]{appendix}%
\usepackage{xcolor}%
\usepackage{textcomp}%
\usepackage{manyfoot}%
\usepackage{booktabs}%
\usepackage{algorithm}%
\usepackage{algorithmicx}%
\usepackage{algpseudocode}%
\usepackage{listings}%
%%%%

%%%%%=============================================================================%%%%
%%%%  Remarks: This template is provided to aid authors with the preparation
%%%%  of original research articles intended for submission to journals published 
%%%%  by Springer Nature. The guidance has been prepared in partnership with 
%%%%  production teams to conform to Springer Nature technical requirements. 
%%%%  Editorial and presentation requirements differ among journal portfolios and 
%%%%  research disciplines. You may find sections in this template are irrelevant 
%%%%  to your work and are empowered to omit any such section if allowed by the 
%%%%  journal you intend to submit to. The submission guidelines and policies 
%%%%  of the journal take precedence. A detailed User Manual is available in the 
%%%%  template package for technical guidance.
%%%%%=============================================================================%%%%

%% as per the requirement new theorem styles can be included as shown below
\theoremstyle{thmstyleone}%
\newtheorem{theorem}{Theorem}%  meant for continuous numbers
%%\newtheorem{theorem}{Theorem}[section]% meant for sectionwise numbers
%% optional argument [theorem] produces theorem numbering sequence instead of independent numbers for Proposition
\newtheorem{proposition}[theorem]{Proposition}% 
%%\newtheorem{proposition}{Proposition}% to get separate numbers for theorem and proposition etc.

\theoremstyle{thmstyletwo}%
\newtheorem{example}{Example}%
\newtheorem{remark}{Remark}%

\theoremstyle{thmstylethree}%
\newtheorem{definition}{Definition}%

\raggedbottom
%%\unnumbered% uncomment this for unnumbered level heads

\UseRawInputEncoding

\begin{document}

\title[Article Title]{Network proximity-based screening and experimental validation of novel drug repurposing candidates in Alzheimer�s Disease}

%%=============================================================%%
%% GivenName	-> \fnm{Joergen W.}
%% Particle	-> \spfx{van der} -> surname prefix
%% FamilyName	-> \sur{Ploeg}
%% Suffix	-> \sfx{IV}
%% \author*[1,2]{\fnm{Joergen W.} \spfx{van der} \sur{Ploeg} 
%%  \sfx{IV}}\email{iauthor@gmail.com}
%%=============================================================%%

\author*[1,2]{\fnm{First} \sur{Author}}\email{iauthor@gmail.com}

\author[2,3]{\fnm{Second} \sur{Author}}\email{iiauthor@gmail.com}
\equalcont{These authors contributed equally to this work.}

\author[1,2]{\fnm{Third} \sur{Author}}\email{iiiauthor@gmail.com}
\equalcont{These authors contributed equally to this work.}

\affil*[1]{\orgdiv{Department}, \orgname{Organization}, \orgaddress{\street{Street}, \city{City}, \postcode{100190}, \state{State}, \country{Country}}}

\affil[2]{\orgdiv{Department}, \orgname{Organization}, \orgaddress{\street{Street}, \city{City}, \postcode{10587}, \state{State}, \country{Country}}}

\affil[3]{\orgdiv{Department}, \orgname{Organization}, \orgaddress{\street{Street}, \city{City}, \postcode{610101}, \state{State}, \country{Country}}}

%%==================================%%
%% Sample for unstructured abstract %%
%%==================================%%

\abstract{The abstract serves both as a general introduction to the topic and as a brief, non-technical summary of the main results and their implications. Authors are advised to check the author instructions for the journal they are submitting to for word limits and if structural elements like subheadings, citations, or equations are permitted.}

%%================================%%
%% Sample for structured abstract %%
%%================================%%

% \abstract{\textbf{Purpose:} The abstract serves both as a general introduction to the topic and as a brief, non-technical summary of the main results and their implications. The abstract must not include subheadings (unless expressly permitted in the journal's Instructions to Authors), equations or citations. As a guide the abstract should not exceed 200 words. Most journals do not set a hard limit however authors are advised to check the author instructions for the journal they are submitting to.
% 
% \textbf{Methods:} The abstract serves both as a general introduction to the topic and as a brief, non-technical summary of the main results and their implications. The abstract must not include subheadings (unless expressly permitted in the journal's Instructions to Authors), equations or citations. As a guide the abstract should not exceed 200 words. Most journals do not set a hard limit however authors are advised to check the author instructions for the journal they are submitting to.
% 
% \textbf{Results:} The abstract serves both as a general introduction to the topic and as a brief, non-technical summary of the main results and their implications. The abstract must not include subheadings (unless expressly permitted in the journal's Instructions to Authors), equations or citations. As a guide the abstract should not exceed 200 words. Most journals do not set a hard limit however authors are advised to check the author instructions for the journal they are submitting to.
% 
% \textbf{Conclusion:} The abstract serves both as a general introduction to the topic and as a brief, non-technical summary of the main results and their implications. The abstract must not include subheadings (unless expressly permitted in the journal's Instructions to Authors), equations or citations. As a guide the abstract should not exceed 200 words. Most journals do not set a hard limit however authors are advised to check the author instructions for the journal they are submitting to.}

\keywords{keyword1, Keyword2, Keyword3, Keyword4}

%%\pacs[JEL Classification]{D8, H51}

%%\pacs[MSC Classification]{35A01, 65L10, 65L12, 65L20, 65L70}

\maketitle

\section{Introduction}\label{sec1}

Alzheimer's disease (AD) is a progressive neurodegenerative disorder and the most
common form of dementia affecting a growing number of individuals in ageing
societies for which effective treatments are
unavailable~\cite{Bondi2017,Masters2015}.  Repurposing drugs approved for other indications is
a promising approach  towards identification of effective disease-modifying AD
treatments~\cite{Pushpakom2019,Fang2021,Taubes2021}.  One of the 
milestones of the National Plan for Alzheimer's Disease is to ``Initiate
research programs for translational bioinformatics and network pharmacology to
support rational drug repositioning''~\cite{NIH/NIA}.%~\citep{Aging}.

Although a few major genetic risk factors for AD are known---most importantly
the $\epsilon 4$ allele of the apolipoprotein E gene (APOE4)
\cite{Yamazaki2019}---our understanding of the early etiological triggers of
the disease is still limited.  Among several reasons for this lack of clarity
are the considerable genetic and cellular complexity of AD, and the
confounding of early etiological processes by compensatory and degenerative
pathologies over the decades-long disease progression~\cite{DeStrooper2016}.
Consequently, discovering treatments targeting early causal drivers of AD
pathogenesis has been a major hurdle to developing effective treatments.  In
this study, we attempted to address this challenge in two ways.  Firstly, we
derived plausible mechanistic insights into AD from previous genome-,
transcriptome- and proteome-wide association studies (GWAS, TWAS, PWAS,
respectively)~\cite{Jansen2019,Kunkle2019,Gerring2020,Baird2021,Schwartzentruber2021,Wightman2021,Wingo2021}
comparing AD and control samples, including analyses of differentially
expressed genes (DEGs) in APOE3/3 versus APOE4/4 individuals as
well as from comparison of DEGs in iPSC-derived neurons, astrocytes and
microglia from APOE3/3 versus APOE4/4
individuals~\cite{Taubes2021,Lin2018}~(Fig.~\ref{fig:workflow}A).

Second, we exploited rich, mechanistic information represented by two
interconnected networks (Fig.~\ref{fig:workflow}B): (1) the network formed by
all drugs and their target genes (drug-target network) and (2) the human
gene-gene interaction network---known as the interactome---, whose Alzheimer's
subnetwork---named Alzheimer's disease module (AD module)~\cite{Barabasi2011}---we
defined based on sets of AD risk genes from
the aforementioned omic studies.  Combining these two networks, we then
estimated the potential efficacy of each drug for AD by evaluating its network
proximity~\cite{Guney2016}, which is based on the drug's network topological
distance from the AD module.  Network proximity was shown~\cite{Cheng2018} to
be a powerful, systems-level, alternative to merely correlative approaches to
computational drug repurposing.  We present the results and computational validation of
our network proximity-based drug screen as well as experimental validation of
selected top ranking plausible novel candidate AD drugs.

\section{Results}

\subsection{Evaluating drugs' network proximities to Alzheimer's disease
modules in the human gene-gene interaction network}

Drugs proximal to a given disease module in the human interactome were previously
shown to be good candidates for repurposing to that disease~\cite{Cheng2018}.
We therefore performed a network proximity-based drug repurposing screen for
AD using 2413 drugs from ChEMBL that were either approved by the United States
Food and Drug Administration, or tested in phase 3 clinical trials, for any
disease indication (Fig.~\ref{fig:workflow}A).

We evaluated each drug's network proximity to each of eight AD modules, which
we derived from distinct complementary information sources (one curated,
knowledge-based source and seven multi-omic sources,
Table~\ref{tab:genesets}).  Thus, given each AD module, we obtained a ranking
of the 2413 drugs according to network proximity such that drugs with the
strongest potential efficiacy for AD were at or near the top (rank 1)
(Fig.~\ref{fig:workflow}A, Fig.~\ref{fig:screen}A-B, Table~S\ref{tab:ranked-drugs}).

Although the eight input AD risk gene sets had little overlap with each other
(Fig.~S\ref{fig:gset-jaccard}), the resulting eight vectors of network proximity had
markedly high correlations (Fig.~S\ref{fig:gset-corr}) so that the eight lists of
drug rankings shared substantial similarity (Fig.~\ref{fig:screen}A).
This suggests that even though the input AD risk gene sets share only a few genes,
the corresponding network modules are close enough to one another in network
topological space to exhibit similar network proximity to any given drug.

Assuming our eight input AD risk gene sets were all equally relevant to
Alzheimer's disease, we aggregated the corresponding drug rankings giving
uniform weight to all gene sets.  This yielded a single final ranked list of
the top 605 drugs (Fig.~\ref{fig:screen}A,
Table~S\ref{tab:genes-in-genesets}).  The highest ranked drugs on the final
list also tended to be ranked high in a majority (five out of eight) of
input-specific rankings (Fig.~\ref{fig:screen}B, Fig.~S\ref{fig:rank-diff}).

\subsection{Computational validation of network proximity-based drug screening}

To validate our network-based drug screen, we asked to what extent the highly
ranked candidate AD drugs were enriched in drugs already investigated in or
approved for treatment of AD.  To quantify such enrichment, we defined
\emph{rediscovery rate} (Methods, Eq.~\ref{eq:rediscovery-rate}) as a function
of the number $k$, delineating the set of top-$k$ drugs.  Rediscovery rate
compares the fraction of drugs investigated for AD in the top-$k$ drugs to
that in the bottom 1000 drugs, while taking into account what stage of
clinical study each drug is in for the treatment of AD.
Fig.~\ref{fig:screen}C shows that the rediscovery rate for AD in the top-$k$
drugs was about $2-3$-fold greater than that for the bottom drugs depending on
the value of $k$.  This finding suggests that our computational approach to AD
drug discovery yields both novel drug candidates as well as those that with a
biological rationale justifying their current status as approved AD drugs or
testing in prior human clinical trials, thereby providing a measure of
methodological validation of the network proximity-based drug screening
approach.

We then re-evaluated rediscovery rate using the same, network proximity-based,
drug ranking as above by assessing the rediscovery of drugs for the treatment
of several common indications other than for AD.  Besides AD, there was
sufficient data (i.e~at least 100 drugs approved or in phase $\ge 1$ clinical
trials) for 35 common indications (Fig.~S\ref{fig:ad-drug-rediscovery-multi}).
Interestingly, several, but not all, of these indications showed a similar
enrichment pattern to that derived earlier for AD.  We observed that approved
and experimental drugs for various cancers especially overlapped with
candidate AD drugs while those for neuropsychiatric indications like major
depressive disorder, schizophrenia and Parkinson's disease did not
(Fig.~S\ref{fig:ad-drug-rediscovery-multi}).  These findings suggest that the
novel candidate drugs for AD from our network-based screen tend to target
overlapping molecular mechanism(s) in other diseases such as cancer. 

%We also checked the agreement between our network proximity based results
%above and the results we obtained by using CMap with the same input AD risk gene
%set.  Strikingly, we found no agreement between the two even when we
%conditioned on the various cell types of the CMap data set
%(Fig.~S\ref{fig:proxim-cmap},~S\ref{fig:proxim-cmap-celltype}).  Moreover,
%different implementations of CMap yielded considerably different results
%(Fig.~S\ref{fig:cmap-cmap}).  Given these conflicting observations and the
%more realistic, systems-based, mechanistic modeling assumptions behind the
%network approach, we did not analyze the results obtained with CMap any
%further.

\subsection{Further prioritization of top ranking candidate AD drugs}

We further priorirtized the network proximity-ranked compounds based on (1)
blood-brain barrier permeability taken from the BBB database
(Table~\ref{tab:selected-drugs}) and (2) prior evidence for roles in molecular
mechanisms relevant to AD pathogenesis (below).  Our prioritization resulted
in three drugs: Arundine, Chenodiol, Cysteamine
(Table~\ref{tab:selected-drugs}).

Arundine, also known as 3,3-diindolylmethane, is the dimeric product of the
natural product indole-3-carbinol.  While Arundine has been mostly
investigated in the context of drug resistant tumors~\cite{Biersack2020}, its
close analogs have been found to cross the blood-brain barrier and protect
mice from 1-methyl-4-phenyl-1,2,3,6-tetrahydropyridine (MPTP)-induced neurotoxicity and
neurodegeneration~\cite{DeMiranda2013}.
Moreover, Arundine was shown to inhibit oxidative stress induced apoptosis in
hippocampal neuronal cells~\cite{Lee2019} and protected primary hippocampal
cell cultures from ischemia induced apoptosis and
autophagy~\cite{Rzemieniec2019}.  Interestingly, the latter effect was found
to depend on Arundine binding to the Aryl Hydrocarbon Receptor, product of the
AHR gene,~\cite{Rzemieniec2019}.

One of the Arundine targets that our analysis accounted for is the RGS4 gene
(regulator of G-protein signalling 4). RGS4 is notable because it is also
targeted by Chenodiol, another of our selected top-ranking drugs.  Chenodiol,
or Chenodeoxycholic acid, has been shown to be neuroprotective in Huntington's
disease~\cite{Keene2002}.  Moreover, Chenodiol is a bile acid, and plausibly
linked to AD based on recent evidence from our group and
others~\cite{Varma2021,Baloni2020}.

Our third selected top drugs was Cysteamine.  Cysteamine, whose targets also
include RGS4, is a derivative of the amino acid cysteine.  Cysteamine can
traverse the blood brain barrier, was approved for cystinosis (MeSH:D005128).
More recently it has been studied as a potential drug for Huntington's and
Parkinson's disease due to its neuroprotective
activity~\cite{Besouw2013,Paul2019}.

%\subsection{Molecular pathways from the selected drugs to AD risk genes}
\subsection{Experimental validation}

We evaluated the ability of Chenodiol, Arundine and Cysteamine to ameliorate
molecular abnormalities relevant to AD in cell culture-based phenotypic
assays. In these studies, we used TUDCA, the taurine conjugated epimer of
Chenodiol as prior studies (TODO: reference) have shown that TUDCA may
exert a neuroprotective role. We also chose to test the hydrochloride salt of
cysteamine (cysteamine hydrochloride) as this drug has been approved by the
FDA for the treatment of corneal cystine crystal deposits in patients with
cystinosis~\cite{Liang2021,Gahl2002}. Cysteamine hydrochloride has  also been shown to improve renal
function in pediatric patients with cystinosis~\cite{Markello1993}.

By modeling each measured dose-response relationship (Methods: Cell-based
assays, section~\ref{sec:methods-cba}), we evaluated the Bayes
factor, defined as the posterior odds of protective and neutral effect.  We
then used $2 \times \log$ of the Bayes factor to quantify the evidence for
protective effect across drugs and assays (Fig.~\ref{fig:cell-based-assays}A).
Firstly, we found strong to very strong evidence that both Arundine and TUDCA
protect iPSCs in four of five assays of LPS-induced neuroinflammation
(namely in the IL-1$\beta$, IL-6, IL-8 and TNF-$\alpha$ assay,
Fig.~\ref{fig:cell-based-assays}Af).  Secondly, moderate evidence suggested
that Arundine also protects BV2 cells in two assays of LPS-induced
neuroinflammation (IL-12p70 and IL-5 assays,
Fig.~\ref{fig:cell-based-assays}Ae).  Thirdly, moderate to very strong
evidence supported that Arundine and TUDCA, respectively, facilitate the
clearance of A$\beta$ peptide in BV2 cells (A$\beta$42 Ly assay,
Fig.~\ref{fig:cell-based-assays}Aa).  Fourthly, we found moderate evidence
that TUDCA protects H4 cells by inhibiting the release of two of three
A$\beta$ species (A$\beta$40, A$\beta$42, Fig.~\ref{fig:cell-based-assays}Ac).

In a smaller study, which included only 15 of the 33 cell-based assays, we
examined two analogs of Arundine (Table~S\ref{tab:arundine-analogs}).  We
found that the analogs did not improve the performance of Arundine
(Fig.S~\ref{fig:violin-posterior-assays-dim},
S\ref{fig:bar-posterior-assays-dim}, S\ref{fig:cell-based-assays-dim}) in the
15 assays, which notably did not include the ones Arunden performed well in
the experiments shown in Fig.~\ref{fig:cell-based-assays}A.

TODO: in Discussion, explain the benefits of our Bayesian approach: no need
for adjustment of multiple hypothesis testing.

We further studied TUDCA's potential utility for AD treatment \emph{in vivo}
in the 5xFAD trangenic mouse model of A$\beta$ pathology by probing
Nfl, a known biomarker of
axonal degeneration.  We measured plasma Nfl longitudinally at 4 sequential
time points after saline injection and found an inverted U-shaped average Nfl
trajectory, which we observed in both wild type (WT) and 5xFAD mice
(Fig.~\ref{fig:cell-based-assays}B left and center panel, respectively),
although the latter had expectedly higher baseline.  In saline-treated 5xFAD
mice, the log-transformed trajectory displayed an initial increase by $38.4 \pm 9.8 \%$ until
week 8, relative to baseline Nfl at week 0.  In TUDCA-treated 5xFAD, however,
the aforementioned relative increase was reduced by $-34.0 \pm 13.8 \%$
relative to baseline ($p=0.02$, $H_0$: zero reduction w.r.t saline).  Thus,
TUDCA effectively abolished the relative increase of Nfl flattening the
inverted U-shaped trajectory (Fig.~\ref{fig:cell-based-assays}B right).

\subsection{Hypothetical molecular pathways linking selected drugs to AD}

Next we aimed at elucidating hypothetical, AD-relevant molecular mechanisms of
the selected drugs.  We identified all of the shortest network paths from each selected
drug to each AD risk gene set in the same interactome and drug-target network that
we used for our network proximity based drug screen.  The shortest path
length, defined as the least number of gene-gene interactions connecting a
given target to a given AD risk gene, ranged typically from 2 to 4 for most
combinations of target and AD risk gene set
(Fig.~S\ref{fig:shortest-path-lengths}).

We primarily focused on paths of length 2 as these are both short and
frequently observed (Fig.~S\ref{fig:shortest-path-lengths}) thereby explaining
a drug's hypothetical beneficial effect in AD.  Such paths have the form:
$\mathrm{target} \rightarrow \mathrm{mediator} \rightarrow \mathrm{AD risk gene}$,
where \emph{mediator} is a gene transmitting the drug's presumed effect between a target
and an AD risk gene (Fig.~\ref{fig:drug-AD-genes-network} thick, colored edges).
The mediator interacting with the largest number of AD risk genes is UBC encoding Ubiquitin C
(Fig.~\ref{fig:drug-AD-genes-network} green edges, Table
Mediator UBC interacts both with the Cysteamine target
CYP3A4 and with the Chenodiol target SLC10A2.  Target CYP3A4 has several
other mediators that interact with multiple AD risk genes: ubiquitin ligases STUB1
and AMFR as well as PRKCA/PRKACA encoding subunits of protein kinase-C
(Table Fig.~\ref{fig:drug-AD-genes-network}).

Mediators for RGS4 include receptor tyrosine protein kinases EGFR and ERBB3,
as well as CALM1 (calmodulin 1), COPB1 (coatomer protein complex subunit
$\beta$), and genes encoding for various G protein subunits (Table
Fig.~\ref{fig:drug-AD-genes-network}).  Among these
mediators EGFR interacts with the most AD risk genes, including well-established
knowledge based AD risk genes of crucial relevance to AD: APP (amyloid precursor
protein, amyloid-$\beta$), MAPT (tau), SNCA---$\alpha$-synuclein
(Fig.~\ref{fig:drug-AD-genes-network}); also including AD risk genes strongly
supported by functional genomics studies (TWAS2+ AD risk gene set): PICALM and
PTK2B.

A few paths were of the form $\mathrm{target} \rightarrow \mathrm{AD risk gene}$
(path length 1).  This was the case only for target RGS4 and 4 genes in 4
relatively large AD risk gene sets (Table~S\ref{tab:nearest-ADgenes}).  Three of
these four genes---ERBB3, GNAI2, CALM1---were noted above as mediators in the
context of other, smaller, AD risk gene sets.  Moreover, RGS4 was found to be an AD
gene itself (path length$=0$) that belongs to the two largest AD risk gene sets
(Table~S\ref{tab:nearest-ADgenes}).

(Note to Madhav: although you asked me to move the next paragraph to
Discussion, I would still keep it here because I already recapitulated and
expanded on it in the Discussion.)

Taken together, these observations suggest that our candidate AD drugs might modulate
multiple key molecular mechanisms: (1)
ubiquitination, (2) protein kinase-C dependent phosphorylation, (3) receptor
tyrosine kinase-dependent and (4) G-protein dependent signal transduction, (5)
Ca$^{2+}$-calmodulin, and (5) the coatomer complex.  We shall return to the
relevance of these mechanisms to AD in the present Discussion.

\section{Discussion}

In this work, we present Chenodiol, its close analog TUDCA as well as Arundine
as promising drug repurposing opportunities to AD.  Evidence for this comes
from our network proximity analysis that ranked these drugs high, our data
from several cell based assays relevant to AD, and, for TUDCA, our \emph{in
vivo} Nfl measurements in the 5xFAD transgenic mouse model of AD.

Detailed look (Fig.~\ref{fig:drug-AD-genes-network},
Table~S\ref{tab:nearest-ADgenes}) at the network paths between Chenodiol,
Arundine and our diverse collection of AD genes reveal Regulator of G protein
signaling 4 (RGS4) as a key target linking these drugs to AD.  RGS itself may
be classified as an AD gene as it is differentially expressed in AD
brains~\cite{Taubes2021}.  Moreover, previous analyses of gene networks as well
as diverse transcriptomic and genomic data sets implicated RGS4 not only in
AD~\cite{Gns2021,Chen2022b,Talwar2014} but also in Parkinson's
disease~\cite{AhlersDannen2020} and
schizophrenia~\cite{Chowdari2002,Schwarz2018}.

RGS4 interacts with several AD genes either directly
(Table~S\ref{tab:nearest-ADgenes}) or indirectly, through mediator genes
(Fig.~\ref{fig:drug-AD-genes-network}).  These interactions implicate several
molecular pathways in the RGS4-dependent mechanism of action of Arundine and
Chenodiol/TUDCA. These include two signalling pathways via the interaction of
RGS4 with two receptor tyrosin kinase of the ErbB familys: the MAPK/ERK
pathway via Epidermal growth factor receptor, or ErbB1 (EGFR), and the
PI3K/Akt/mTOR pathway via ErbB3 (ERBB3).  EGFR and ErbB3 regulate cellular
proliferation, differentiation and maturation in the nervous system and
elsewhere and play important role in
cancer~\cite{GandulloSanchez2022,Wee2017}. While relatively little knonw about
ErbB3 in the context of neurodegenerative diseases~\cite{Ou2021}, there is
broad evidence for EGFR dysregulation in AD, Parkinson's disease and
ALS~\cite{Romano2020}.  EGFR dysregulation was found to be associated with
cognition, A$\beta$ pathology and neuroinflammation in multiple
studies~\cite{Choi2023}.  Intriguingly, our present results from cell-based
assays provide evidence that Arundine and TUDCA protect cells by inhibiting
neuroinflammation, by facilitating A$\beta$ clearance and inhibiting A$\beta$
release, which suggests these drugs exerted their protective effects, at least
partly, via acting on EGFR.  Moreover, Arundine was shown to exert
neuroprotective effect in hippocampal neurons through activating the TrkB/AKT
pathway~\cite{Lee2019} cosistent with the possibility of upstream EGFR action
by the drug.

EGFR's critical role in development and differentiation, particularly of
epithelial and glial cells, gave rise to FDA approved EGFR-targeted therapies,
including EGFR inhibitors, especially for head and neck cancers~\cite{Xu2017}.
This explains the unexpected result of our computational validation, that
drugs approved for gliobastoma, carcinomas, and various neoplasms were
rediscovered by our computational screen aimed at AD.  Thus, our results
prompt the repurposing potential of anti-cancer EGFR inhibitor drugs to AD,
which was recently pointed out~\cite{Choi2023}.

Notably, RGS4 is also well-positioned in the network to modulate cellular AD
pathologies via the COPB1 (part of the COPI complex in vesicle trafficking)
and CALM1 (calmodulin 1)~\cite{Tan2019,Nakamura2021}

Our network pharmacology approach is limited by the underlying data, in
particular by the completeness and accuracy of the gene-gene interaction
network and the drug-target network.  Arundine and Chenodiol/TUDCA may,
therefore, have exerted their neuroprotective effects seen in our cell-based
assays partly via molecular mechanisms that are missing from
Fig.~\ref{fig:drug-AD-genes-network}.  One such mechanism is Arundine
targeting AHR (aryl hydrocarbon receptor), which was previously shown to
protect mouse hippocampal neurons against ischemia~\cite{Rzemieniec2019}. 


%In this study, we combined network pharmacology analyses with cell-based
%experimental validation to discover novel drug
%repurposing opportunities in AD. We defined AD risk gene sets from diverse
%sources, including knowledge bases and DEGs from transcriptomic studies as
%well as single cell RNA-Seq data from APOE $\epsilon$4 and $\epsilon$3-iPSC derived
%microglia, astrocytes, and neurons. We also included results of the most
%recent genomic fine mapping studies (GWAS, TWAS and PWAS) in
%AD~\cite{Zhu2018,Lau2020}. Our approach exploits the rich information in
%gene-gene and drug-target networks to identify drugs that plausibly target
%causal disease pathways in AD. Using this strategy, we nominated Arundine,
%Chenodiol and Cysteamine as candidate AD drugs that act proximally to
%established AD risk genes. Moreover, using cell culture-based phenotypic assays,
%we show that both Arundine and the Chenodiol-like TUDCA exhibit protective
%effects given several molecular perturbations relevant to AD pathogenesis.  In
%spite of some adverse effects, this further validates our computational drug
%discovery methodology.
%
%Our present work
%also reveals several plausible molecular mechanisms for the potential
%therapeutic actions of these drugs in AD
%(Fig.~\ref{fig:drug-AD-genes-network}).  All three potential AD drugs target
%RGS4 (Regulator of G-Protein Signalling 4) which is differentially expressed
%in the AD brain.  RGS4 was found to be associated
%with schizophrenia~\cite{Chowdari2002}, which shares substantial
%heritability with AD~\cite{Consortium2018} and in which, similar to AD,
%autoimmune pathways have been found to play a significant
%role~\cite{Sekar2016a}.  Moreover, RGS4 is among the few hundred
%differentially expressed genes in the brain in AD
%\cite{Taubes2021}.
%
%Our network analysis also revealed the receptor tyrosine kinase EGFR as the key
%molecular link between RGS4 and crucial AD risk genes like APP (producing
%A$\beta$), MAPT (tau), SNCA ($\alpha$-synuclein) and PICALM.  Interaction
%between RGS4 and EGFR might play a role in the known crosstalk between EGFR
%and GPCRs~\cite{Wang2016c}.  Moreoveer, EGFR's driver role in several forms of
%cancer~\cite{Sigismund2018} converges with our obvervation that our present
%drug repurposing screen identifies drugs previously tested or
%approved not only for AD but also for various cancers.  While
%the network proximity based screen in this study suggested EGFR as a novel
%regulator of previously known AD risk genes, at the time of our analyses there was
%little additional evidence for this notion.  A recent GWAS also
%discovered AD-associated SNPs regulating EGFR expression, establishing EGFR as
%a novel AD risk gene~\cite{Bellenguez2022}.  Besides EGFR, our network analysis implicates
%other genes, notably CALM1 (calmodulin 1), that link the drug target gene RGS4 to
%other AD risk genes.  Binding of calmodulin to ryanodine receptor was recently found
%to be neuroprotective in AD~\cite{Nakamura2021}.
%
% We also found that SLC10A2
%and CYP3A4---targeted by Chenodiol and Cysteamine, respectively---may act
%through ubiquitination and, in the case of CYP3A4, also through PKC-dependent
%phosphorylation.  Both mechanisms have been implicated in AD
%pathogenesis~\cite{Hegde2019,Alfonso2016}.

Computational drug repurposing is a new, rapidly evolving field with a
multitude of varied approaches applicable to a growing, diverse multi-omic
datasets and curated knowledge~\cite{Pushpakom2019} with a few recent
applications to AD~\cite{Taubes2021,Fang2021}. A key strengths of our
computational drug screen is the utilization of multiple, diverse AD risk gene sets
including the most recent genomic fine mapping studies and single cell
transcriptomic data from iPSC-derived neurons, microglia and astrocytes.
Furthermore, our systems-based network pharmacology method leverages rich
information in available drug-target and AD risk gene-gene networks to identify
plausible AD drugs targeting putative causal disease pathways. Finally, our
demonstration that selected top-ranking drugs from this approach ameliorate
key molecular abnormalities in AD provides (i) further evidence of the utility
of our strategy and (ii) a biological rationale for further testing these
drugs as novel experimental AD treatments.

%\begin{enumerate}
%  \item We based our screen on a more diverse collection of AD risk gene sets than
%    either \cite{Taubes2021} or \cite{Fang2021}.  Like these authors, we also
%    used gene sets from knowledge bases and DEGs from transcriptomic studies.
%    But our work is unique in that it also leverages results of the most
%    recent genomic fine mapping studies (GWAS, TWAS and PWAS) on AD, which
%    detect causative AD risk genes with much higher specificity than DEG
%    analyses~\cite{Zhu2018,Lau2020}.
%\item We and~\cite{Fang2021} used a systems-based network pharmacology
%  approach~\cite{Guney2016,Cheng2018}, which exploits the rich information in
%  available drug-target and gene-gene networks to identify drugs that are
%  mechanistically supported to target putative disease pathways in AD.  In
%  contrast, \cite{Taubes2021} used CMap, a merely correlative
%  approach~\cite{Lamb2006}, which cannot incorporate existing knowledge on
%  the molecular pathomechanisms of AD.  Intrestingly, our present work also
%  shows that these two approaches to fully discordant, uncorrelated results.
%  Moreover, besides the aforementioned shortcoming of CMap, we find that its
%  various published implementations lead to quite different, albeit still
%  correlated, results.
%\item We aggregated results of multiple drug screens (each conditioned on a
%  different gene set) with a probabilistic rank aggregation algorithm with an
%  optimality criterion.  \cite{Fang2021} also carried multiple screens but
%  their aggregation was a less principled, ad-hoc method.  Notably, our rank
%  aggregation strategy also revealed that even some non-overlapping AD risk gene
%  sets resulted lead to the discovery of highly overlapping sets of very top
%  drugs demonstrating the robustness of our workflow.  The non-neuronal AD
%  gene sets were exception of this tendency pointing at the value of further
%  characterization of non-neuronal cells in
%  AD~\cite{Lopes2022,Mathys2019,DeStrooper2016}.
%\end{enumerate}

%Our computational validation, based on the rediscovery of drugs previously
%developed or repurposed for AD, revealed the enrichment of our top ranked
%drugs for also some common non-AD indications.  Genetic correlations between
%disease phenotypes~\cite{Consortium2018}, market-driven selection biases, and
%other factors may underlie not only this observation but also the known
%comorbidities of AD~\cite{Santiago2021}.  Such overlap between different
%drugs' indications is also consistent with the known relationship between drug
%repurposing and side effects~\cite{Ye2014}.  Further research is needed to
%dissect relationships between genetic correlations, comorbidities, side
%effects, protein and gene network topology, etc to be able to harness these and
%integrate into drug repurposing techniques.  That, in combination with
%expanding multi-celltype, multi-omic characterization of AD and its molecular
%subtypes~\cite{Neff2021}, will herald a new generation of drugs for
%Alzheimer's disease.

%Some methodological limitations of our work that must be considered include
%the inability of the AD risk gene sets utilized to capture longitudinal changes
%reflecting the long preclinical prodromal phase of disease progression. We
%also did not account for distinct pathological phenotypes of AD based on
%transcriptomic profiling of the disease or the role of interactions between
%specific cell types in influencing disease progression. 
%
%In summary, we have applied a network pharmacology approach to identify novel
%candidate AD treatments that target causal disease pathways and correct key
%molecular abnormalities associated with AD pathogenesis.

%TODOs
%\begin{itemize}
%  \item Discussion: what have I missed?
%  \item Ask Vijay or Tina to decipher the drug concentrations for cell based
%    assays
%\end{itemize}

\section{Methods}

\subsection{AD risk gene sets}

Given an AD risk gene set, the network structure of the human interactome, and
another network formed by drugs and their targets (drug-target network), the
topological proximity of a given drug to the AD risk gene set can be
quantified\cite{Guney2016} and used as evidence for repurposing that drug for
AD.  Rather than compiling a single AD risk gene set from multiple studies, we
opted for evaluating the proximity of any drug separately for multiple AD risk gene
sets and aggregate the multiple proximity values subsequently (see
Discussion).  Table~\ref{tab:genesets} summarizes the AD risk gene sets used in
this study.  The genes contained in each set are listed in Table~S1.

\subsection{Previous and novel TWAS on AD}

We obtained AD risk gene sets from multiple TWAS.  Besides collecting AD risk genes from
previous transcriptome- and proteome-wide association studies (TWAS, PWAS) we
conducted our own applying the S-MultiXcan approach~\cite{Barbeira2018} to
the two largest AD-GWAS to date~\cite{Schwartzentruber2021,Wightman2021}.
S-MultiXcan borrows statistical strength from multiple tissue types; we used a
predictive model of gene expression built on all tissues of the GTEx data set
or on only brain tissues.

We conservatively took the top 30 genes and investigated their agreement with
top gene sets from seven previous TWAS and one previous PWAS (see Table~S2
listing genes contained in each set).  Hierarchical clustering of studies
based on shared genes (Fig.~S\ref{fig:twas-clustermap}) showed a fairly low
agreement for all pairs of studies.  The most similar study pair was our
present TWAS and the fine mapping of~\cite{Jansen2019} despite several
methodological differences between these two studies, which helps validate our
novel TWAS results.  As might be expected, the least similarity was found
between the only PWAS in our collection and all the other studies.

\subsection{Drug-target network}

We queried ChEMBL~\cite{Gaulton2017} (v29, July 2021) for phase 3 and 4 drugs
and those of their targets that satisfied the following conditions: (1) the
target is a human protein (2) its average $-\log_{10}\mathrm{Activity}$ or
$p\mathrm{Activity}$ is $\ge 5$, which corresponds to $\le 10 \mu M$
following~\cite{Cheng2018}.  Average $p\mathrm{Activity}$ activity is defined
as $\sum_{i=1}^n p\mathrm{Activity}_i \ge 5$, where $n$ is the number of
activity measurements for a given drug-protein pair and $\mathrm{Activity}_i$
is the standard value of the $i$-th measurement stored in the
activities.standard\_value SQL variable.  See
\href{https://github.com/attilagk/CTNS-notebook/blob/main/2021-10-24-chembl-query/drug_target_avg_activity.sql}{our SQLite script} for implementation.

The result of these operations is a set of 25,329 drug--target pairs, which constitute
the drug-target network.

\subsection{Gene-gene interaction network}

We used the network compiled by \cite{Cheng2019}, which consists 11133 nodes
(genes) and 217160 edges (gene-gene interactions).

\subsection{Network proximity calculations}

\cite{Guney2016} developed the network proximity method for in silico drug efficacy screening.  It
quantifies the topological proximity $d$ in the interactome of the set $T$ of a drug's target genes
to the set $S$ of disease genes: 

\begin{equation}
  d(S, T) = \frac{1}{|T|}\sum_{t \in T} \min_{s \in S} d(s, t)
\end{equation}
where $d(s,t)$ is the shortest path length between disease gene $s$ and drug
target $t$.  The equation can be interpreted as the average length of shortest
paths taken from all targets of a drug to all the disease genes.

We used \href{https://github.com/attilagk/guney_code}{our modified clone} of
the original, now obsolete, implementation of \cite{Guney2016}, which allows
running the code under Python~3.

\subsection{Rediscovery rate of drugs for a disease indication}

Let $T_k$ be the set of top-$k$ ranked drugs and $B_l$ that of the bottom-$l$
ranked drugs.  We define $\bar{\phi}_{T_k}(i)$, the average clinical phase of the top-$k$ drugs for
disease indication $i$, as follows:

\begin{equation}
  \bar{\phi}_{T_k}(i) = \frac{1}{k} \sum_{d \in T_k} \phi(d, i).
  \label{eq:phase-top-k}
\end{equation}

Moreover, we define $\bar{\phi}_{B_l}(i)$, the average clinical phase of the bottom-$l$
drugs for $i$ analogously.  Given some disease indication $i$, we then define the rediscovery rate of the top-$k$
drugs relative to that of the bottom-$l$ drugs as:
\begin{equation}
  r_l(k, i) = \frac{\bar{\phi}_{T_k}(i)}{\bar{\phi}_{B_l}(i)}
  \label{eq:rediscovery-rate}
\end{equation}

The $r_l(k, i)$ notation emphasizes the facts that
\begin{enumerate}
  \item rediscovery rate depends on the disease indication $i$ at hand, and
    that
  \item for
    Fig.~\ref{fig:ad-drug-rediscovery-multi}
    we held $l=1808$ for the bottom-$l$ drugs constant but varied $k$ for the
    top-$k$ drugs.
\end{enumerate}

\subsection{CMap analyses}

Since there are several implementations of CMap (Connectivity Map) analysis was carried out with several tools.  First, we used
\url{https://clue.io/} \cite{Lamb2006} and, for each drug, averaged
``norm\_cs'',
the normalized connectivity scores.  Second, we used L1000CDS$^2$ at
\url{https://maayanlab.cloud/L1000CDS2} and the averaged ``score'' variable.
Third, we used... TODO: ask Sudhir what implementation he used.

\subsection{Rank aggregation}

As Fig.~\ref{fig:workflow} shows, we aggregated 8 lists of network
proximity-ranked drugs into a single list with rank aggregation algorithms.
We performed rank aggregation with the TopKLists R package
%\cite{Schimek2015}
several times, each time with different algorithm.  Then we assessed those
algorithm's performance with Modified Kendall Distance and found the MC3
algorithm to perform the best (Fig.~S\ref{fig:kendall-dist}).  Finally, we
used the MC3-aggregated list in our workflow.

\subsection{Cell-based assays}
\label{sec:methods-cba}

We tested whether our selected drugs can rescue molecular phenotypes relevant
to Alzheimer's disease in 33 cell-based assays grouped into nine
experiments~\cite{Varma2020,Desai2022a}. Table~S\ref{tab:protect-sign} lists
these assay and experiments and, for each assay, indicates whether the ideal,
neuroprotective effect corresponds to an increase or decrease in bioactivity.
In all assays, described in the following subsections, cells treated with
drugs were compared to either the vehicle control (VC) or lesion control.
%was performed in GraphPad Prism. Group differences were evaluated for each
%test item separately by one-way analysis of variance (ANOVA) followed by
%Dunnett's multiple comparison test versus vehicle or lesion control.

For each combination of drugs and assays we evaluated the posterior
probability $P(H_i|\mathrm{data})$ of protective, neutral and adverse drug
effect (formal hypotheses $H_1, H_0, H_2$, respectively).  To this end we
modeled dose--response relationships ($x$--$y$ data) with the Bayesian nonlinear regression
model shown in Fig.~S\ref{fig:model-2-plate} after comparing the fit of several models (not shown):
\begin{eqnarray}
  \label{eq:model-2-a}
  y &\sim& N(\mu, \sigma_\mu) \\
  \label{eq:model-2-b}
  \sigma_\mu &=& \sigma \mu \\
  \label{eq:model-2-c}
  \mu &=& \left| y_1 + \frac{y_0 - y_1}{(1 + \exp(k * (x - \mathrm{EC}_{50})} \right|
\end{eqnarray}

We interpret the parameter $y_0, y_1$ as expected bioactivity at zero and
saturating drug concentration and define fold change as $\mathrm{FC}_y = y_1 /
y_0$ (Fig.~S\ref{fig:prior-posterior-fit-H102}).  We chose a gamma prior density for the fold change as
\begin{equation}
  \label{eq:priors}
  \mathrm{FC}_y \sim \Gamma(\alpha = 5, \beta = 5),
\end{equation}
and performed Bayesian estimation of the posterior density of the fold change
using default settings of the Markov chain Monte Carlo sampler implemented by
the PyMC python library (\url{https://www.pymc.io/}).  For efficient sampling
we standardized each bioactivity data set such that 10 units corresponded to
one standard deviation (for priors and other details see
Fig.~S\ref{fig:model-2-plate} and the definition of the sample\_sigmoid\_2
function in
\url{https://github.com/attilagk/CTNS-notebook/blob/main/src/cellbayesassay.py}).
We inspected diagnostic criteria (posterior sample of fitted curves, effective
sample sizes, values of the $\hat{r}$ statistic, and Markov chain standard
errors) for the goodness of fit and excluded three poorly fitted data sets of
a total of 91 data sets.  The estimated posterior densities of fold change are
presented in Fig.~S\ref{fig:violin-posterior-assays}.

Fig.~S\ref{fig:prior-posterior-fit-H102} explains how we defined the hypothesis
of protective, neutral and adverse drug effect ($H_1, H_0, H_2$, respectively)
and how we used the posterior density estimate of fold change to calculate the
posterior probability of each of the three hypotheses.  Note that
Eq.~\ref{eq:priors} implies that the prior mean of fold change is 1, which
means we expect zero effect of any drug a priori.  Moreover, the hypotheses were
defined such that any drug has 0.2 probability of having protective or adverse
effect and thus 0.6 probability of neutral effect \emph{a
priori}~(Fig.~S\ref{fig:prior-posterior-fit-H102}).

\subsubsection{A$\beta$ clearance}

For A$\beta$ clearance assay, 20,000 BV2 cells per well (uncoated 96-well
plates) were plated out. After changing cells to treatment medium, drug
compounds were administered 1 hour before A$\beta$ stimulation (Bachem,
4061966; final concentration in well, 200 ng/ml; dilutions in medium). Cells
treated with vehicle and cells treated with A$\beta$ alone served as controls.
After 3 hours of A$\beta$ stimulation, cell supernatants were collected for
the A$\beta$ measurement and cells were carefully washed twice with PBS and
thereafter lysed in 35 $\mu$l of cell lysis buffer (50 mM tris-HCl, pH 7.4,
150 mM NaCl, 5 mM EDTA, and 1\% SDS) supplemented with protease inhibitors.
Supernatants and cell lysates were analyzed for human A$\beta$42 with the MSD
V-PLEX Human A$\beta$42 Peptide (6E10) Kit (K151LBE, Mesoscale Discovery). The
immune assay was carried out according to the manual, and plates were read on
MESO QuickPlex SQ 120.

\subsubsection{A$\beta$ secretion}

H4-hAPP cells were cultivated in Opti-MEM supplemented with 10\% FCS, 1\%
penicillin/streptomycin, hygromycin B (200 $\mu$g/ml), and blasticidin S (2.5
$\mu$g/ml).  H4-hAPP cells were seeded into 96-well plates (2 $\times$ 104
cells per well). On the next day, cells in 96-well plates were treated with
compounds, reference item [400 nM
N-[N-(3,5-difluorophenacetyl-l-alanyl)]-S-phenylglycine t-butyl ester (DAPT)],
or vehicle. Twenty-four hours later, supernatants were collected for further
A$\beta$ measurements by MSD [V-PLEX A$\beta$ Peptide Panel 1 (6E10) Kit,
K15200E, Mesoscale Discovery].

\subsubsection{Tau phosphorylation}

SH-SY5Y-hTau441(V337M/R406W) cells were maintained in culture medium [DMEM
medium, 10\% FCS, 1\% nonessential amino acids (NEAA), 1\% l-glutamine,
gentamycin (100 $\mu$g/ml), and geneticin G-418 (300 $\mu$g/ml)] and
differentiated with 10 $\mu$M retinoic acid for 5 days changing medium every 2
to 3 days. Before the treatment, cells were seeded onto 24-well plates at a
cell density of 2 $\times$ 105 cells per well on day one of in vitro culture
(DIV1). Drug compounds were applied on DIV2. After 24 hours of incubation
(DIV3), cells on 24-well plates were harvested in 60 $\mu$l of RIPA buffer [50
mM tris (pH 7.4), 1\% NP-40, 0.25\% Na-deoxycholate, 150 mM NaCl, 1 mM EDTA
supplemented with freshly added 1 $\mu$M NaF, 0.2 mM Na-orthovanadate, 80
$\mu$M glycerophosphate, protease (Calbiochem), and phosphatase
(Sigma-Aldrich) inhibitor cocktail]. Protein concentration was determined by
BCA assay (Pierce, Thermo Fisher Scientific), and samples were adjusted to a
uniform total protein concentration. Total tau and phosphorylated tau were
determined by immunosorbent assay from Mesoscale Discovery
[Phospho(Thr231)/Total Tau Kit K15121D, Mesoscale Discovery].

\subsubsection{LPS-induced neuroinflammation}

The murine microglial cell line BV2 was cultivated in Dulbecco�s modified
Eagle's medium (DMEM) medium supplemented with 10\% fetal calf serum (FCS),
1\% penicillin/streptomycin, and 2 mM l-glutamine (culture medium). For LPS
stimulation assay, 5000 BV2 cells per well (uncoated 96-well plates) were
plated out and the medium was changed to treatment medium (DMEM, 5\% FCS, and
2 mM l-glutamine). After changing cells to treatment medium, drug compounds
were administered 1 hour before LPS stimulation [Sigma-Aldrich; L6529; 1 mg/ml
stock in ddH2O; final concentration in well, 100 ng/ml (dilutions in medium)].
Cells treated with vehicle, cells treated with LPS alone, and cells treated
with LPS plus reference item (dexamethasone, 10 $\mu$M; Sigma-Aldrich, D4902)
served as controls. After 24 hours of stimulation, cell supernatants were
collected for the cytokine measurement (V-PLEX Proinflammatory Panel 1 Mouse
Kit, K15048D, Mesoscale) and cells were subjected to
3-(4,5-dimethylthiazol-2-yl)-2,5-diphenyltetrazolium bromide (MTT) assay.

\subsubsection{Neurite outgrowth and neurogenesis}

Primary hippocampal neurons were prepared from E18.5 timed pregnant
C57BL/6JRccHsd mice as previously described. Cells were seeded in
poly-D-lysine pre-coated 96-well plates at a density of 2.6 $\times$ 104
cells/well in medium (Neurobasal,
2\% B-27, 0.5 mM glutamine, 25 $\mu$M glutamate,
1\% Penicillin�Streptomycin). Directly on DIV1, the drug of interest or VC was
applied. On DIV2, 10 $\mu$M Bromodeuxyuridine (BrdU; B5002 Sigma Aldrich) was
added and cells were fixed after additional 24h. Cells were permeabilized with
0.1\% Triton-X and incubated with primary Beta Tubulin Isotype III (T8660,
Sigma Aldrich) and BrdU antibodies (MAS25$\circ$C, AHrlan-Sera Lab) overnight
at 4$\circ$C. Afterwards, cells were washed two times with PBS and incubated
with fluorescently labelled secondary antibodies and DAPI for 1.5 h at room
temperature (RT) in the dark. Cells were rinsed three times with PBS and
imaged with the Cytation 5 Multimode reader (BioTek) at 10 $\times$
magnification (six images per well). BrdU-positive cells were counted as a
marker of neurogenesis and Beta Tubulin Isotype III signal was used for
macro-based quantification of neurite outgrowth.

\subsubsection{Trophic factor withdrawal}

Primary cortical neurons from embryonic day 18 (E18) C57Bl/6 mice were
prepared as previously described. On the day of preparation (DIV1), cortical
neurons were seeded on poly-d-lysine precoated 96-well plates at a density of
3 $\times$ 104 cells per well. Every 4 to 6 days, a half medium exchange using full
medium (Neurobasal, 2\% B-27, 0.5 mM glutamine, and 1\% penicillin-streptomycin)
was carried out. On DIV8, a full medium exchange to B-27 free medium
(Neurobasal, 0.5 mM glutamine, and 1\% penicillin-streptomycin) was performed
and drug compounds were applied thereafter. The experiment was carried out
with six technical replicates per condition, and vehicle-treated cells
served as control. After 28 hours on B-27�free medium, cells were subjected to
YO-PRO/propidium iodide (PI) and MTT as well as lactate dehydrogenase (LDH)
assay.


\subsection{In vivo plasma Nfl measurements}
\label{sec:methods-nfl}

The NF-light (Neurofilament-light) ELISA 10-7001 CE from UmanDiagnostics was
used for analysis. Samples were diluted to 1:3 in assay buffer and analyzed
according to the manufacturers protocol. CSF and in vivo samples are analyzed
in single replicates only, due to limited sample volume.

In brief, after dilution, 100 $\mu$l of sample were added to the pre-coated
wells and incubated for 1.5 hours at RT with gentle agitation (800 rpm).

Wells were washed three times with assay wash buffer and 100 $\mu$l of the tracer
antibody are added. After 45 min incubation (RT, 800 rpm) wells are again
washed three times.

Thereafter 100$\mu$l of conjugate were added and incubated for 30 min (RT, 800
rpm). After 3 $\times$ washing 100 $\mu$l of TMB substrate are added to each well and
incubated for 15 min at RT.

50 $\mu$l stop reagent were added and after short gentle agitation, the plate
was
read at 450 nm (reference 620-650 nm) on the Cytation 5 multimode reader
(Biotek).

Data were evaluated in comparison to calibration curves provided in the kit and
were expressed as pg/ml plasma.

Inspecting the data we found that log-transformed fold change of Nfl level at
week 8 w.r.t week 0 was approximately
normally distributed.  We fitted the following model to the data:

\begin{equation}
  \log \frac{\mathrm{Nfl}_8}{\mathrm{Nfl}_0} = \beta_0 + \beta_1 x + \epsilon
  \label{eq:nfl}
\end{equation}

where $\mathrm{Nfl}_t$ is the Nfl level at week $t\in \{0, 8\}$; $x=0$ for
saline treated mice and $x=1$ for TUDCA treated mice; $\epsilon$ is a normally
distributed noise term, i.i.d across mouse subjects.  Our inference is based on coefficient $\beta_1$,
which mediates the effect of TUDCA; our null hypothesis is $H_0: \; \beta_1 =
0$, which means TUDCA has no effect.

\backmatter

\bmhead{Supplementary information}

If your article has accompanying supplementary file/s please state so here. 

Authors reporting data from electrophoretic gels and blots should supply the full unprocessed scans for key as part of their Supplementary information. This may be requested by the editorial team/s if it is missing.

Please refer to Journal-level guidance for any specific requirements.

\bmhead{Acknowledgements}

Acknowledgements are not compulsory. Where included they should be brief. Grant or contribution numbers may be acknowledged.

Please refer to Journal-level guidance for any specific requirements.

\section*{Declarations}

Some journals require declarations to be submitted in a standardised format. Please check the Instructions for Authors of the journal to which you are submitting to see if you need to complete this section. If yes, your manuscript must contain the following sections under the heading `Declarations':

\begin{itemize}
\item Funding
\item Conflict of interest/Competing interests (check journal-specific guidelines for which heading to use)
\item Ethics approval and consent to participate
\item Consent for publication
\item Data availability 
\item Materials availability
\item Code availability 
\item Author contribution
\end{itemize}

\noindent
If any of the sections are not relevant to your manuscript, please include the heading and write `Not applicable' for that section. 

%%===================================================%%
%% For presentation purpose, we have included        %%
%% \bigskip command. Please ignore this.             %%
%%===================================================%%
\bigskip
\begin{flushleft}%
Editorial Policies for:

\bigskip\noindent
Springer journals and proceedings: \url{https://www.springer.com/gp/editorial-policies}

\bigskip\noindent
Nature Portfolio journals: \url{https://www.nature.com/nature-research/editorial-policies}

\bigskip\noindent
\textit{Scientific Reports}: \url{https://www.nature.com/srep/journal-policies/editorial-policies}

\bigskip\noindent
BMC journals: \url{https://www.biomedcentral.com/getpublished/editorial-policies}
\end{flushleft}

\clearpage
\section{Tables}

\begin{table}[h]
\footnotesize
\begin{tabular}{rrll}
\toprule
AD risk gene set          &Size & Description & Reference  \\
\hline                     
knowledge            &  27 & curated AD risk genes from the DISEASES database & \cite{PletscherFrankild2015} \\
TWAS2+               &  32 & genes supported by $\ge 2$ AD-TWAS/PWAS & \cite{Gerring2020,Baird2021,Jansen2019,Kunkle2019,Wingo2021,Schwartzentruber2021}  \\
agora2+              &  64 & genes supported by $\ge 2$ Agora studies & https://agora.adknowledgeportal.org \\
AD DE APOE3-APOE3    & 277 & AD vs control DEGs: APOE3/APOE3 background & \cite{Taubes2021} \\
AD DE APOE4-APOE4    & 274 & AD vs control DEGs: APOE4/APOE4 background & \cite{Taubes2021} \\
APOE3-4 DE neuron    &  46 & APOE4 vs APOE3 DEGs: iPSC-derived neurons& \cite{Lin2018} \\
APOE3-4 DE astrocyte & 128 & APOE4 vs APOE3 DEGs: iPSC-derived astrocytes& \cite{Lin2018} \\
APOE3-4 DE microglia & 140 & APOE4 vs APOE3 DEGs: iPSC-derived microglia& \cite{Lin2018} \\
\bottomrule
\end{tabular}
\caption{
AD risk gene sets used as inputs to drug repurposing screens of this study.  For
the TWAS2+ gene set we combined gene sets from the prior published TWAS with
the gene set from our own TWAS (Methods).  The genes in each set are listed in
Table~S\ref{tab:genes-in-genesets}.
}
\label{tab:genesets}
\end{table}

\begin{sidewaystable}[h]
\footnotesize
\begin{tabular}{rr|ccc}
%\begin{tabular}{p{0.2\textwidth} | p{0.2\textwidth} p{0.2\textwidth} p{0.2\textwidth}}
%\begin{tabular}{l | p{3cm} p{3cm} p{3cm}}
\toprule
&                                              Name &                           Arundine &              Chenodiol &                          Cysteamine \\
\midrule
  \multirow{3}{*}{General} & Synonym & 3,3-Diindolylmethane & Chenodeoxycholic acid & Cystagon \\
  & ChEMBL ID & CHEMBL446452 &           CHEMBL240597 &                           CHEMBL602 \\
  & Approved indications &          cervical cancer, phase 3 & cholesterol gallstones &            cystinosis, eye diseases \\
\midrule
  \multirow{2}{*}{Target genes} & In present screen &    RGS4 &            RGS4, SLC10A2 &                         RGS4, CYP3A4 \\
                                & Additional targets$^\ast$ &    AHR \cite{Rzemieniec2019} &             &                          \\
\midrule
  \multirow{3}{*}{Relevance to AD}
  & Rank (present screen)  &                                  4 &                     28 &                                  51 \\
  & \emph{In vivo} evidence & \cite{DeMiranda2013,DeMiranda2014} &       \cite{Keene2002} &                \cite{Cicchetti2019} \\
\end{tabular}
\caption{
Selected drugs ranked high by the present computational screening of 2413
drugs from ChEMBL. ChEMBL Ranking was established based on a given drug�s
permeability across the blood-brain barrier, network proximity of drug-target
and AD risk genes and their prior evidence for roles in AD pathogenesis.
Notes: Targets in the present screen were taken from ChEMBL filtering for
human experiments. $^\ast$``Additional targets '' refer to those we found
manually in the biomedical literature and are not in the drug-target network
used for our screen.
%$^\ast$Cysteamine-HCl (CHEMBL1256137) was ranked 323 and targets GEMI, LMNA, PLK1, BLM, NF2L2, and TYDP1 (ChEMBL, human).
}
\label{tab:selected-drugs}
\end{sidewaystable}

\clearpage
%\newpage
\section{Figures}

\begin{figure}[h]
\includegraphics[width=1.0\textwidth]{figures/Fig1.png}
\caption{
  (A) Workflow of computational drug repurposing in Alzheimer's disease (AD).
  (B) Principle of network proximity-based drug screen for AD.  Drug 4 and 5
  are plausible candidate AD drugs because they target either directly or
  indirectly an AD risk gene (i.e a gene mechanistically involved in AD), respectively .  On the other
  hand, drugs 1--3 only target genes at least three interactions away from AD
  genes and therefore are likely not plausible candidate AD drugs.
}
\label{fig:workflow}
\end{figure}

\begin{figure}[h]
\includegraphics[]{figures/screen-rediscovery.pdf}
%\includegraphics[]{figures/proximity-ranked-drugs.pdf}
%\includegraphics[scale=0.4]{../../notebooks/2022-01-14-top-drugs/named-figure/top-bottom-ratio-top-k.pdf}
\caption{Computational drug screen and computational validation.  (A-B) Scoring and ranking 2413 drugs based on their network proximity to
  various AD risk gene sets.  Eight AD risk gene sets were input to the workflow.
  The separate drug rankings across imput AD risk gene sets were aggregated into a
  single final ranking, which is color coded (yellow: top-ranked drugs, blue:
  bottom-ranked drugs). (A) shows the top $\approx 600$ drugs while (B) only
  the top 51.  (C) Rediscovery rate $>1$ for the top-$k$ drugs suggests these
  drugs, highly ranked by network proximity, are enriched in drugs already investigated or approved
  for AD.
}
\label{fig:screen}
\end{figure}

\begin{figure}[h]
\includegraphics[]{figures/experimental-validation.pdf}
%\includegraphics[scale=0.45]{../../notebooks/2023-09-26-cell-bayes-assays/named-figure/H10-bayes-factors.pdf}
%\includegraphics[scale=0.45]{../../notebooks/2023-11-15-5xfad-nfl-gfap/named-figure/NF-L-in-vivo-treatments-tg-TUDCA.pdf}
\caption{
  Experimental validation. (A) Assessing the hypothetical neuroprotective effect of Arundine, Cysteamine
  and TUDCA in 33 cell-based assays from 9 experiments, labeled as (a)--(i).
  Note that TUDCA is the taurine conjugated epimer of Chenodiol.  Green bars
  indicate moderate to very strong evidence for protective effect as quantified by the
  Bayes factor, which is the posterior odds of protective and neutral effect.
  Blank (white) slots indicate either that the experiment was not performed on
  the drug or that model fit was poor.  (B) Effect of TUDCA on the longitudinal
  trajectory of plasma Nfl level measured \emph{in vivo} in wild type (WT) or
  transgenic 5xFAD mice.  Observe the inverted U shape of the average Nfl
  trajectory (quadratic fit) in saline treated WT and 5xFAD mice and the
  flattened trajectory in TUDCA treated 5xFAD.
}
\label{fig:cell-based-assays}
\end{figure}

\begin{figure}[h]
\includegraphics[width=1.0\textwidth]{figures/drug-target-mediator-AD_gene.pdf}
\caption{RGS4 (Regulator of G protein signaling 4) is a key target of
  Arundine, Chenodiol, and Cysteamine, suggested by network paths between these
  drugs and AD genes.  The figure depicts only three edge-long paths, which
  all have the form: drug $\xrightarrow{1}$ target $\xrightarrow{2}$  mediator
  $\xrightarrow{3}$ AD gene. Such paths are both relatively frequent and
  short to explain potential pharmacological effects on AD genes. There are
  only 4 two edge-long paths; all of those involve RGS4 as target (drug
  $\xrightarrow{1}$ RGS4 $\xrightarrow{2}$  AD gene, see
  Table~S\ref{tab:nearest-ADgenes}).  Moreover, RGS4 can itself be considered
  an AD gene based on its membership in two of the eight AD
  gene sets of this study (one edge-long paths, Table~S\ref{tab:nearest-ADgenes}).
  (A) The top, middle and bottom panel shows mediators (rows, $y$ axis)
  interacting with drug targets RGS4, SLC10A2, CYP3A4, respectively.  The
  heatmaps show both the fraction of genes in a given AD gene set (columns,
  $x$ axis) that interact with a given mediator.  The number of interacting AD
  genes is also indicated as text inside the heatmap's cells. 
  (B) All three edge-long paths between the three drugs and knowledge-based AD
  genes. For mediator genes thick, color edges mark paths formed by target
  $\rightarrow$  mediator $\rightarrow$ AD gene. For paths involving
  RGS4, the color codes for mediator genes are as follows.  Orange: EGFR
  (Epidermal growth factor receptor), purple: CALM1 (Calmodulin 1), gray: COPB1
  (Coatomer subunit beta), also gray: GNAO1 (Guanine nucleotide-binding
  protein G(o) subunit alpha).
}
\label{fig:drug-AD-genes-network}
\end{figure}

%\newpage

\clearpage
\begin{appendices}

\section*{Supplementary Material}

\setcounter{table}{0}
\makeatletter 
\renewcommand{\tablename}{Table S} % nice
%\renewcommand{\figurename}{Table S} % nice
\makeatother

\setcounter{figure}{0}
\makeatletter 
\renewcommand{\figurename}{Figure S} % nice
\makeatother

\begin{table}[h]
  \begin{tabular}{p{0.8\textwidth}}
  See file Table-S1.xlsx \\
  \end{tabular}
\caption{
  Genes of the AD risk gene sets used as inputs to the present computational drug screen.
}
\label{tab:genes-in-genesets}
\end{table}

\begin{table}[h]
  \begin{tabular}{p{0.8\textwidth}}
  See file Table-S2.csv \\
  \end{tabular}
\caption{
  The 2413 drugs ranked according to their network proximity to each of the eight
  AD risk gene sets used as input.  The drugs' final, aggregate rank is also shown
  as well as their ChEMBL ID, standard InChI, indication class, and
  blood-brain-barrier permeability taken (if available) from the BBB
  database~\cite{Meng2021}.  Moreover, the UniProt name of each drug's
  targets is also indicated.
}
\label{tab:ranked-drugs}
\end{table}

\begin{table}[h]
\begin{tabular}{cc||c|c|c}
\toprule
                       &                     &  \multicolumn{3}{c}{target}        \\
\midrule
                       &   {}                &  CYP3A4 &          RGS4 &  SLC10A2 \\
\midrule
shortest path length   & AD risk gene set         &  \multicolumn{3}{c}{AD risk genes}        \\
\midrule
                    1  &   AD DE APOE3-APOE3 &    -    &          RGS4 &    -     \\
                    1  &   AD DE APOE4-APOE4 &    -    &          RGS4 &    -     \\
\midrule
                    2  &   agora2+           &    -    &         ERBB3 &    -     \\
                    2  &   AD DE APOE3-APOE3 &    -    &         GNAI2 &    -     \\
                    2  &   AD DE APOE4-APOE4 &    -    &  CALM1; GNAI2 &    -     \\
                    2  &   APOE3-4 DE neuron &    -    &         PLCB1 &    -     \\
\bottomrule
\end{tabular}
\caption{
  AD risk genes $g$ that are either targets of Arundine, Cysteamine or
  Chenodiol, or directly
  interact with a target of those drugs.  The length of the network path
  between a drug and an AD gene is 1 for the former and 2 for the latter
  category.
}
\label{tab:nearest-ADgenes}
\end{table}

\begin{table}[h]
\begin{tabular}{lll}
\toprule
experiment & assay & ideal effect  \\
\midrule
%\multirow[t]{10}{*}{LPS neuroinflammation (BV2 cells)} & IFN-$\gamma$ & decrease \\
LPS neuroinflammation (BV2 cells) & IFN-$\gamma$ & decrease \\
 & IL-10 & decrease \\
 & IL-12p70 & decrease \\
 & IL-1$\beta$ & decrease \\
 & IL-2 & decrease \\
 & IL-4 & decrease \\
 & IL-5 & decrease \\
 & IL-6 & decrease \\
 & KC/GRO & decrease \\
 & TNF-$\alpha$ & decrease \\
\hline
A$\beta$ toxicity (primary neurons) & MTT & increase \\
\hline
A$\beta$ release (H4 cells) & A$\beta$38 & decrease \\
 & A$\beta$40 & decrease \\
 & A$\beta$42 & decrease \\
\hline
A$\beta$ clearance (BV2 cells) & A$\beta$42 SN & decrease \\
 & A$\beta$42 Ly & increase \\
\hline
Trophic factor withdrawal (primary neurons) & PI & decrease \\
 & YOPRO & decrease \\
 & MTT & increase \\
 & LDH & decrease \\
\hline
Tau phosphorylation & Tau & increase \\
 & pTau (T231) & decrease \\
 & pT/T ratio & decrease \\
\hline
Neurite outgrowth (primary neurons) & $\sum$ neurite area & increase \\
 & branch points & increase \\
 & neurogenesis & increase \\
 & longest neurite & increase \\
\hline
A$\beta$ clearance (iPSC) & pHrodo-4h & increase \\
 & supernatant & decrease \\
\hline
LPS neuroinflammation (iPSC) & IL-1$\beta$ & decrease \\
 & IL-6 & decrease \\
 & IL-8 & decrease \\
 & MTT & decrease \\
 & TNF-$\alpha$ & decrease \\
\hline
\bottomrule
\end{tabular}
\caption{
  Ideal, neuroprotective, effect for each assay.  Ideal effect is the
  direction of effect of an ideal, hypothetical, neuroprotective drug on a
  given assay.
}
\label{tab:protect-sign}
\end{table}

\begin{table}[h]
\begin{tabular}{lll}
\toprule
name & ChEMBL ID & standard InChI \\
\midrule
Arundine & CHEMBL446452 & VFTRKSBEFQDZKX-UHFFFAOYSA-N \\
C-DIM5 & CHEMBL1939127 & QCPDFNWJBQMXLI-UHFFFAOYSA-N \\
C-DIM12 & CHEMBL2376559 & LTLRXTDMXOFBDW-UHFFFAOYSA-N \\
\bottomrule
\end{tabular}
\caption{
  Arundine and its two analogs that were tested in cell-based assays.
}
\label{tab:arundine-analogs}
\end{table}

\clearpage

\begin{figure}[h]
\includegraphics[scale=0.4]{../../notebooks/2022-01-14-top-drugs/named-figure/rank-diff.pdf}
\includegraphics[scale=0.4]{../../notebooks/2022-01-14-top-drugs/named-figure/rank-diff-100.pdf}
\caption{
Difference of drug rank between the final, aggregated list and the list
resulting from each network proximity based drug screen.
}
\label{fig:rank-diff}
\end{figure}

\begin{figure}[h]
\includegraphics[scale=0.6]{../../notebooks/2021-07-01-high-conf-ADgenes/named-figure/cluster-experiments-genes.pdf}
\caption{
Similarity of TWA/PWA studies on AD in terms of shared genes.
}
\label{fig:twas-clustermap}
\end{figure}

\begin{figure}[h]
\includegraphics[scale=0.6]{../../notebooks/2021-12-02-proximity-various-ADgenesets/named-figure/jaccard-input-AD-sets.pdf}
\caption{
Eight AD risk gene sets were input to the workflow.
  Jaccard indices quantifying shared genes show that the sets are highly
  dissimilar to each other.
}
\label{fig:gset-jaccard}
\end{figure}

\begin{figure}[h]
\includegraphics[scale=0.6]{../../notebooks/2021-12-02-proximity-various-ADgenesets/named-figure/corr-coef-input-AD-sets.pdf}
\caption{
The network proximity scores of all
  screened drugs are fairly similar to each other across the different input
  AD risk gene sets, in spite of the marked dissimilarity among those.
}
\label{fig:gset-corr}
\end{figure}

\begin{figure}[h]
\includegraphics[scale=0.4]{../../results/2022-01-14-rank-aggregation/kendall-distance.png}
\caption{
Several rank aggregation algorithms' performance assessed by modified Kendall
distance.
}
\label{fig:kendall-dist}
\end{figure}

\begin{figure}[h]
\includegraphics[scale=0.4]{../../notebooks/2022-01-14-top-drugs/named-figure/top-bottom-ratio-top-k-multi-1.pdf}
\caption{
Network proximity-based rediscovery of drugs in phase 1--4 clinical trials for
AD and 35 other disease indications.  Disease indications written on top of
individual plots.  Orange ticks: drugs that are in phase 1 or more advanced
clinical study for the given indication. Blue dots: rediscovery rate
(Eq.~\ref{eq:rediscovery-rate}) for the top-$k$ drugs, ranked by network
proximity, relative to that for the bottom-$1808$ drugs.  Rediscovery rate of
$>1$ means that top-ranked drugs tend to be those that are in phase 1--4
clinical trials for the given indication; these drugs are marked by orange
symbols above the $x$ axes.
}
\label{fig:ad-drug-rediscovery-multi}
\end{figure}

\begin{figure}[h]
\includegraphics[scale=0.4]{../../notebooks/2022-09-26-selected-drugs/named-figure/shortest-path-lengths-gsets-hist_noHCl.pdf}
\caption{
  Shortest path length distribution from a given drug target (columns) to a given AD risk gene
  set (rows) across all genes in the set.
}
\label{fig:shortest-path-lengths}
\end{figure}

\begin{figure}[h]
\includegraphics[height=0.5\textheight]{../../notebooks/2023-09-13-cell-based-assays-bayes/named-figure/sigmoid-2.png}
\caption{Dependency graph of the Bayesian nonlinear regression model used to
  fit dose-response ($x$--$y$) data from cell-based assays.
  For more details see~Eq.~\ref{eq:model-2-a}-\ref{eq:model-2-c},
  Eq.~\ref{eq:priors}.  This model is named ``sigmoid 2'' in
  \url{https://github.com/attilagk/CTNS-notebook/blob/main/src/cellbayesassay.py}
  (see the definition of the sample\_sigmoid\_2 function therein).
}
\label{fig:model-2-plate}
\end{figure}

\begin{figure}[h]
\includegraphics[]{figures/prior-posterior-fit-H102.pdf}
\caption{Estimating the prior and posterior probability of the protective, neutral
  and adverse effect of a drug in a cell-based assay.  (A) Sigmoid regression
  curves sampled from the prior or posterior distribution of model parameters.
  The desired direction of drug effect (decrease in this particular assay) and
  thresholds $t_1<1$ and $t_2>1$ together define the three hypotheses of
  interest: protective ($H_1$), neutral ($H_0$) and adverse ($H_2$) drug
  effect.  The average fold change of expected bioactivity $\mathrm{FC}_y$ at
  saturating concentration, $y_1=\mathrm{FC}_y y_0$, relative to the expected activity
  at zero drug concentration, $y_0$, is marked by a dashed horizontal line
  given a sample from the prior or posterior distribution.  (B) Prior and
  posterior probability density of fold change for the same drug and assay as
  in (A).  Observe how the posterior density is shifted towards the desired
  direction and has a decreased standard deviation (vertical error bar)
  relative to the prior density.  Note that the prior density (as well as
  $t_1, t_2$) was chosen such
  that the mean fold change is 1 and such that the prior probability of
  protective ($H_1$) and adverse ($H_2$) effect is both 0.2, whereas that of
  neutral ($H_0$) effect is 0.6.
}
\label{fig:prior-posterior-fit-H102}
\end{figure}

\begin{figure}[h]
\includegraphics[scale=0.5]{../../notebooks/2023-09-26-cell-bayes-assays/named-figure/violin-posterior-pdf-legend.pdf}
\caption{Posterior probability density of drug-induced fold change in
  bioactivity across three candidate
  drugs and 33 cell-based assays. Blank (white) slots indicate either that the experiment was not
  performed on the drug or that model fit was poor.  One or more assays constitute an experiment
  (see parenthesized one-letter codes and the right center legend).  Each
  posterior density has a green, gray and red component corresponding to
  protective, neutral and adverse drug effect given the direction of the
  desired drug effect and fold-change thresholds $t_1, t_2$ (see
  Fig.~S\ref{fig:prior-posterior-fit-H102}).  The posterior mean (and standard
  deviation) of fold change is shown as a yellow circle (and yellow error
  bars).
}
\label{fig:violin-posterior-assays}
\end{figure}

\begin{figure}[h]
\includegraphics[scale=0.5]{../../notebooks/2024-01-21-cell-bayes-assays-dim/named-figure/violin-posterior-pdf-legend.pdf}
\caption{Posterior probability density of drug-induced fold change in
  bioactivity for Arundine and its two analogs.  For details see
  the caption of Fig.~S\ref{fig:violin-posterior-assays}.
}
\label{fig:violin-posterior-assays-dim}
\end{figure}

\begin{figure}[h]
\includegraphics[scale=0.5]{../../notebooks/2023-09-26-cell-bayes-assays/named-figure/H102_posteriors-barchart-e2l_textbox.pdf}
\caption{The posterior probability of protective, neutral and adverse effect
  for Arundine, Cysteamine
  and TUDCA.
}
\label{fig:bar-posterior-assays}
\end{figure}

\begin{figure}[h]
\includegraphics[scale=0.5]{../../notebooks/2024-01-21-cell-bayes-assays-dim/named-figure/H102_posteriors-barchart-e2l_textbox.pdf}
\caption{The posterior probability of protective, neutral and adverse effect
  for Arundine and its two analogs.
}
\label{fig:bar-posterior-assays-dim}
\end{figure}

\begin{figure}[h]
\includegraphics[scale=0.45]{../../notebooks/2024-01-21-cell-bayes-assays-dim/named-figure/H10-bayes-factors.pdf}
\caption{Assessing the hypothetical neuroprotective effect of Arundine and its
  two analogs in 15 cell-based assays from 5 experiments, labeled as (a)--(e).  For details see
  the caption of Fig.~\ref{fig:cell-based-assays}.
}
\label{fig:cell-based-assays-dim}
\end{figure}

\end{appendices}

\clearpage
%%===========================================================================================%%
%% If you are submitting to one of the Nature Portfolio journals, using the eJP submission   %%
%% system, please include the references within the manuscript file itself. You may do this  %%
%% by copying the reference list from your .bbl file, paste it into the main manuscript .tex %%
%% file, and delete the associated \verb+\bibliography+ commands.                            %%
%%===========================================================================================%%

%\bibliography{sn-bibliography}% common bib file
\bibliography{repurposing-ms}
%% if required, the content of .bbl file can be included here once bbl is generated
%%\input sn-article.bbl

\end{document}
