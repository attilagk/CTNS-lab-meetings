\documentclass[letterpaper]{article}
%\documentclass[12pt,letterpaper]{article}
%\setlength{\textwidth}{480pt}
%\setlength{\textheight}{630pt}
%\setlength{\voffset}{0pt}

\usepackage{amsmath, geometry, graphicx}
\usepackage{natbib}
%\usepackage{float}
\usepackage{url}
\bibliographystyle{plainnat}

% https://tex.stackexchange.com/questions/6758/how-can-i-create-a-bibliography-like-a-section
%\usepackage{etoolbox}
%\patchcmd{\thebibliography}{\section*}{\section}{}{}

\pagestyle{plain}

\title{Novel Drug Repurposing Opportunities for Alzheimer's Disease from
  Genomic Fine Mapping and Transcriptomic Disease Signatures}

\author{Attila Jones, Evan Wu, Sudhir Varma, ..., Hae Kyung Im, Madhav Thambisetty}

\date{National Institute on Aging, NIH}

\begin{document}

\maketitle

\section{Introduction}

Late onset Alzheimer's disease (AD) is a neurodegenerative disorder and the most
common form of dementia affecting a growing number of individuals in ageing
societies but effective treatment is still
missing~\citep{Bondi2017,Masters2015}.  This is, in great part,
due to our poor understanding of the disease's early molecular etiology,
which potentially involves a multitude of cellular pathways and cell types,
and which becomes gradually confounded by a large number of compensatory
and degenerative pathologies over the decades-long disease
progress~\citep{DeStrooper2016}.

Genome-wide association studies (GWAS) on AD discovered $\approx 40$ risk loci
to date~\citep{Jansen2019,Kunkle2019,Schwartzentruber2021,Wightman2021}
suggesting polygenic, or at least oligogenic~\citep{Zhang2020}, mechanism
involving 100 or somewhat fewer causal genes.  Unfortunately, translating GWAS to
efficient AD drugs has been prevented so far by several obstacles: (1) the
identity of causal genes within risk loci is unclear except for a few like
Apoe tODO: ref; (2) GWAS, in general, cannot detect all causal genes; (3) not
all causal AD genes may be
druggable~\citep{Cao2014,Lau2020,Floris2018,Finan2017}.  The first obstacle
was addressed by several transcriptome- or proteome-wide association
studies (TWAS/PWAS) on
AD~\citep{Jansen2019,Kunkle2019,Gerring2020,Baird2021,Schwartzentruber2021,Wingo2021},
which combined Mendelian randomization~\citep{DaveySmith2014,Lawlor2008},
imputation techniques~\citep{Barbeira2018,Barbeira2019a}, or Bayesian
colocalization~\citep{Wen2017} with large transcriptomic or proteomic data
sets.  However, the causal genes discovered by these recent TWAS/PWAS remain
until now without experimental validation.  Also, it is unclear how concordant
discoveries these methodologically different studies resulted in.

Analysis of differentially expressed genes (DEGs) from transcriptomic and
proteomic profiling of diseased and control samples is a complementary
approach to GWAS that has been extensively applied to AD at various disease
stages, tissues and cell types TODO: ref.  Such studies implicated hundreds to
thousands of genes in AD but nominating drug targets is hindered by the
difficulty of dissecting causal or driver genes from the large number of
correlated genes.  With that aim a number of studies in the AMP-AD
program~\citep{Greenwood2020} nominated hundreds of putative
targets~(https://agora.adknowledgeportal.org) based in
great part on DEG discovery.  Genetically regulated components of gene
expression may confound transcriptomic profiling but this received little attention
except for a few studies that controlled for the genotype of
APOE~\citep{Taubes2021,Lin2018,RobertsJackson2021}, whose $\epsilon 4$
isoform is the single strongest genetic risk factor for AD.

In this study we first conduct a TWAS using~\citep{Barbeira2018} on AD using the
latest, largest GWAS and transcriptomic data sets and meta analyze that with
previous TWAS results as well as with DEGs from transcriptomic profiling with
known APOE genotypes.  We then collect DEGs from APOE-controlled
transcriptomic studies on AD.  Building on these multiple gene sets we then
address the obstacles to translating GWAS and transcriptomic results on AD by
network proximity based drug repurposing~\citep{Cheng2018}.  In contrast to
better known, correlative, approaches like the connectivity map,
CMap~\citep{Lamb2006}, the network proximity approach models a drug's effect
on disease genes mechanistically by considering the topological details of the
human interactome, i.e the network of protein-protein and gene-gene
interactions~\citep{Guney2016,Barabasi2011}.  By evaluating network proximity
to AD genes for each of $\approx 2400$ approved or phase 3 drugs, we nominate
TODO: $X, Y, Z$ for repurposing to AD.

\section{Results}

\subsection{Meta-analysis of present and previous TWAS results}

Given an AD gene set, the network structure of the human interactome, and a
network formed by drugs and their targets (drug-target network), topological
proximity of a given drug to the AD gene set can be
quantified\citep{Guney2016} and used as evidence for repurposing that drug for
AD.  Rather than compiling a single AD gene set from multiple studies, we
opted for evaluating the proximity of any drug separately for multiple AD gene
sets and aggregate the multiple proximity values subsequently (see Discussion).

We set out by obtaining AD gene sets from multiple TWAS.  Besides collecting
AD genes from previous TWA and PWA studies (TODO: supplementary table) we
conducted our own applying the S-MultiXcan approach~\citep{Barbeira2018} to
the largest AD-GWAS to date~\citep{Schwartzentruber2021}.  TODO: describe
here the results of Haky et al: number of significant genes (check with Haky
et al on FDR); find interesting novel genes based on their membership in
AD-relevant pathways.

\begin{figure}
\includegraphics[scale=0.6]{../../notebooks/2021-07-01-high-conf-ADgenes/named-figure/cluster-experiments-genes.pdf}
\caption{
Similarity of TWA/PWA studies on AD in terms of shared genes.
  TODO: color genes supported by multiple studies differently
}
\label{fig:twas-clustermap}
\end{figure}

We conservatively took the top 30 genes and investigated their agreement with
top gene sets from seven previous TWAS and one previous PWAS.  Hierarchical clustering of studies
based on shared genes (Fig.~\ref{fig:twas-clustermap}) showed a fairly low
agreement for all pairs of studies (TODO: mean and stdev Jaccard index).  The
most similar study pair was our present TWAS and the fine mapping
of~\cite{Jansen2019}, which helps validate our novel TWAS results.  As might
be expected, the least similarity was found between the only PWAS in our collection
and all the other studies.

\subsection{Divergent AD gene sets and drug rankings}

We aggregated the nine TWAS/PWAS studies analyzed above by taking genes
supported by two or more studies.  The resulting set of 29 genes, designated
as TWAS2+, was used as input to a network proximity based computational
repurposing screen of 2413 drugs in phase 3 or 4 (approved) for some
indication (TODO: Methods).  The standardized proximity score, $Z$, reached
its minimum at $-6.49$ at drugs with the sole target ACE (TODO: supplementary
table) providing strong evidence ($p=4\times 10^{-11}$) for the repurposing
potential of these drugs to AD given the input TWAS2+ AD gene set.

\begin{table}
\footnotesize
\begin{tabular}{cll}
name     & description & reference  \\
\hline
TWAS2+   & genes supported by $\ge 2$ AD-TWAS/PWAS & (TODO: Supp Table)  \\
knowledge& curated AD genes from the DISEASES database & \cite{PletscherFrankild2015} \\
agora2+  & genes supported by $\ge 2$ Agora studies & https://agora.adknowledgeportal.org \\
AD DE APOE3-APOE3 & AD vs control DEGs: APOE3/APOE3 background & \cite{Taubes2021} \\
AD DE APOE4-APOE4 & AD vs control DEGs: APOE4/APOE4 background & \cite{Taubes2021} \\
APOE3-4 DE neuron & APOE4 vs APOE3 DEGs: iPSC-derived neurons& \cite{Lin2018} \\
APOE3-4 DE astrocyte & APOE4 vs APOE3 DEGs: iPSC-derived astrocytes& \cite{Lin2018} \\
APOE3-4 DE microglia & APOE4 vs APOE3 DEGs: iPSC-derived microglia& \cite{Lin2018} \\
\end{tabular}
\caption{
Input AD gene sets to drug repurposing screens of this study.
}
\label{tab:genesets}
\end{table}

However, we expected our TWAS2+ gene set far from complete in terms of
causal AD genes, which likely resulted in many false discoveries in the
drug screen above.  Therefore, we repeated the screen given several alternative AD gene
sets (Table~\ref{tab:genesets}) with a dual goal: (1) to assess the
sensitivity of the approach to its input AD gene set
(Fig.~\ref{fig:divergent-results}) and (2) to produce multiple ranked lists of
drugs for subsequent rank aggregation (Table~\ref{tab:top-drugs}).

\begin{figure}
\includegraphics[scale=0.5]{../../notebooks/2021-12-02-proximity-various-ADgenesets/named-figure/jaccard-index-input-AD-sets.pdf}
\includegraphics[scale=0.5]{../../notebooks/2021-12-02-proximity-various-ADgenesets/named-figure/corr-coef-input-AD-sets.pdf}
\caption{
  Sensitivity of drug repurposing score to input AD gene set. (A) Pairwise
  similarity of AD gene sets used as inputs for network proximity based drug
  screen.  (B) Correlation of the corresponding proximity scores.
TODO: remove DESudhir, since that wasn't used for the aggregated drug ranking;
TODO: find a better colormap
}
\label{fig:divergent-results}
\end{figure}

\subsection{Aggregated drug ranking and computational validation}

\begin{table}
\footnotesize
\begin{tabular}{cccc}
drug     & aggregate rank & targets & previous indication(s)  \\
\hline
TODO & TODO & TODO & TODO \\
TODO & TODO & TODO & TODO \\
TODO & TODO & TODO & TODO \\
TODO & TODO & TODO & TODO \\
\end{tabular}
\caption{
Top drugs resulted by the aggregation of multiple network proximity based
screens (cf.~Table~\ref{tab:genesets})
}
\label{tab:top-drugs}
\end{table}

\begin{figure}
\includegraphics[scale=0.6]{../../notebooks/2022-01-14-top-drugs/named-figure/top-bottom-ratio-top-k.pdf}
\caption{
Validation: rediscovery of existing phase 1--4 drugs for AD.
}
\label{fig:ad-drug-rediscovery}
\end{figure}

\subsection{Experimental validation}

\section{Discussion}

Computational drug repurposing is a new, rapidly evolving field without
established standards and with a multitude of varied approaches applicable to
a growing, diverse multi-omic body of data and curated
knowledge~\citep{Pushpakom2019}.  During our present investigations two drug
repurposing studies on AD were published: \cite{Fang2021} nominated sildenafil
while~\cite{Taubes2021} bumetanide.  These two works differed both in their
definition of AD genes as well as in the approach used to score and rank
drugs: \cite{Taubes2021} combined APOE genotype-specific DEGs with
CMap~\citep{Lamb2006}, while \cite{Fang2021} used DEGs without known APOE
genotype, as well as knowledge-based AD genes, as inputs to a network
proximity based drug repurposing screen~\citep{Cheng2018}.  In this paper we
show that, while the input gene sets have considerable effect on network
proximity based drug scores, the latter still show significant correlation.
Surprisingly, we also find that, even given the same input gene set, the
network proximity approach taken by \cite{Fang2021} and CMap used by
\cite{Taubes2021} leads to fully discordant, uncorrelated results.

Selection of AD genes: TWAS and APOE controlled DEGs
\begin{itemize}
\item S-MultiXcan borrows statistical strength from multiple tissue types; we used
a predictive model of gene expression built on the latest version to date of
the entire GTEx data set
\item we found a relatively close agreement between our TWAS and that of
\cite{Jansen2019} despite
several methodological differences between these two studies 
\end{itemize}

Network proximity vs CMap

Enrichment of top drugs in phase 1-3 AD and other indications.  Genetic
correlations, comorbidities of AD~\citep{Consortium2018,Santiago2021}.

\section{Methods}

\subsection{AD gene sets}

\subsection{TWAS on AD using PrediXcan}

\subsection{Drug-target network}

\subsection{Network proximity calculations}

\subsection{CMap analyses}

\subsection{Rank aggregation}

\bibliography{repurposing-ms}

\section*{Supplementary Material}

\setcounter{table}{0}
\makeatletter 
\renewcommand{\figurename}{Supplementary Table} % nice
\makeatother

\setcounter{figure}{0}
\makeatletter 
\renewcommand{\figurename}{Supplementary Figure} % nice
\makeatother

\begin{figure}[p]
\includegraphics[scale=0.4]{../../results/2022-01-14-rank-aggregation/kendall-distance.png}
\caption{
Several rank aggregation algorithms' performance assessed by modified Kendall
distance.
}
\label{fig:rank-diff}
\end{figure}

\begin{figure}[p]
\includegraphics[scale=0.4]{../../notebooks/2022-01-14-top-drugs/named-figure/rank-diff.pdf}
\includegraphics[scale=0.4]{../../notebooks/2022-01-14-top-drugs/named-figure/rank-diff-100.pdf}
\caption{
Difference of drug rank between the final, aggregated list and the list
resulting from each network proximity based drug screen.
}
\label{fig:rank-diff}
\end{figure}

\begin{figure}[p]
\includegraphics[scale=0.4]{../../notebooks/2022-01-14-top-drugs/named-figure/top-bottom-ratio-top-k-multi-1.pdf}
\caption{
Rediscovery of existing phase 1--4 drugs for AD and other indications.
}
\label{fig:ad-drug-rediscovery-multi}
\end{figure}

\end{document}

\begin{table}
  \footnotesize
\begin{tabular}{cc}
  gene       & supporting TWAS \\
  \hline
  CR1        & 6\\
  PVR        & 5\\
  BIN1       & 5\\
  MS4A6A     & 4\\
  CLU        & 4\\
  PTK2B      & 4\\
  ABCA7      & 3\\
  ACE        & 3\\
  TOMM40     & 3\\
  HLA-DRB1   & 3\\
  CLPTM1     & 3\\
  MS4A4A     & 3\\
  HLA-DRA    & 2\\
  PICALM     & 2\\
  HLA-DQA1   & 2\\
  APOC4      & 2\\
  CASS4      & 2\\
  PVRIG      & 2\\
  NECTIN2    & 2\\
  APOC1      & 2\\
  EPHA1      & 2\\
  PRSS36     & 2\\
  SPI1       & 2\\
  CEACAM19   & 2\\
  CCDC6      & 2\\
  TAS2R60    & 2\\
  TSPAN14    & 2\\
  KAT8       & 2\\
  APOE       & 2\\
\end{tabular}
\caption{
Genes supported by multiple TWAS and fine mapping studies.  TODO: more info
needs to be added to this table: pathways involving genes, experimental
studies
}
\label{tab:twas-genes}
\end{table}

