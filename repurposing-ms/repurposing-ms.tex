\documentclass[letterpaper]{article}
%\documentclass[12pt,letterpaper]{article}
%\setlength{\textwidth}{480pt}
%\setlength{\textheight}{630pt}
%\setlength{\voffset}{0pt}

\usepackage{amsmath, geometry, graphicx}
\usepackage{natbib}
%\usepackage{float}
\usepackage{url,hyperref}
\usepackage{booktabs}
\bibliographystyle{plainnat}

% https://tex.stackexchange.com/questions/6758/how-can-i-create-a-bibliography-like-a-section
%\usepackage{etoolbox}
%\patchcmd{\thebibliography}{\section*}{\section}{}{}

\pagestyle{plain}

\title{Novel Drug Repurposing Opportunities for Alzheimer's Disease from
  Genomic Fine Mapping and Transcriptomic Disease Signatures}

  \author{Attila Jones, Evan Wu, Sudhir Varma, Tina(?), ..., Hae Kyung Im, Madhav Thambisetty}

\date{National Institute on Aging, NIH}

\begin{document}

\maketitle

\section{Introduction}

Hello World!

Late onset Alzheimer's disease (AD) is a neurodegenerative disorder and the most
common form of dementia affecting a growing number of individuals in ageing
societies for which effective treatments are
unavailable~\citep{Bondi2017,Masters2015}.  Repurposing drugs approved for other indications is
a promising approach  towards identification of effective disease-modifying AD
treatments~\citep{Pushpakom2019,Fang2021,Taubes2021}.  One of the key
milestones of the National Plan for Alzheimer’s Disease is to ``Initiate
research programs for translational bioinformatics and network pharmacology to
support rational drug repositioning''~\citep{Aging}.

However, drug discovery in AD is hindered by our limited understanding of the
early etiological triggers of the disease, which potentially implicate a
multitude of cellular pathways and cell types, and which becomes gradually
confounded by a large number of compensatory and degenerative pathologies over
the decades-long disease progression~\citep{DeStrooper2016}.  In this work we
attempted to address this challenge in two ways.  Firstly, we derive plausible
mechanistic insights into AD from previous genome- and transcriptome- and
proteome-wide association studies (GWAS, TWAS,
PWAS)~\citep{Jansen2019,Kunkle2019,Gerring2020,Baird2021,Schwartzentruber2021,Wightman2021,Wingo2021}
comparing AD and control samples, including analyses of differentially
expressed genes expression (DEGs) in APOE3/3 versus APOE4/4 individuals as
well as from comparison of DEGs in iPSC-derived neurons, astrocytes and
microglia from APOE3/3 versus APOE4/4
individuals~\citep{Taubes2021,Lin2018}~(\ref{fig:workflow}A).

\begin{figure}
\hspace{0.05\textwidth}A\hspace{0.7\textwidth}B

\includegraphics[scale=0.85]{../../../figures/by-me/repurposing-study-desing/repurposing-study-design.pdf}
\includegraphics[scale=0.85]{../../../figures/by-me/drug-target-disgene-network/drug-target-disgene-network.pdf}
\caption{
  A) Workflow of computational drug repurposing to Alzheimer's disease (AD).
  B) Principle of network proximity-based drug screen for AD.  Drug 4 and 5
  are likely efficient for AD because they target either directly or
  indirectly an AD gene (i.e a gene mechanistically involved in AD).  On the other
  hand, drugs 1--3 only target genes at least three interactions away from AD
  genes and therefore are likely not efficient for AD.
}
\label{fig:workflow}
\end{figure}

Secondly, we exploit rich, mechanistic information represented by two
interconnected networks: (1) the network formed by all drugs and their target
genes (drug-target network) and (2) the human gene-gene interaction network
(known as the interactome), whose Alzheimer's disease
module~\citep{Barabasi2011} is defined by AD genes, a set of genes
mechanistically involved in AD, Fig~\ref{fig:workflow}B.  Given these
networks, we estimate the efficacy of each drug for AD by evaluating network
proximity~\citep{Guney2016}, which is based on the drug's network topological
distance from the AD module and thus estimates the drug's effect on AD genes.
Network proximity was shown~\citep{Cheng2018} to be a powerful, systems-level,
alternative to wider used, but merely correlative, approaches to computational
drug repurposing such as the connectivity map, CMap~\citep{Lamb2006}.  We
present the results and computational validation of our network proximity
based drug screen as well as experimental validation of three selected top
ranking drugs.

\section{Results}

\subsection{Computational drug screen: evaluating network proximity}

Drugs with low network proximity score for any given set of disease genes were
previously shown to be good candidates for repurposing to that
disease~\citep{Cheng2018}.  We performed a network proximity based drug
repurposing screen for AD on 2413 drugs from ChEMBL either approved by the FDA
or in phase 3 clinical trials (Fig.~\ref{fig:workflow}A).  For each drug we
computed eight values of network proximity given eight sets of AD genes that
we derived from eight complementary information sources (one curated knowledge
based source and seven multi-omic sources, Table~\ref{tab:genesets}).  This
resulted in eight distinct lists of the same 2413 drugs ranked according to
increasing network proximity (Fig.~\ref{fig:workflow}) so that top ranking
drugs could be considered for repurposing.

\begin{table}
\footnotesize
\begin{tabular}{cll}
Name     & Description & Reference  \\
\hline
TWAS2+   & genes supported by $\ge 2$ AD-TWAS/PWAS & See caption  \\
knowledge& curated AD genes from the DISEASES database & \cite{PletscherFrankild2015} \\
agora2+  & genes supported by $\ge 2$ Agora studies & https://agora.adknowledgeportal.org \\
AD DE APOE3-APOE3 & AD vs control DEGs: APOE3/APOE3 background & \cite{Taubes2021} \\
AD DE APOE4-APOE4 & AD vs control DEGs: APOE4/APOE4 background & \cite{Taubes2021} \\
APOE3-4 DE neuron & APOE4 vs APOE3 DEGs: iPSC-derived neurons& \cite{Lin2018} \\
APOE3-4 DE astrocyte & APOE4 vs APOE3 DEGs: iPSC-derived astrocytes& \cite{Lin2018} \\
APOE3-4 DE microglia & APOE4 vs APOE3 DEGs: iPSC-derived microglia& \cite{Lin2018} \\
\end{tabular}
\caption{
AD gene sets used as inputs to drug repurposing screens of this study.  For
the TWAS2+ gene set we combined gene sets from the TWAS of
\cite{Gerring2020,Baird2021,Jansen2019,Kunkle2019,Wingo2021,Schwartzentruber2021}
as well as from our own TWAS (Methods).  The genes in each set are listed in
Table~S\ref{tab:genes-in-genesets}.
}
\label{tab:genesets}
\end{table}

Although the eight input AD gene sets had little overlap with each other
(Fig.~\ref{fig:screen}A), the resulting eight vectors of network proximity had
markedly high correlations (Fig.~\ref{fig:screen}B) so that the eight lists of
drug rankings shared substantial similarity (Fig.~\ref{fig:screen}C right).
This suggests that even though the input AD gene sets share only a few genes,
the corresponding network modules are close enough to one another in network
topological space to exhibit similar network proximity to any given drug.

%TODO: Discussion how does it help validate our approach
%the input AD gene sets, although hardly overlapping, are still close enough to
%each other in the network's topology to result in similar drug rankings
%(Fig.~\ref{fig:screen}C right), which helps validate our approach.

\begin{figure}
\hspace{0.05\textwidth}A\hspace{0.7\textwidth}B

\includegraphics[scale=0.5]{../../notebooks/2021-12-02-proximity-various-ADgenesets/named-figure/jaccard-input-AD-sets.pdf}
\includegraphics[scale=0.5]{../../notebooks/2021-12-02-proximity-various-ADgenesets/named-figure/corr-coef-input-AD-sets.pdf}

\hspace{0.05\textwidth}C\hspace{0.5\textwidth}D

\includegraphics[scale=0.5]{../../notebooks/2022-01-14-top-drugs/named-figure/separate-aggregate-ranks-heatmap.pdf}
\includegraphics[scale=0.5]{../../notebooks/2022-01-14-top-drugs/named-figure/separate-aggregate-ranks-heatmap-topk.pdf}

\caption{Scoring and ranking 2413 drugs based on their network proximity to
  various AD gene sets.  A) Eight AD gene sets were input to the workflow.
  Jaccard indices quantifying shared genes show that the sets are highly
  dissimilar to each other.  B) The resulting network proximity scores of all
  screened drugs, in contrast, are fairly similar across the different input
  AD gene sets.  C-D) The separate drug rankings across imput AD gene sets
  were aggregated into a single final ranking, which is color coded (yellow:
  top drugs, blue: bottom drugs).  Note that the area delimited by red
  rectangle in C) is expanded into D).
}
\label{fig:screen}
\end{figure}

Assuming our eight input AD gene sets were all equally relevant to Alzheimer's
disease, we aggregated the corresponding drug rankings giving uniform weight
to all gene sets.  This yielded a single final ranked list of the top 605
drugs (Fig.~\ref{fig:screen}C, Table~S\ref{tab:genes-in-genesets}).  The very top
ranked ($\le 30$) drugs on the final list also tended to be top ranked on five
out of eight input rankings (Fig.~\ref{fig:screen}D,
Fig.~S\ref{fig:rank-diff}).

\subsection{Computational validation}

To validate our network based screen we asked to what extent the top ranked
drugs were enriched in drugs already investigated or approved for AD.  To that
end we examined the rediscovery rate for the top-$t$ drugs relative to that
for the bottom-$1808$ drugs (see Methods for definitions).
Fig.~S\ref{fig:ad-drug-rediscovery} shows that rediscovery rate for the
top-$t$ drugs was about $2-3$-fold relative to that for the bottom drugs
depending on the threshold $t$.  This finding validates our approach.
Several, but not all, other common indications showed a similar enrichment
pattern (Fig.~S\ref{fig:ad-drug-rediscovery-multi}).  Interestingly, our
approach tended to rediscover drugs developed for various forms of neoplasm
and cancer while failed to rediscover drugs for neuropsychiatric indications
like Parkinson's disease, schizophrenia and major depressive disorder
(Fig.~S\ref{fig:ad-drug-rediscovery-multi}). % TODO: discuss result in Discussion

We also checked the agreement between our network proximity based results
above and the results we obtained by using CMap with the same input AD gene
set.  Strikingly, we found no agreement between the two even when we
conditioned on the various cell types of the CMap data set
(Fig.~S\ref{fig:proxim-cmap},~S\ref{fig:proxim-cmap-celltype}).  Moreover,
different implementations of CMap yielded considerably different results
(Fig.~S\ref{fig:cmap-cmap}).  Given these conflicting observations and the
more realistic, systems-based, mechanistic modeling assumptions behind the
network approach, we did not analyze the results obtained with CMap any
further.

\subsection{Selected top ranking drugs}

We selected four top ranking compounds for further investigation
(Table~\ref{tab:selected-drugs}) based on evidence from \emph{in vivo} animal
experiments or cell based assays that suggest applicability to AD and other
neurodegenerative diseases (below).

Arundine, also known as 3,3-diindolylmethane, is the dimeric product of the
natural product indole-3-carbinol.  While Arundine has been mostly
investigated in the context of drug resistant tumors~\citep{Biersack2020}, its
close analogs have been found to cross the blood-brain barrier and protect
mice from MPTP-induced neurotoxicity and
neurodegeneration~\citep{DeMiranda2013} by suppressing
gliosis~\citep{DeMiranda2014}, a key hallmark of AD~\citep{DeStrooper2016}.
Moreover, Arundine was shown to inhibit oxidative stress induced apoptosis in
hippocampal neuronal cells~\citep{Lee2019} and protected primary hippocampal
cell cultures from ischemia induced apoptosis and
autophagy~\citep{Rzemieniec2019}.  Interestingly, the latter effect was found
to depend on Arundine binding to the Aryl Hydrocarbon Receptor, product of the
AHR gene,~\citep{Rzemieniec2019} which was found to be a target of Arundine in
mouse but has not yet been confirmed in humans and consequently was not taken
into account in our human-centered network proximity based drug screen.
Therefore, the Arundine-AHR interaction represents an additional piece of
evidence for the potential utility of Arundine in AD treatment.

One of the Arundine target that our analysis did account for is the RGS4 gene
(regulator of G-protein signalling 4). RGS4 was found to be associated with
schizophrenia~\citep{Chowdari2002}, one of the best genetically characterized
neuropsychological disorder, which shares substantial heritability with
AD~\citep{Consortium2018} and in which, like in AD, autoimmune pathways have
been found to play significant role~\citep{Sekar2016a}.  RGS4 is notable
because it is also targeted by Chenodiol, another of our selected top-ranking
drugs.  Chenodiol, or Chenodeoxycholic acid, has been shown to be
neuroprotective in Huntington's disease~\citep{Keene2002}.  Moreover,
Chenodiol is a bile acid, and thus plausibly linked to
AD based on recent evidence by us and others~\citep{Varma2021,Baloni2020}.

Our third and fourth selected top drugs were Cysteamine and Cysteamine-HCl.
Cysteamine, whose targets also include RGS4, a derivative of the amino acid
cysteine.  Cysteamine can traverse the blood brain barrier, was approved for
cystinosis and eye diseases, and more recently it has been extensively
investigated as a potential drug for neurodegenerative diseases, specifically
Huntingon's and Parkinson's diseases due to its neuroprotective
activity~\citep{Besouw2013,Paul2019}.  Previous evidence for Cysteamine's
potential utility for AD includes its favorable effect on the impaired
cognitive abilities of APP-Psen1 mice~\citep{Cicchetti2019}.

\begin{table}
\footnotesize
\begin{tabular}{p{0.2\textwidth} | p{0.2\textwidth} p{0.2\textwidth} p{0.2\textwidth}}
%\begin{tabular}{l | p{3cm} p{3cm} p{3cm}}
\toprule
                                              Name &                           Arundine &              Chenodiol &                          Cysteamine \\
\midrule
                                  Alternative form &
                                                   &                        & Cysteamine-HCl$^\ast$ \\
                                           Synonym &               3,3-Diindolylmethane &  Chenodeoxycholic acid &                            Cystagon \\
                                         ChEMBL ID &                       CHEMBL446452 &           CHEMBL240597 &                           CHEMBL602 \\
                              Approved indications &          cervical cancer, phase 3 & cholesterol gallstones &            cystinosis, eye diseases \\
\midrule
\multicolumn{4}{l}{\scshape Target genes} \\
                                      %Gene targets &                                    &                        &                                     \\
                                     ChEMBL, human &                        RGS4, GPR84 &            RGS4, NTCP2 &                         RGS4, CP3A4 \\
%                        ChEMBL, human (alt. form) &                                    &                        & GEMI, LMNA, PLK1, BLM, NF2L2, TYDP1 \\
                                             Other &                                AHR &                        &                                     \\
\midrule
\multicolumn{4}{l}{\scshape AD-relevance, present study} \\
                             Rank among 2413 screened drugs &                                  4 &                     28 &                                  51 \\
\midrule
\multicolumn{4}{l}{\scshape AD-relevance, previous studies} \\
                                  In vivo evidence & \cite{DeMiranda2013,DeMiranda2014} &       \cite{Keene2002} &                \cite{Cicchetti2019} \\
                                 In vitro evidence &      \cite{Lee2019,Rzemieniec2019} &                        &          \cite{Besouw2013,Paul2019} \\
\bottomrule
\end{tabular}
\caption{
Selected drugs ranked high in our present computational screen of 2413 drugs
from ChEMBL.  Notes: Targets were taken from ChEMBL filtering for human
experiments. ``Other'' targets refer to those we found manually in the
biomedical literature. $^\ast$Cysteamine-HCl (CHEMBL1256137) was ranked 323
and targets GEMI, LMNA, PLK1, BLM, NF2L2, and TYDP1 (ChEMBL, human).
}
\label{tab:selected-drugs}
\end{table}

%\subsection{Molecular pathways from the selected drugs to AD genes}
\subsection{Hypothetical molecular pathways linking selected drugs to AD}

Next we aimed at elucidating hypothetical, AD-relevant molecular mechanisms of
the selected drugs.  We searched for all shortest pathways from each selected
drug to each AD gene set in the same interactome and drug-target network that
we used for our network proximity based drug screen.  The shortest path
length, defined as the least number of gene-gene interactions connecting a
given target to a given AD gene, ranged typically from 2 to 4 for most
combinations of target and AD gene set (Fig.~S\ref{fig:shortest-path-lengths},
Fig.~S\ref{fig:shortest-path-bar-knowledge}-S\ref{fig:shortest-path-bar-APOE3-4
DE microglia}).  However, in the case of RGS4, targeted by all three selected
drugs (Table~\ref{tab:selected-drugs}), the shortest path length to some AD
genes was 1 meaning that RGS4 directly interacts with these genes: with ERBB3
in the agora2+ set, with GNAI2 in the AD DE APOE3-APOE3 set, with CALM1 and
GNAI2 in the AD DE APOE4-APOE4 set, and with PLCB1 in the APOE3-4 DE neuron
set (Table~S\ref{tab:nearest-ADgenes},
Fig.~S\ref{fig:shortest-path-bar-knowledge}-S\ref{fig:shortest-path-bar-APOE3-4
DE microglia}).  Moreover, target RGS4 is part of two of the largest AD gene
sets of our study, AD DE APOE3-APOE3 and AD DE APOE4-APOE4, which corresponded
to shortest path length of zero (Table~S\ref{tab:nearest-ADgenes}).

As noted above, for most AD gene sets---especially for the smaller sized,
higher confidence sets---the shortest paths between targets and AD genes were
found to be of length 2 or more implying mediator genes along the paths from
targets to AD genes (Fig.~S\ref{fig:shortest-path-lengths}).  Inspecting
$>100$ paths of length 2 connecting targets revealed that the most frequent mediator genes
are UBC, EGFR, CALM1, STUB1, PRKCA, and PRKACA (TODO: make histogram).
This is illustrated by Fig.~\ref{fig:drug-AD-genes-network}, in which
target to AD gene paths of length 2 are indicated by thick edges.

\begin{figure}
\includegraphics[width=0.95\textwidth]{../../results/2022-09-26-selected-drugs/knowledge-dtn.sif.pdf}
  \caption{Shortest paths between selected drug(s) and AD genes in the
    gene-gene interaction network.  TODO: make figure (the present figure is
    just a placeholder).
}
\label{fig:drug-AD-genes-network}
\end{figure}

\subsection{Experimental validation}

We performed an extensive set of AD-relevant cell based phenotypic assays
($23$ assays) to investigate the utility of three of the four selected drugs
introduced above: Arundine, Chenodiol and Cysteamine-HCl
(Fig.~\ref{fig:cell-based-assays}.  (Note that Cysteamine-HCl is shortened to
``Cysteamine'' in the figure.)  For each assay we determined \emph{a priori}
whether the effect of a hypothetical, ideal neuroprotective drug would
increase or decrease the assay's outcome (Table~S\ref{tab:protect-sign}).
Using that information, we then evaluated each drug's neuroprotective utility
for each assay (Fig.~\ref{fig:cell-based-assays}, also see below).

Amyloid-$\beta$
(A$\beta$) clearance in BV2 cells was improved by both Arundine and Chenodiol
but worsened by Cysteamine-HCl (Fig.~\ref{fig:cell-based-assays}A).
Fig.~\ref{fig:cell-based-assays}B shows that secretion of all three A$\beta$
species tested in H4 cells was significantly decreased by both Arundine and
Chenodiol, which further supports these drugs' potential for AD treatment
(Fig.~\ref{fig:cell-based-assays}B); Cysteamine-HCl, on the other hand,
decreased only one species (A$\beta$1-42).  Neither Tau phosphorylation,
measured as pTau231 level, nor total Tau was significantly changed
(Fig.~\ref{fig:cell-based-assays}C), which suggests that none of the selected
drugs appears to improve Tau-dependent AD pathophysiology (although note the
unexpected, significant, increase of pTau231:Tau ratio for Arundine).  The
same conclusion is supported by Fig.~\ref{fig:cell-based-assays}H showing
unchanged Tau aggregation in a cell-free assay. BV2 cells were protected from
lipopolysaccharide (LPS) induced neuroinflammation by Chenodiol as measured by
the MTT assay (Fig.~\ref{fig:cell-based-assays}D), while Arundine had no
effect and Cysteamine even increased cell death.  On the other hand, Arundine
was the only drug that protected cells from LPS-induced TNF-$\alpha$ increase
(Fig.~\ref{fig:cell-based-assays}D).  In the same neuroinflammation model,
interleukins (IL-10, IL-1b, IL-6) were not changed by any of the drugs except
for a weakly significant increase by Chenodiol
(Fig.~\ref{fig:cell-based-assays}D).  Only Arundine protected primary neurones
from any aspect of trophic factor withdrawal (PI cell death,
Fig.~\ref{fig:cell-based-assays}F); in contrast, Chenodiol and Cysteamine-HCl
actually promoted certain aspects of cell death (LDH, MTT,
PI~Fig.~\ref{fig:cell-based-assays}F).  Finally, none of the drugs impacted
various assays of neurogenesis and neurite outgrowth in primary cultured
neurons (Fig.~\ref{fig:cell-based-assays}e).

Taken these experimental results together, Arundine and Chenodiol, but not
Cysteamine-HCl, exhibit several AD-relevant neuroprotective properties.

\begin{figure}
\includegraphics[scale=0.5]{../../notebooks/2022-09-21-cell-based-assays/named-figure/cell-based-assays-simplified.pdf}
\caption{Results of cell-based assays: testing AD-relevant protective effects of three
  selected top drugs from the present computational screen: Chenodiol,
  Cysteamine-HCl (shortened to ``Cysteamine'' in this figure), and Arundine.
  TODO: ask Tina about details: Throughout panels A-H tested drug
  concentrations were $c_1 = ?$, $c_2 = ?$, $c_3 = ?$, $c_4 = ?$, $c_5 = ?$,
  $c_6 = ?$.  Statistically significant effects are marked with $\ast$,
  $\ast\ast$, $\ast\ast\ast$, and with increasingly deeper blue or red,
  denoting $p < ?$, $p < ?$,  $p < ?$, respectively.
  A) A$\beta$ clearance in BV2 cells; improved clearance is indicated by
  increased A$\beta$ in lysate, decreased A$\beta$ in supernatant and
  decreased supernatant:lysate ratio.
  B) Secretion of three A$\beta$ species in H4 cells.
  C) pTau231 and total Tau level.
  D) Lipopolysaccharide (LPS) induced neuroinflammation in BV2 cells; MTT
  assay of cell viability and the following citokine levels: interleukins
  (IL-10, IL1b, IL-6), the chemokine KC/GRO, and tumor necrosis factor
  (TNF)-$\alpha$.
  E) Assays of neurogenesis and neurite outgrowth in primary cultured neurons.
  F) Assays of cell death, viability and apoptosis in response to trophic
  factor withdrawal in primary neurones.
}
\label{fig:cell-based-assays}
\end{figure}


\section{Discussion}

Computational drug repurposing is a new, rapidly evolving field without
established standards and with a multitude of varied approaches applicable to
a growing, diverse multi-omic body of data and curated
knowledge~\citep{Pushpakom2019}.  During our present investigations two drug
repurposing studies on AD were published: \cite{Fang2021} nominated sildenafil
while~\cite{Taubes2021} bumetanide.  These two works differed both in their
definition of AD genes as well as in the approach used to score and rank
drugs: \cite{Taubes2021} analyzed APOE genotype-specific DEGs with
CMap~\citep{Lamb2006}, while \cite{Fang2021} used knowledge-based AD genes, as
well as DEGs without known APOE genotype, as inputs to a network proximity
based drug repurposing screen~\citep{Cheng2018}.  In this paper we show that,
while the input gene sets have considerable effect on network proximity based
drug scores, the latter still show significant correlation.  Surprisingly, we
also find that the network proximity approach taken by \cite{Fang2021} and
CMap used by \cite{Taubes2021} leads to fully discordant, uncorrelated
results.  Moreover, we find that various published implementation of CMap lead
to quite different, albeit still correlated, results.  These observations
caution about methodological details.

Although the present study shares several methodological limitations with
those of \cite{Fang2021,Taubes2021}, we may argue to have carried out the most
carefully designed drug repurposing screen to date on AD.  We based our screen
on a diverse collection of AD gene sets that included a curated knowledge
based gene set, meta-analyzed, aggregated gene sets from TWAS and PWAS as well
as DEG sets from transcriptomic studies controlling for APOE genotype and
other genetic factors~\citep{Lin2018}.  These varied AD gene sources are
expected to introduce markedly varying bias into our drug repurposing screen: for
instance, the TWAS based gene set is small as it only contains genes with
strong statistical evidence for causal involvement in AD but likely misses
many other causal genes, while the DEG sets are large and likely inflated with
many genes that are not causal to AD.  It is easy to see how combination of
such diverse gene sets prior to the screen may further worsen the associated
biases. To prevent that, we conducted multiple screens, each conditioned on a
gene set and performed \emph{a posteriori} rank aggregation.  The rank
aggregation strategy also revealed that even some non-overlapping AD gene sets
resulted lead to the discovery of highly overlapping sets of very top drugs
demonstrating the robustness of our workflow.  Notably, the non-neuronal
AD gene sets were exception of this tendency pointing at the value of further
characterization of non-neuronal cells in
AD~\citep{Lopes2022,Mathys2019,DeStrooper2016}.

TODO: Top drugs and selected top drugs.  Drugs nominated
by~\cite{Fang2021,Taubes2021}.

Our computational validation, based on the rediscovery of drugs previously
developed or repurposed for AD, revealed the enrichment of our top ranked
drugs for also some common non-AD indications.  Genetic correlations between
disease phenotypes~\citep{Consortium2018}, market-driven selection biases, and
other factors may underlie not only this observation but also the known
comorbidities of AD~\citep{Santiago2021}.  Such overlap between different
drugs' indications is also consistent with the known relationship between drug
repurposing and side effects~\citep{Ye2014}.  Further research is needed to
dissect relationships between genetic correlations, comorbidities, side
effects, protein and gene network topology, etc to be able to harness these and
integrate into drug repurposing techniques.  That, in combination with
expanding multi-celltype, multi-omic characterization of AD and its molecular
subtypes~\citep{Neff2021}, will herald a new generation of drugs for
Alzheimer's disease.

\section{Methods}

\subsection{AD gene sets}

Given an AD gene set, the network structure of the human interactome, and
another network formed by drugs and their targets (drug-target network), the
topological proximity of a given drug to the AD gene set can be
quantified\citep{Guney2016} and used as evidence for repurposing that drug for
AD.  Rather than compiling a single AD gene set from multiple studies, we
opted for evaluating the proximity of any drug separately for multiple AD gene
sets and aggregate the multiple proximity values subsequently (see
Discussion).  Table~\ref{tab:genesets} summarizes the AD gene sets used in
this study.  The genes contained in each set are listed in Table~S1.

\subsection{Previous and novel TWAS on AD}

We obtained AD gene sets from multiple TWAS.  Besides collecting
AD genes from previous transcriptome- and proteome-wide association studies
(TWAS, PWAS) we
conducted our own applying the S-MultiXcan approach~\citep{Barbeira2018} to
the two largest AD-GWAS to date~\citep{Schwartzentruber2021,Wightman2021}.
S-MultiXcan borrows statistical strength from multiple tissue types; we used a
predictive model of gene expression built on all tissues of the GTEx data set
or on only brain tissues.
TODO: describe here the results of Haky et al:
number of significant genes (check with Haky et al on FDR)
%find interesting
%novel genes based on their membership in AD-relevant pathways.

We conservatively took the top 30 genes and investigated their agreement with
top gene sets from seven previous TWAS and one previous PWAS (see Table~S2
listing genes contained in each set).  Hierarchical clustering of studies
based on shared genes (Fig.~S\ref{fig:twas-clustermap}) showed a fairly low
agreement for all pairs of studies.  The most similar study pair was our
present TWAS and the fine mapping of~\cite{Jansen2019} despite several
methodological differences between these two studies, which helps validate our
novel TWAS results.  As might be expected, the least similarity was found
between the only PWAS in our collection and all the other studies.

\subsection{Drug-target network}

We queried ChEMBL~\citep{Gaulton2017} (v29, July 2021) for phase 3 and 4 drugs
and those of their targets that satisfied the following conditions: (1) the
target is a human protein (2) its average $-\log_{10}\mathrm{Activity}$ or
$p\mathrm{Activity}$ is $\ge 5$, which corresponds to $\le 10 \mu M$
following~\cite{Cheng2018}.  Average $p\mathrm{Activity}$ activity is defined
as $\sum_{i=1}^n p\mathrm{Activity}_i \ge 5$, where $n$ is the number of
activity measurements for a given drug-protein pair and $\mathrm{Activity}_i$
is the standard value of the $i$-th measurement stored in the
activities.standard\_value SQL variable.  See
\href{https://github.com/attilagk/CTNS-notebook/blob/main/2021-10-24-chembl-query/drug_target_avg_activity.sql}{our SQLite script} for implementation.

\subsection{Gene-gene interaction network}

We used the network compiled by \cite{Cheng2019}, which consists 11133 nodes
(genes) and 217160 edges (gene-gene interactions).

\subsection{Network proximity calculations}

\cite{Guney2016} developed the network proximity method for in silico drug efficacy screening.  It
quantifies the topological proximity $d$ in the interactome of the set $T$ of a drug's target genes
to the set $S$ of disease genes: 

\begin{equation}
  d(S, T) = \frac{1}{|T|}\sum_{t \in T} \min_{s \in S} d(s, t)
\end{equation}
where $d(s,t)$ is the shortest path length between disease gene $s$ and drug
target $t$.  The equation can be interpreted as the average length of shortest
paths taken from all targets of a drug to all the disease genes.

We used \href{https://github.com/attilagk/guney_code}{our modified clone} of
the original, now obsolete, implementation of \cite{Guney2016}, which allows
running the code under Python~3.

\subsection{CMap analyses}

Since there are several implementations of CMap (Connectivity Map) analysis was carried out with several tools.  First, we used
\url{https://clue.io/} \citep{Lamb2006} and, for each drug, averaged
``norm\_cs'',
the normalized connectivity scores.  Second, we used L1000CDS$^2$ at
\url{https://maayanlab.cloud/L1000CDS2} and the averaged ``score'' variable.
Third, we used... TODO: ask Sudhir what implementation he used.

\subsection{Rank aggregation}

As Fig.~\ref{fig:workflow} shows, we aggregated 8 lists of network
proximity-ranked drugs into a single list with rank aggregation algorithms.
We performed rank aggregation with the TopKLists R package
%\citep{Schimek2015}
several times, each time with different algorithm.  Then we assessed those
algorithm's performance with Modified Kendall Distance and found the MC3
algorithm to perform the best (Fig.~S\ref{fig:kendall-dist}).  Finally, we
used the MC3-aggregated list in our workflow.

\subsection{Cell-based assays}

We tested whether our selected drugs can rescue molecular phenotypes relevant
to Alzheimer's disease including tau phosphorylation, A$\beta$1-42 clearance,
A$\beta$ secretion, A$\beta$ toxicity, lipopolysaccharide (LPS)-induced
neuroinflammation, cell death due to trophic factor withdrawal and neurite
outgrowth and neurogenesis~\citep{Varma2020,Desai2022a}. In all assays
described below, cells treated with drugs were compared to either the vehicle
control (VC) or lesion control.

Statistical analysis was performed in GraphPad Prism. Group differences
were evaluated for each test item separately by one-way analysis of variance
(ANOVA) followed by Dunnett's multiple comparison test versus vehicle or
lesion control.

\subsubsection{A$\beta$ clearance}

For A$\beta$ clearance assay, 20,000 BV2 cells per well (uncoated 96-well
plates) were plated out. After changing cells to treatment medium, drug
compounds were administered 1 hour before A$\beta$ stimulation (Bachem,
4061966; final concentration in well, 200 ng/ml; dilutions in medium). Cells
treated with vehicle and cells treated with A$\beta$ alone served as controls.
After 3 hours of A$\beta$ stimulation, cell supernatants were collected for
the A$\beta$ measurement and cells were carefully washed twice with PBS and
thereafter lysed in 35 $\mu$l of cell lysis buffer (50 mM tris-HCl, pH 7.4,
150 mM NaCl, 5 mM EDTA, and 1\% SDS) supplemented with protease inhibitors.
Supernatants and cell lysates were analyzed for human A$\beta$42 with the MSD
V-PLEX Human A$\beta$42 Peptide (6E10) Kit (K151LBE, Mesoscale Discovery). The
immune assay was carried out according to the manual, and plates were read on
MESO QuickPlex SQ 120.

\subsubsection{A$\beta$ secretion}

H4-hAPP cells were cultivated in Opti-MEM supplemented with 10\% FCS, 1\%
penicillin/streptomycin, hygromycin B (200 $\mu$g/ml), and blasticidin S (2.5
$\mu$g/ml).  H4-hAPP cells were seeded into 96-well plates (2 $\times$ 104
cells per well). On the next day, cells in 96-well plates were treated with
compounds, reference item [400 nM
N-[N-(3,5-difluorophenacetyl-l-alanyl)]-S-phenylglycine t-butyl ester (DAPT)],
or vehicle. Twenty-four hours later, supernatants were collected for further
A$\beta$ measurements by MSD [V-PLEX A$\beta$ Peptide Panel 1 (6E10) Kit,
K15200E, Mesoscale Discovery].

\subsubsection{Tau phosphorylation}

SH-SY5Y-hTau441(V337M/R406W) cells were maintained in culture medium [DMEM
medium, 10\% FCS, 1\% nonessential amino acids (NEAA), 1\% l-glutamine,
gentamycin (100 $\mu$g/ml), and geneticin G-418 (300 $\mu$g/ml)] and
differentiated with 10 $\mu$M retinoic acid for 5 days changing medium every 2
to 3 days. Before the treatment, cells were seeded onto 24-well plates at a
cell density of 2 $\times$ 105 cells per well on day one of in vitro culture
(DIV1). Drug compounds were applied on DIV2. After 24 hours of incubation
(DIV3), cells on 24-well plates were harvested in 60 $\mu$l of RIPA buffer [50
mM tris (pH 7.4), 1\% NP-40, 0.25\% Na-deoxycholate, 150 mM NaCl, 1 mM EDTA
supplemented with freshly added 1 $\mu$M NaF, 0.2 mM Na-orthovanadate, 80
$\mu$M glycerophosphate, protease (Calbiochem), and phosphatase
(Sigma-Aldrich) inhibitor cocktail]. Protein concentration was determined by
BCA assay (Pierce, Thermo Fisher Scientific), and samples were adjusted to a
uniform total protein concentration. Total tau and phosphorylated tau were
determined by immunosorbent assay from Mesoscale Discovery
[Phospho(Thr231)/Total Tau Kit K15121D, Mesoscale Discovery].

\subsubsection{LPS-induced neuroinflammation}

The murine microglial cell line BV2 was cultivated in Dulbecco’s modified
Eagle's medium (DMEM) medium supplemented with 10\% fetal calf serum (FCS),
1\% penicillin/streptomycin, and 2 mM l-glutamine (culture medium). For LPS
stimulation assay, 5000 BV2 cells per well (uncoated 96-well plates) were
plated out and the medium was changed to treatment medium (DMEM, 5\% FCS, and
2 mM l-glutamine). After changing cells to treatment medium, drug compounds
were administered 1 hour before LPS stimulation [Sigma-Aldrich; L6529; 1 mg/ml
stock in ddH2O; final concentration in well, 100 ng/ml (dilutions in medium)].
Cells treated with vehicle, cells treated with LPS alone, and cells treated
with LPS plus reference item (dexamethasone, 10 $\mu$M; Sigma-Aldrich, D4902)
served as controls. After 24 hours of stimulation, cell supernatants were
collected for the cytokine measurement (V-PLEX Proinflammatory Panel 1 Mouse
Kit, K15048D, Mesoscale) and cells were subjected to
3-(4,5-dimethylthiazol-2-yl)-2,5-diphenyltetrazolium bromide (MTT) assay.

\subsubsection{Neurite outgrowth and neurogenesis}

Primary hippocampal neurons were prepared from E18.5 timed pregnant
C57BL/6JRccHsd mice as previously described. Cells were seeded in
poly-D-lysine pre-coated 96-well plates at a density of 2.6 $\times$ 104
cells/well in medium (Neurobasal,
2\% B-27, 0.5 mM glutamine, 25 $\mu$M glutamate,
1\% Penicillin–Streptomycin). Directly on DIV1, the drug of interest or VC was
applied. On DIV2, 10 $\mu$M Bromodeuxyuridine (BrdU; B5002 Sigma Aldrich) was
added and cells were fixed after additional 24h. Cells were permeabilized with
0.1\% Triton-X and incubated with primary Beta Tubulin Isotype III (T8660,
Sigma Aldrich) and BrdU antibodies (MAS25$\circ$C, AHrlan-Sera Lab) overnight
at 4$\circ$C. Afterwards, cells were washed two times with PBS and incubated
with fluorescently labelled secondary antibodies and DAPI for 1.5 h at room
temperature (RT) in the dark. Cells were rinsed three times with PBS and
imaged with the Cytation 5 Multimode reader (BioTek) at 10 $\times$
magnification (six images per well). BrdU-positive cells were counted as a
marker of neurogenesis and Beta Tubulin Isotype III signal was used for
macro-based quantification of neurite outgrowth.

\subsubsection{Trophic factor withdrawal}

Primary cortical neurons from embryonic day 18 (E18) C57Bl/6 mice were
prepared as previously described. On the day of preparation (DIV1), cortical
neurons were seeded on poly-d-lysine precoated 96-well plates at a density of
3 $\times$ 104 cells per well. Every 4 to 6 days, a half medium exchange using full
medium (Neurobasal, 2\% B-27, 0.5 mM glutamine, and 1\% penicillin-streptomycin)
was carried out. On DIV8, a full medium exchange to B-27 free medium
(Neurobasal, 0.5 mM glutamine, and 1\% penicillin-streptomycin) was performed
and drug compounds were applied thereafter. The experiment was carried out
with six technical replicates per condition, and vehicle-treated cells
served as control. After 28 hours on B-27–free medium, cells were subjected to
YO-PRO/propidium iodide (PI) and MTT as well as lactate dehydrogenase (LDH)
assay.


\bibliography{repurposing-ms}

\newpage

\section*{Supplementary Material}

\setcounter{table}{0}
\makeatletter 
\renewcommand{\tablename}{Table S} % nice
%\renewcommand{\figurename}{Table S} % nice
\makeatother

\setcounter{figure}{0}
\makeatletter 
\renewcommand{\figurename}{Figure S} % nice
\makeatother

\begin{table}[p]
\caption{
  Genes of the AD gene sets used as inputs to the present computational drug screen.
}
\label{tab:genes-in-genesets}
\end{table}

\begin{table}[p]
\caption{
  The 2413 drugs ranked according to their network proximity to each of the eight
  AD gene sets used as input.  The drugs' final, aggregate rank is also shown
  as well as their ChEMBL ID, standard InChI, indication class, and
  blood-brain-barrier permeability taken (if available) from the BBB
  database~\citep{Meng2021}.  Moreover, the UniProt name of each drug's
  targets is also indicated.
}
\label{tab:ranked-drugs}
\end{table}

\begin{table}[p]
\begin{tabular}{lrlrr}
\toprule
{} &  CYP3A4 &          RGS4 &  SLC10A2 &  shortest path length \\
\midrule
AD DE APOE3-APOE3 &         &          RGS4 &          &                     0 \\
AD DE APOE4-APOE4 &         &          RGS4 &          &                     0 \\
agora2+           &         &         ERBB3 &          &                     1 \\
AD DE APOE3-APOE3 &         &         GNAI2 &          &                     1 \\
AD DE APOE4-APOE4 &         &  CALM1; GNAI2 &          &                     1 \\
APOE3-4 DE neuron &         &         PLCB1 &          &                     1 \\
\bottomrule
\end{tabular}
\caption{
  Nearest AD gene(s) to a given target.  Shortest path length
  of 0 and 1 mean that the target is identical to or directly interacts with
  the AD gene listed, respectively.  Path lengths $\ge 2$ are not considered
  here.
}
\label{tab:nearest-ADgenes}
\end{table}

\begin{table}[p]
\begin{tabular}{llr}
\toprule
                              &                   &  ideal effect: increase (1) or decrease (-1)\\
class & assay &                                                                      \\
\midrule
A)  A$\beta$ clearance & Abeta in Lysate &                                                  1 \\
                              & Abeta in SN &                                                 -1 \\
                              & ratio &                                                 -1 \\
B)  A$\beta$ secretion & Ab1-38 &                                                 -1 \\
                              & Ab1-40 &                                                 -1 \\
                              & Ab1-42 &                                                 -1 \\
C)  Tau phosphorylation & pTau231 &                                                 -1 \\
                              & ratio &                                                 -1 \\
                              & total Tau &                                                  1 \\
D)  LPS neuroinflammation & IL-10 &                                                 -1 \\
                              & IL-1b &                                                 -1 \\
                              & IL-6 &                                                 -1 \\
                              & KC/GRO &                                                 -1 \\
                              & MTT &                                                  1 \\
                              & TNF-a &                                                 -1 \\
E)  Neurogenesis, neurite outgrowth & BrdU positive neurons &                                                  1 \\
                              & average longest neurite &                                                  1 \\
                              & number of branches &                                                  1 \\
                              & total neurite area &                                                  1 \\
F)  Trophic factor withdrawal & LDH (cell death) &                                                 -1 \\
                              & MTT (viability) &                                                  1 \\
                              & PI (cell death) &                                                 -1 \\
                              & YOPRO (apoptosis) &                                                 -1 \\
\bottomrule
\end{tabular}
\caption{
  Ideal, neuroprotective, effect for each assay.  Ideal effect is the
  direction of effect of an ideal, hypothetical, neuroprotective drug on a
  given assay.
}
\label{tab:protect-sign}
\end{table}

\begin{figure}[p]
\includegraphics[scale=0.4]{../../notebooks/2022-01-14-top-drugs/named-figure/rank-diff.pdf}
\includegraphics[scale=0.4]{../../notebooks/2022-01-14-top-drugs/named-figure/rank-diff-100.pdf}
\caption{
Difference of drug rank between the final, aggregated list and the list
resulting from each network proximity based drug screen.
}
\label{fig:rank-diff}
\end{figure}

%\begin{figure}[p]
%\includegraphics[scale=0.28]{../../notebooks/2021-12-02-proximity-various-ADgenesets/named-figure/prox-z-scatterplot_matrix.png}
%\caption{
%Z score of 2413 drugs (points) based on various input AD gene sets (rows and
%columns).
%}
%\label{fig:prox-z-scatterplot-m}
%\end{figure}

\begin{figure}
\includegraphics[scale=0.6]{../../notebooks/2021-07-01-high-conf-ADgenes/named-figure/cluster-experiments-genes.pdf}
\caption{
Similarity of TWA/PWA studies on AD in terms of shared genes.
}
\label{fig:twas-clustermap}
\end{figure}

\begin{figure}[p]
\includegraphics[scale=0.6]{../../notebooks/2021-11-30-proximity-cmap/named-figure/cmap-proxim-apoe4apoe4.png}
\caption{
Drug repurposing scores obtained by network proximity approach and CMap
}
\label{fig:proxim-cmap}
\end{figure}

\begin{figure}[p]
\includegraphics[scale=0.4]{../../notebooks/2021-11-30-proximity-cmap/named-figure/cmap-proxim-celltype-apoe4-apoe4.png}
\caption{
Drug repurposing scores obtained by network proximity approach and CMap in
given the cell type of the CMap expression profiling.
}
\label{fig:proxim-cmap-celltype}
\end{figure}

\begin{figure}[p]
\includegraphics[scale=0.6]{../../notebooks/2021-10-04-CMap-discussion/named-figure/cmap-tools-consistency.pdf}
\caption{
Results under different implementations of CMap.
TODO: remove Sudhir's?
}
\label{fig:cmap-cmap}
\end{figure}

\begin{figure}[p]
\includegraphics[scale=0.4]{../../results/2022-01-14-rank-aggregation/kendall-distance.png}
\caption{
Several rank aggregation algorithms' performance assessed by modified Kendall
distance.
}
\label{fig:kendall-dist}
\end{figure}

\begin{figure}[p]
\includegraphics[scale=0.6]{../../notebooks/2022-01-14-top-drugs/named-figure/top-bottom-ratio-top-k.pdf}
\caption{
Computational validation: relative rediscovery rate of drugs already investigated or
approved for AD in the top-$t$ drugs ranked by network proximity.  The relative
rediscovery rate being $>1$ validates our network based screen.
}
\label{fig:ad-drug-rediscovery}
\end{figure}

\begin{figure}[p]
\includegraphics[scale=0.4]{../../notebooks/2022-01-14-top-drugs/named-figure/top-bottom-ratio-top-k-multi-1.pdf}
\caption{
Rediscovery of existing phase 1--4 drugs for AD and other indications.
}
\label{fig:ad-drug-rediscovery-multi}
\end{figure}

\begin{figure}[p]
\includegraphics[scale=0.4]{../../notebooks/2022-09-26-selected-drugs/named-figure/shortest-path-lengths-gsets-hist_noHCl.pdf}
\caption{
  Shortest path length distribution from a given drug target (columns) to a given AD gene
  set (rows) across all genes in the set.
}
\label{fig:shortest-path-lengths}
\end{figure}

\begin{figure}[p]
\includegraphics[scale=0.4]{../../notebooks/2022-09-26-selected-drugs/named-figure/shortest-path-lengths-barh_noHCl-knowledge.pdf}
\caption{
  Shortest path lengths from drug targets to genes of the knowledge based AD
  gene set.
}
\label{fig:shortest-path-bar-knowledge}
\end{figure}

\begin{figure}[p]
\includegraphics[scale=0.4]{../../notebooks/2022-09-26-selected-drugs/named-figure/shortest-path-lengths-barh_noHCl-TWAS2+.pdf}
\caption{
  Shortest path lengths from drug targets to genes of the TWAS2+ based AD
  gene set.
}
\label{fig:shortest-path-bar-TWAS2+}
\end{figure}

\begin{figure}[p]
\includegraphics[scale=0.4]{../../notebooks/2022-09-26-selected-drugs/named-figure/shortest-path-lengths-barh_noHCl-agora2+.pdf}
\caption{
  Shortest path lengths from drug targets to genes of the agora2+ based AD
  gene set.
}
\label{fig:shortest-path-bar-agora2+}
\end{figure}

\begin{figure}[p]
\includegraphics[scale=0.15]{../../notebooks/2022-09-26-selected-drugs/named-figure/shortest-path-lengths-barh_noHCl-AD DE APOE3-APOE3.pdf}
\caption{
  Shortest path lengths from drug targets to genes of the AD DE APOE3-APOE3 based AD
  gene set.
}
\label{fig:shortest-path-bar-AD DE APOE3-APOE3}
\end{figure}

\begin{figure}[p]
\includegraphics[scale=0.15]{../../notebooks/2022-09-26-selected-drugs/named-figure/shortest-path-lengths-barh_noHCl-AD DE APOE4-APOE4.pdf}
\caption{
  Shortest path lengths from drug targets to genes of the AD DE APOE4-APOE4 based AD
  gene set.
}
\label{fig:shortest-path-bar-AD DE APOE4-APOE4}
\end{figure}

\begin{figure}[p]
\includegraphics[scale=0.4]{../../notebooks/2022-09-26-selected-drugs/named-figure/shortest-path-lengths-barh_noHCl-APOE3-4 DE neuron.pdf}
\caption{
  Shortest path lengths from drug targets to genes of the APOE3-4 DE neuron based AD
  gene set.
}
\label{fig:shortest-path-bar-APOE3-4 DE neuron}
\end{figure}

\begin{figure}[p]
\includegraphics[scale=0.3]{../../notebooks/2022-09-26-selected-drugs/named-figure/shortest-path-lengths-barh_noHCl-APOE3-4 DE astrocyte.pdf}
\caption{
  Shortest path lengths from drug targets to genes of the APOE3-4 DE astrocyte based AD
  gene set.
}
\label{fig:shortest-path-bar-APOE3-4 DE astrocyte}
\end{figure}

\begin{figure}[p]
\includegraphics[scale=0.3]{../../notebooks/2022-09-26-selected-drugs/named-figure/shortest-path-lengths-barh_noHCl-APOE3-4 DE microglia.pdf}
\caption{
  Shortest path lengths from drug targets to genes of the APOE3-4 DE microglia based AD
  gene set.
}
\label{fig:shortest-path-bar-APOE3-4 DE microglia}
\end{figure}

\end{document}
