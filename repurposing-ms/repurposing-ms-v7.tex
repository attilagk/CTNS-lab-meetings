\documentclass[letterpaper]{article}
%\documentclass[12pt,letterpaper]{article}
%\setlength{\textwidth}{480pt}
%\setlength{\textheight}{630pt}
%\setlength{\voffset}{0pt}

\usepackage{amsmath, geometry, graphicx}
\usepackage{natbib}
%\usepackage{float}
\usepackage{url,hyperref}
\usepackage{booktabs}
\bibliographystyle{plainnat}

% https://tex.stackexchange.com/questions/6758/how-can-i-create-a-bibliography-like-a-section
%\usepackage{etoolbox}
%\patchcmd{\thebibliography}{\section*}{\section}{}{}

\pagestyle{plain}

\title{Novel Drug Repurposing Opportunities for Alzheimer's Disease from
  Genomic Fine Mapping and Transcriptomic Disease Signatures}

  \author{Attila Jones, ..., Madhav Thambisetty}

\date{Coauthors in alphabetical order: Hae Kyung Im, Tina Loeffler, Irene
Schilcher, Vijay Varma, Evan Wu}

\begin{document}

\maketitle
\section{Introduction}

Late onset Alzheimer's disease (AD) is a progressive neurodegenerative disorder and the most
common form of dementia affecting a growing number of individuals in ageing
societies for which effective treatments are
unavailable~\citep{Bondi2017,Masters2015}.  Repurposing drugs approved for other indications is
a promising approach  towards identification of effective disease-modifying AD
treatments~\citep{Pushpakom2019,Fang2021,Taubes2021}.  One of the 
milestones of the National Plan for Alzheimer’s Disease is to ``Initiate
research programs for translational bioinformatics and network pharmacology to
support rational drug repositioning''~\citep{NIH/NIA}.%~\citep{Aging}.

Although a few major genetic risk factors for AD are known---most importantly
the $\epsilon 4$ allele of the apolipoprotein E gene (APOE4)
\cite{Yamazaki2019}---our understanding of the early etiological triggers of
the disease is still limited.  Among several reasons for this lack of clarity
are the considerable genetic and cellular complexity of AD, and the
confounding early etiological processes by compensatory and degenerative
pathologies over the decades-long disease progression~\citep{DeStrooper2016}.
Consequently, discovering treatments targeting early causal drivers of AD
pathogenesis has been a major hurdle to developing effective treatments.  In
this study, we attempted to address this challenge in two ways.  Firstly, we
derived plausible mechanistic insights into AD from previous genome-,
transcriptome- and proteome-wide association studies (GWAS, TWAS, PWAS,
respectively)~\citep{Jansen2019,Kunkle2019,Gerring2020,Baird2021,Schwartzentruber2021,Wightman2021,Wingo2021}
comparing AD and control samples, including analyses of differentially
expressed genes (DEGs) in APOE3/3 versus APOE4/4 individuals as
well as from comparison of DEGs in iPSC-derived neurons, astrocytes and
microglia from APOE3/3 versus APOE4/4
individuals~\citep{Taubes2021,Lin2018}~(Fig.~\ref{fig:workflow}A).

\begin{figure}
\includegraphics[width=1.0\textwidth]{figures/Fig1.png}
\caption{
  A) Workflow of computational drug repurposing in Alzheimer's disease (AD).
  B) Principle of network proximity-based drug screen for AD.  Drug 4 and 5
  are plausible candidate AD drugs because they target either directly or
  indirectly an AD risk gene (i.e a gene mechanistically involved in AD), respectively .  On the other
  hand, drugs 1--3 only target genes at least three interactions away from AD
  genes and therefore are likely not plausible candidate AD drugs.
}
\label{fig:workflow}
\end{figure}

Second, we exploited rich, mechanistic information represented by two
interconnected networks (Fig~\ref{fig:workflow}B): (1) the network formed by
all drugs and their target genes (drug-target network) and (2) the human
gene-gene interaction network---known as the interactome---, whose Alzheimer's
subnetwork---named Alzheimer's disease module~\citep{Barabasi2011}---we
defined based on sets of AD risk genes from
the aforementioned omic studies.  Combining these two networks, we then
estimated the potential efficacy of each drug for AD by evaluating its network
proximity~\citep{Guney2016}, which is based on the drug's network topological
distance from the AD module.  Network proximity was shown~\citep{Cheng2018} to
be a powerful, systems-level, alternative to merely correlative approaches to
computational drug repurposing such as the connectivity map,
CMap~\citep{Lamb2006}.  We present the results and computational validation of
our network proximity-based drug screen as well as experimental validation of
selected top ranking plausible novel candidate AD drugs.

\section{Results}

\subsection{Evaluating drugs' network proximities to Alzheimer's disease
modules in the interactome}

Drugs proximal to some disease module in the human interactome were previously
shown to be good candidates for repurposing to that disease~\citep{Cheng2018}.
We therefore performed a network proximity-based drug repurposing screen for
AD using 2413 drugs from ChEMBL that were either approved by the United States
Food and Drug Administration, or are in phase 3 clinical trials, for any
disease indication (Fig.~\ref{fig:workflow}A).  We evaluated each drug's
network proximity to each of eight AD modules, which we derived from distinct
complementary information sources (one curated, knowledge-based source and
seven multi-omic sources, Table~\ref{tab:genesets}).  This resulted in eight
distinct lists of the same 2413 drugs ranked according to increasing network
proximity score and hence decreasing relevance for AD
(Fig.~\ref{fig:workflow}A). We then nominated the top-ranking drugs as
promising novel candidates for repurposing in AD.

\begin{table}
\footnotesize
\begin{tabular}{rrll}
\toprule
AD risk gene set          &Size & Description & Reference  \\
\hline                     
knowledge            &  27 & curated AD risk genes from the DISEASES database & \cite{PletscherFrankild2015} \\
TWAS2+               &  32 & genes supported by $\ge 2$ AD-TWAS/PWAS & Ref.~in caption  \\
agora2+              &  64 & genes supported by $\ge 2$ Agora studies & https://agora.adknowledgeportal.org \\
AD DE APOE3-APOE3    & 277 & AD vs control DEGs: APOE3/APOE3 background & \cite{Taubes2021} \\
AD DE APOE4-APOE4    & 274 & AD vs control DEGs: APOE4/APOE4 background & \cite{Taubes2021} \\
APOE3-4 DE neuron    &  46 & APOE4 vs APOE3 DEGs: iPSC-derived neurons& \cite{Lin2018} \\
APOE3-4 DE astrocyte & 128 & APOE4 vs APOE3 DEGs: iPSC-derived astrocytes& \cite{Lin2018} \\
APOE3-4 DE microglia & 140 & APOE4 vs APOE3 DEGs: iPSC-derived microglia& \cite{Lin2018} \\
\bottomrule
\end{tabular}
\caption{
AD risk gene sets used as inputs to drug repurposing screens of this study.  For
the TWAS2+ gene set we combined gene sets from the prior published TWAS of
\cite{Gerring2020,Baird2021,Jansen2019,Kunkle2019,Wingo2021,Schwartzentruber2021}
as well as from our own TWAS (Methods).  The genes in each set are listed in
Table~S\ref{tab:genes-in-genesets}.
}
\label{tab:genesets}
\end{table}

Although the eight input AD risk gene sets had little overlap with each other
(Fig.~S\ref{fig:gset-jaccard}), the resulting eight vectors of network proximity had
markedly high correlations (Fig.~S\ref{fig:gset-corr}) so that the eight lists of
drug rankings shared substantial similarity (Fig.~\ref{fig:screen}A).
This suggests that even though the input AD risk gene sets share only a few genes,
the corresponding network modules are close enough to one another in network
topological space to exhibit similar network proximity to any given drug.

\begin{figure}
\includegraphics[width=\textwidth]{figures/Fig2.png}\hspace{0.05\textwidth}
%A\hspace{0.7\textwidth}B
%
%\includegraphics[scale=0.5]{../../notebooks/2021-12-02-proximity-various-ADgenesets/named-figure/jaccard-input-AD-sets.pdf}
%\includegraphics[scale=0.5]{../../notebooks/2021-12-02-proximity-various-ADgenesets/named-figure/corr-coef-input-AD-sets.pdf}
%
%\hspace{0.05\textwidth}C\hspace{0.5\textwidth}D
%
%\includegraphics[scale=0.5]{../../notebooks/2022-01-14-top-drugs/named-figure/separate-aggregate-ranks-heatmap.pdf}
%\includegraphics[scale=0.5]{../../notebooks/2022-01-14-top-drugs/named-figure/separate-aggregate-ranks-heatmap-topk.pdf}

\caption{Scoring and ranking 2413 drugs based on their network proximity to
  various AD risk gene sets.  (A-B) Eight AD risk gene sets were input to the workflow.
  The separate drug rankings across imput AD risk gene sets were aggregated into a
  single final ranking, which is color coded (yellow: top-ranked drugs, blue:
  bottom-ranked drugs).  Note that the area delimited by red rectangle in A)
  is expanded into B).
}
\label{fig:screen}
\end{figure}

Assuming our eight input AD risk gene sets were all equally relevant to Alzheimer's
disease, we aggregated the corresponding drug rankings giving uniform weight
to all gene sets.  This yielded a single final ranked list of the top 605
drugs (Fig.~\ref{fig:screen}A, Table~S\ref{tab:genes-in-genesets}).  The very
top ranked drugs on the final list also tended to be top ranked in a majority
(five out of eight) of input-specific rankings (Fig.~\ref{fig:screen}B,
Fig.~S\ref{fig:rank-diff}).

\subsection{Computational validation}

To validate our network-based drug screen, we asked to what extent the
top-ranked drugs were enriched in drugs already investigated or approved for
AD.  To quantify such enrichment, we defined \emph{rediscovery rate} (Methods,
Eq.~\ref{eq:rediscovery-rate}) for the top-$k$ drugs, ranked by network
proximity, relative to that for the bottom-$1808$ drugs.  The top left panel
of Fig.~\ref{fig:ad-drug-rediscovery-multi} shows that the rediscovery rate
for AD in the top-$k$ drugs was about $2-3$-fold greater than that for the
bottom drugs depending on the value of $k$.  This finding suggests that our
computational approach to AD drug discovery yields both novel drug candidates
as well as those that with a biological rationale justifying their current
status as approved AD drugs or testing in prior human clinical trials, thereby
providing a measure of methodological validation.

\begin{figure}[p]
\includegraphics[scale=0.4]{../../notebooks/2022-01-14-top-drugs/named-figure/top-bottom-ratio-top-k-multi-1.pdf}
\caption{
Network proximity-based rediscovery of drugs in phase 1--4 clinical trials for
AD and 35 other disease indications.  Disease indications written on top of
individual plots.  Blue dots: rediscovery rate (Eq.~\ref{eq:rediscovery-rate})
for the top-$k$ drugs, ranked by network proximity, relative to that for the
bottom-$1808$ drugs.  Rediscovery rate of $>1$ means that top-ranked drugs
tend to be those that are in phase 1--4 clinical trials for the given
indication; these drugs are marked by orange symbols above the $x$ axes.
}
\label{fig:ad-drug-rediscovery-multi}
\end{figure}

We then re-evaluated rediscovery rate using the same, network proximity-based,
drug ranking as above but this time we queried the rediscovery of drugs for
some other common indication rather than for AD.  Besides AD, there was
sufficient data (i.e~at least 100 drugs approved or in phase $\ge 1$ clinical
trials) for 35 common indications (Fig.~\ref{fig:ad-drug-rediscovery-multi}).
Interestingly, several, but not all, of these indications showed a similar
enrichment pattern to that derived earlier for AD.  We observed that approved
and experimental drugs for various cancers especially overlapped with
candidate AD drugs while those for neuropsychiatric indications like major
depressive disorder, schizophrenia and Parkinson's disease did not
(Fig.~S\ref{fig:ad-drug-rediscovery-multi}).  These findings suggest that the
novel candidate drugs for AD from our network-based screen tend to target
overlapping molecular mechanism(s) in other diseases such as cancer. 

%We also checked the agreement between our network proximity based results
%above and the results we obtained by using CMap with the same input AD risk gene
%set.  Strikingly, we found no agreement between the two even when we
%conditioned on the various cell types of the CMap data set
%(Fig.~S\ref{fig:proxim-cmap},~S\ref{fig:proxim-cmap-celltype}).  Moreover,
%different implementations of CMap yielded considerably different results
%(Fig.~S\ref{fig:cmap-cmap}).  Given these conflicting observations and the
%more realistic, systems-based, mechanistic modeling assumptions behind the
%network approach, we did not analyze the results obtained with CMap any
%further.

\subsection{Selected top ranking candidate AD drugs}

We selected three top ranking compounds for further experimental validation as
plausible candidate AD treatments using cell culture based phenotypic assays
that probe key molecular outcomes relevant to AD pathogenesis
(Table~\ref{tab:selected-drugs}). These candidate drugs were selected based on the following criteria:
\begin{enumerate}
  \item brain penetration/permeability across the blood brain barrier,
  \item close network proximity of the drug target genes to AD risk genes through short path lengths connected by single mediator genes,
  \item prior evidence for roles in molecular mechanisms relevant to neurodegeneration.
\end{enumerate}

Arundine, also known as 3,3-diindolylmethane, is the dimeric product of the
natural product indole-3-carbinol.  While Arundine has been mostly
investigated in the context of drug resistant tumors~\citep{Biersack2020}, its
close analogs have been found to cross the blood-brain barrier and protect
mice from 1-methyl-4-phenyl-1,2,3,6-tetrahydropyridine (MPTP)-induced neurotoxicity and
neurodegeneration~\citep{DeMiranda2013}.
Moreover, Arundine was shown to inhibit oxidative stress induced apoptosis in
hippocampal neuronal cells~\citep{Lee2019} and protected primary hippocampal
cell cultures from ischemia induced apoptosis and
autophagy~\citep{Rzemieniec2019}.  Interestingly, the latter effect was found
to depend on Arundine binding to the Aryl Hydrocarbon Receptor, product of the
AHR gene,~\citep{Rzemieniec2019}.

One of the Arundine targets that our analysis accounted for is the RGS4 gene
(regulator of G-protein signalling 4). RGS4 is notable because it is also
targeted by Chenodiol, another of our selected top-ranking drugs.  Chenodiol,
or Chenodeoxycholic acid, has been shown to be neuroprotective in Huntington's
disease~\citep{Keene2002}.  Moreover, Chenodiol is a bile acid, and plausibly
linked to AD based on recent evidence from our group and
others~\citep{Varma2021,Baloni2020}.

Our third selected top drugs was Cysteamine.  Cysteamine, whose targets also
include RGS4, is a derivative of the amino acid cysteine.  Cysteamine can
traverse the blood brain barrier, was approved for cystinosis (MeSH:D005128).
More recently it has been studied as a potential drug for Huntington's and
Parkinson's disease due to its neuroprotective
activity~\citep{Besouw2013,Paul2019}.

\begin{table}
\footnotesize
\begin{tabular}{r|c|c|c}
%\begin{tabular}{p{0.2\textwidth} | p{0.2\textwidth} p{0.2\textwidth} p{0.2\textwidth}}
%\begin{tabular}{l | p{3cm} p{3cm} p{3cm}}
\toprule
                                              Name &                           Arundine &              Chenodiol &                          Cysteamine \\
\midrule
\multicolumn{4}{c}{\scshape General information} \\
\midrule
                                           Synonym &               3,3-Diindolylmethane &  Chenodeoxycholic acid &                            Cystagon \\
                                         ChEMBL ID &                       CHEMBL446452 &           CHEMBL240597 &                           CHEMBL602 \\
                              Approved indications &          cervical cancer, phase 3 & cholesterol gallstones &            cystinosis, eye diseases \\
%                                  Alternative form &                                    &                        & Cysteamine-HCl$^\ast$ \\
\midrule
\multicolumn{4}{c}{\scshape Target genes} \\
\midrule
                                      %Gene targets &                                    &                        &                                     \\
   In present screen &
   RGS4 &            RGS4, SLC10A2 &                         RGS4, CYP3A4 \\
%                        ChEMBL, human (alt. form) &                                    &                        & GEMI, LMNA, PLK1, BLM, NF2L2, TYDP1 \\
                                             Other &                                AHR &                        &                                     \\
\midrule
\multicolumn{4}{c}{\scshape Relevance to Alzheimer's disease} \\
\midrule
                                             Rank (present screen)  &                                  4 &                     28 &                                  51 \\
%\midrule
%\multicolumn{4}{l}{\scshape AD-relevance, previous studies} \\
               \emph{In vivo} evidence & \cite{DeMiranda2013,DeMiranda2014} &       \cite{Keene2002} &                \cite{Cicchetti2019} \\
               \emph{In vitro} evidence &      \cite{Lee2019,Rzemieniec2019} &                        &          \cite{Besouw2013,Paul2019} \\
\bottomrule
\end{tabular}
\caption{
Selected drugs ranked high in our present computational screen of 2413 drugs
from ChEMBL.  Notes: Targets in the present screen were taken from ChEMBL filtering for human
experiments. ``Other'' targets refer to those we found manually in the
biomedical literature.
%$^\ast$Cysteamine-HCl (CHEMBL1256137) was ranked 323 and targets GEMI, LMNA, PLK1, BLM, NF2L2, and TYDP1 (ChEMBL, human).
}
\label{tab:selected-drugs}
\end{table}

%\subsection{Molecular pathways from the selected drugs to AD risk genes}
\subsection{Hypothetical molecular pathways linking selected drugs to AD}

Next we aimed at elucidating hypothetical, AD-relevant molecular mechanisms of
the selected drugs.  We identified all of the shortest network paths from each selected
drug to each AD risk gene set in the same interactome and drug-target network that
we used for our network proximity based drug screen.  The shortest path
length, defined as the least number of gene-gene interactions connecting a
given target to a given AD risk gene, ranged typically from 2 to 4 for most
combinations of target and AD risk gene set
(Fig.~S\ref{fig:shortest-path-lengths}).

\begin{figure}
\includegraphics[width=0.95\textwidth]{../../results/2022-09-26-selected-drugs/knowledge-dtn.sif.pdf}
\caption{Hypothetical molecular mechanisms linking drugs selected for
  experimental validation (red nodes), targets
  (salmon nodes) and AD risk genes (light blue nodes).  Targets interact with AD
  genes via mediators (salmon) like UBC (ubiquitin C), EGFR, or CALM1
  (calmodulin 1).  Such target$\rightarrow$mediator$\rightarrow$AD risk gene paths
  are highlighted with colored, thick edges.
}
\label{fig:drug-AD-genes-network}
\end{figure}

We primarily focused on paths of length 2 as these are both short and
frequently observed (Fig.~S\ref{fig:shortest-path-lengths}) thereby explaining
a drug's hypothetical beneficial effect in AD.  Such paths have the form:
$\mathrm{target} \rightarrow \mathrm{mediator} \rightarrow \mathrm{AD risk gene}$,
where \emph{mediator} is a gene transmitting the drug's presumed effect between a target
and an AD risk gene (Fig.~\ref{fig:drug-AD-genes-network} thick, colored edges).
The mediator interacting with the largest number of AD risk genes is UBC encoding Ubiquitin C
(Fig.~\ref{fig:drug-AD-genes-network} green edges, Table
S\ref{tab:mediators}). Mediator UBC interacts both with the Cysteamine target
CYP3A4 and with the Chenodiol target SLC10A2.  Target CYP3A4 has several
other mediators that interact with multiple AD risk genes: ubiquitin ligases STUB1
and AMFR as well as PRKCA/PRKACA encoding subunits of protein kinase-C
(Table S\ref{tab:mediators}, Fig.~\ref{fig:drug-AD-genes-network}).

Mediators for RGS4 include receptor tyrosine protein kinases EGFR and ERBB3,
as well as CALM1 (calmodulin 1), COPB1 (coatomer protein complex subunit
$\beta$), and genes encoding for various G protein subunits (Table
S\ref{tab:mediators}, Fig.~\ref{fig:drug-AD-genes-network}).  Among these
mediators EGFR interacts with the most AD risk genes, including well-established
knowledge based AD risk genes of crucial relevance to AD: APP (amyloid precursor
protein, amyloid-$\beta$), MAPT (tau), SNCA---$\alpha$-synuclein
(Fig.~\ref{fig:drug-AD-genes-network}); also including AD risk genes strongly
supported by functional genomics studies (TWAS2+ AD risk gene set): PICALM and
PTK2B.

A few paths were of the form $\mathrm{target} \rightarrow \mathrm{AD risk gene}$
(path length 1).  This was the case only for target RGS4 and 4 genes in 4
relatively large AD risk gene sets (Table~S\ref{tab:nearest-ADgenes}).  Three of
these four genes---ERBB3, GNAI2, CALM1---were noted above as mediators in the
context of other, smaller, AD risk gene sets.  Moreover, RGS4 was found to be an AD
gene itself (path length$=0$) that belongs to the two largest AD risk gene sets
(Table~S\ref{tab:nearest-ADgenes}).

(Note to Madhav: although you asked me to move the next paragraph to
Discussion, I would still keep it here because I already recapitulated and
expanded on it in the Discussion.)

Taken together, these observations suggest that our candidate AD drugs might modulate
multiple key molecular mechanisms: (1)
ubiquitination, (2) protein kinase-C dependent phosphorylation, (3) receptor
tyrosine kinase-dependent and (4) G-protein dependent signal transduction, (5)
Ca$^{2+}$-calmodulin, and (5) the coatomer complex.  We shall return to the
relevance of these mechanisms to AD in the present Discussion.

\subsection{Experimental validation}

To test the hypotheses that our candidate AD drugs exerted beneficial effects
on molecular mechanisms underlying AD pathogenesis, we performed cell
culture-based phenotypic assays  three candidate AD drugs: Arundine, Chenodiol
and the hydrochloride form of Cysteamine  (Fig.~\ref{fig:cell-based-assays}).

\begin{figure}
\includegraphics[scale=0.5]{../../notebooks/2022-09-21-cell-based-assays/named-figure/cell-based-assays-simplified.pdf}
\caption{Results of cell-based assays: testing AD-relevant protective effects of three
  selected top drugs from the present computational screen: Chenodiol,
  Cysteamine, and Arundine.
  Drug effect in a given assay is either statistically not significant (n.s,
  $p \ge 0.05$, yellow), or significantly protective (green) or adverse (red).
  %TODO: ask Tina about details: Throughout panels A-H tested drug
  %concentrations were $c_1 = ?$, $c_2 = ?$, $c_3 = ?$, $c_4 = ?$, $c_5 = ?$,
  %$c_6 = ?$.  Statistically significant effects are marked with $\ast$,
  %$\ast\ast$, $\ast\ast\ast$, and with increasingly deeper blue or red,
  %denoting $p < ?$, $p < ?$,  $p < ?$, respectively.
  A) A$\beta$ clearance in BV2 microglial cells; improved clearance is indicated by
  increased A$\beta$ in lysate, decreased A$\beta$ in supernatant and
  decreased supernatant:lysate ratio.
  B) Secretion of three A$\beta$ species in H4 neuroglioma cells.
  C) pTau231 and total Tau level in SH-SY5Y-hTau441(V337M/R406W) cell.
  D) Lipopolysaccharide (LPS) induced neuroinflammation in BV2 microglial cells; MTT
  assay of cell viability and the following cytokine levels: interleukins
  (IL-10, IL1b, IL-6), the chemokine KC/GRO, and tumor necrosis factor
  (TNF)-$\alpha$.
  E) Assays of neurogenesis and neurite outgrowth in primary cultured neurons.
  F) Assays of cell death, viability and apoptosis in response to trophic
  factor withdrawal in primary neurons.
}
\label{fig:cell-based-assays}
\end{figure}


Amyloid-$\beta$ (A$\beta$) 1--42 clearance in BV2 microglial cells was
enhanced by both Arundine and Chenodiol (Fig.~\ref{fig:cell-based-assays}A).
Fig.~\ref{fig:cell-based-assays}B shows that secretion of all three A$\beta$
species tested in H4 cells was significantly decreased by both Arundine and
Chenodiol (Fig.~\ref{fig:cell-based-assays}B); Cysteamine, on the other hand,
decreased secretion of only A$\beta$1-42.  Neither Tau phosphorylation,
measured as pTau231 level, nor total Tau was significantly altered by these
drugs (Fig.~\ref{fig:cell-based-assays}C).  The same conclusion is supported
by unchanged Tau aggregation in a cell-free assay (not shown). BV2 cells were
protected from lipopolysaccharide (LPS) induced neuroinflammation by Chenodiol
as measured by the MTT assay (Fig.~\ref{fig:cell-based-assays}D).  On the
other hand, Arundine reduced secretion of the pro-inflammatory cytokine
TNF-$\alpha$ in the LPS-induced neuroinflammation assay
(Fig.~\ref{fig:cell-based-assays}D), interleukins (IL-10, IL-1b, IL-6) were
not changed by any of the drugs except for a weakly significant increase by
Chenodiol (Fig.~\ref{fig:cell-based-assays}D) and also protected primary
neurons from any trophic factor withdrawal (PI cell death,
Fig.~\ref{fig:cell-based-assays}F); in contrast, Chenodiol and Cysteamine
actually promoted certain aspects of cell death (LDH, MTT,
PI~Fig.~\ref{fig:cell-based-assays}F).  None of the drugs impacted various
assays of neurogenesis and neurite outgrowth in primary cultured neurons
(Fig.~\ref{fig:cell-based-assays}E).

\section{Discussion}

In this study, we applied network pharmacology analyses to discover novel drug
repurposing opportunities in AD. .  We defined AD risk gene sets from diverse
sources, including knowledge bases and DEGs from transcriptomic studies as
well as single cell RNA-Seq data from APOE $\eps$4 and $\eps$3-iPSC derived microglia,
astrocytes, and neurons. We also included results of the most recent genomic
fine mapping studies (GWAS, TWAS and PWAS) in AD~\citep{Zhu2018,Lau2020}. Our
approach exploits the rich information in gene-gene and drug-target networks
to identify drugs that plausibly target causal disease pathways in AD. Using
this strategy, we nominated Arundine, Chenodiol and Cysteamine as candidate AD
drugs that act proximally to established AD risk genes. Furthermore, using
cell culture-based phenotypic assays, we show that these drugs ameliorate
distinct molecular perturbations relevant to AD pathogenesis further
validating our computational drug discovery methodology.

Our present work
also reveals several plausible molecular mechanisms for the potential
therapeutic actions of these drugs in AD
(Fig.~\ref{fig:drug-AD-genes-network}).  All three potential AD drugs target
RGS4 (Regulator of G-Protein Signalling 4) which is differentially expressed
in the AD brain.  RGS4 was found to be associated
with schizophrenia~\citep{Chowdari2002}, which shares substantial
heritability with AD~\citep{Consortium2018} and in which, similar to AD,
autoimmune pathways have been found to play a significant
role~\citep{Sekar2016a}.  Moreover, RGS4 is among the few hundred
differentially expressed genes in the brain in AD
\cite{Taubes2021}.

Our network analysis also revealed the receptor tyrosine kinase EGFR as the key
molecular link between RGS4 and crucial AD risk genes like APP (producing
A$\beta$), MAPT (tau), SNCA ($\alpha$-synuclein) and PICALM.  Interaction
between RGS4 and EGFR might play a role in the known crosstalk between EGFR
and GPCRs~\cite{Wang2016c}.  Moreoveer, EGFR's driver role in several forms of
cancer~\cite{Sigismund2018} converges with our obvervation that our present
drug repurposing screen identifies drugs previously tested or
approved not only for AD but also for various cancers.  While
the network proximity based screen in this study suggested EGFR as a novel
regulator of previously known AD risk genes, at the time of our analyses there was
little additional evidence for this notion.  A recent GWAS also
discovered AD-associated SNPs regulating EGFR expression, establishing EGFR as
a novel AD risk gene~\cite{Bellenguez2022}.  Besides EGFR, our network analysis implicates
other genes, notably CALM1 (calmodulin 1), that link the drug target gene RGS4 to
other AD risk genes.  Binding of calmodulin to ryanodine receptor was recently found
to be neuroprotective in AD~\cite{Nakamura2021}.

AHR (aryl hydrocarbon receptor) is known as another target of Arundine.  The
Arundine--AHR interaction was previously shown to protect mouse hippocampal
neurons against ischemia~\cite{Rzemieniec2019}.  We also found that SLC10A2
and CYP3A4---targeted by Chenodiol and Cysteamine, respectively---may act
through ubiquitination and, in the case of CYP3A4, also through PKC-dependent
phosphorylation.  Both mechanisms have been implicated in AD
pathogenesis~\cite{Hegde2019,Alfonso2016}.

Computational drug repurposing is a new, rapidly evolving field with a
multitude of varied approaches applicable to a growing, diverse multi-omic
datasets and curated knowledge~\citep{Pushpakom2019} with a few recent
applications to AD~\citep{Taubes2021,Fang2021}. A key strengths of our
computational drug screen is the utilization of multiple, diverse AD risk gene sets
including the most recent genomic fine mapping studies and single cell
transcriptomic data from iPSC-derived neurons, microglia and astrocytes.
Furthermore, our systems-based network pharmacology method leverages rich
information in available drug-target and AD risk gene-gene networks to identify
plausible AD drugs targeting putative causal disease pathways. Finally, our
demonstration that selected top-ranking drugs from this approach ameliorate
key molecular abnormalities in AD provides (i) further evidence of the utility
of our strategy and (ii) a biological rationale for further testing these
drugs as novel experimental AD treatments.

%\begin{enumerate}
%  \item We based our screen on a more diverse collection of AD risk gene sets than
%    either \cite{Taubes2021} or \cite{Fang2021}.  Like these authors, we also
%    used gene sets from knowledge bases and DEGs from transcriptomic studies.
%    But our work is unique in that it also leverages results of the most
%    recent genomic fine mapping studies (GWAS, TWAS and PWAS) on AD, which
%    detect causative AD risk genes with much higher specificity than DEG
%    analyses~\citep{Zhu2018,Lau2020}.
%\item We and~\cite{Fang2021} used a systems-based network pharmacology
%  approach~\citep{Guney2016,Cheng2018}, which exploits the rich information in
%  available drug-target and gene-gene networks to identify drugs that are
%  mechanistically supported to target putative disease pathways in AD.  In
%  contrast, \cite{Taubes2021} used CMap, a merely correlative
%  approach~\citep{Lamb2006}, which cannot incorporate existing knowledge on
%  the molecular pathomechanisms of AD.  Intrestingly, our present work also
%  shows that these two approaches to fully discordant, uncorrelated results.
%  Moreover, besides the aforementioned shortcoming of CMap, we find that its
%  various published implementations lead to quite different, albeit still
%  correlated, results.
%\item We aggregated results of multiple drug screens (each conditioned on a
%  different gene set) with a probabilistic rank aggregation algorithm with an
%  optimality criterion.  \cite{Fang2021} also carried multiple screens but
%  their aggregation was a less principled, ad-hoc method.  Notably, our rank
%  aggregation strategy also revealed that even some non-overlapping AD risk gene
%  sets resulted lead to the discovery of highly overlapping sets of very top
%  drugs demonstrating the robustness of our workflow.  The non-neuronal AD
%  gene sets were exception of this tendency pointing at the value of further
%  characterization of non-neuronal cells in
%  AD~\citep{Lopes2022,Mathys2019,DeStrooper2016}.
%\end{enumerate}

%Our computational validation, based on the rediscovery of drugs previously
%developed or repurposed for AD, revealed the enrichment of our top ranked
%drugs for also some common non-AD indications.  Genetic correlations between
%disease phenotypes~\citep{Consortium2018}, market-driven selection biases, and
%other factors may underlie not only this observation but also the known
%comorbidities of AD~\citep{Santiago2021}.  Such overlap between different
%drugs' indications is also consistent with the known relationship between drug
%repurposing and side effects~\citep{Ye2014}.  Further research is needed to
%dissect relationships between genetic correlations, comorbidities, side
%effects, protein and gene network topology, etc to be able to harness these and
%integrate into drug repurposing techniques.  That, in combination with
%expanding multi-celltype, multi-omic characterization of AD and its molecular
%subtypes~\citep{Neff2021}, will herald a new generation of drugs for
%Alzheimer's disease.

Some methodological limitations of our work that must be considered include
the inability of the AD risk gene sets utilized to capture longitudinal changes
reflecting the long preclinical prodromal phase of disease progression. We
also did not account for distinct pathological phenotypes of AD based on
transcriptomic profiling of the disease or the role of interactions between
specific cell types in influencing disease progression. 

In summary, we have applied a network pharmacology approach to identify novel
candidate AD treatments that target causal disease pathways and correct key
molecular abnormalities associated with AD pathogenesis.

%TODOs
%\begin{itemize}
%  \item Discussion: what have I missed?
%  \item Ask Vijay or Tina to decipher the drug concentrations for cell based
%    assays
%\end{itemize}

\section{Methods}

\subsection{AD risk gene sets}

Given an AD risk gene set, the network structure of the human interactome, and
another network formed by drugs and their targets (drug-target network), the
topological proximity of a given drug to the AD risk gene set can be
quantified\citep{Guney2016} and used as evidence for repurposing that drug for
AD.  Rather than compiling a single AD risk gene set from multiple studies, we
opted for evaluating the proximity of any drug separately for multiple AD risk gene
sets and aggregate the multiple proximity values subsequently (see
Discussion).  Table~\ref{tab:genesets} summarizes the AD risk gene sets used in
this study.  The genes contained in each set are listed in Table~S1.

\subsection{Previous and novel TWAS on AD}

We obtained AD risk gene sets from multiple TWAS.  Besides collecting AD risk genes from
previous transcriptome- and proteome-wide association studies (TWAS, PWAS) we
conducted our own applying the S-MultiXcan approach~\citep{Barbeira2018} to
the two largest AD-GWAS to date~\citep{Schwartzentruber2021,Wightman2021}.
S-MultiXcan borrows statistical strength from multiple tissue types; we used a
predictive model of gene expression built on all tissues of the GTEx data set
or on only brain tissues.

We conservatively took the top 30 genes and investigated their agreement with
top gene sets from seven previous TWAS and one previous PWAS (see Table~S2
listing genes contained in each set).  Hierarchical clustering of studies
based on shared genes (Fig.~S\ref{fig:twas-clustermap}) showed a fairly low
agreement for all pairs of studies.  The most similar study pair was our
present TWAS and the fine mapping of~\cite{Jansen2019} despite several
methodological differences between these two studies, which helps validate our
novel TWAS results.  As might be expected, the least similarity was found
between the only PWAS in our collection and all the other studies.

\subsection{Drug-target network}

We queried ChEMBL~\citep{Gaulton2017} (v29, July 2021) for phase 3 and 4 drugs
and those of their targets that satisfied the following conditions: (1) the
target is a human protein (2) its average $-\log_{10}\mathrm{Activity}$ or
$p\mathrm{Activity}$ is $\ge 5$, which corresponds to $\le 10 \mu M$
following~\cite{Cheng2018}.  Average $p\mathrm{Activity}$ activity is defined
as $\sum_{i=1}^n p\mathrm{Activity}_i \ge 5$, where $n$ is the number of
activity measurements for a given drug-protein pair and $\mathrm{Activity}_i$
is the standard value of the $i$-th measurement stored in the
activities.standard\_value SQL variable.  See
\href{https://github.com/attilagk/CTNS-notebook/blob/main/2021-10-24-chembl-query/drug_target_avg_activity.sql}{our SQLite script} for implementation.

The result of these operations is a set of 25,329 drug--target pairs, which constitute
the drug-target network.

\subsection{Gene-gene interaction network}

We used the network compiled by \cite{Cheng2019}, which consists 11133 nodes
(genes) and 217160 edges (gene-gene interactions).

\subsection{Network proximity calculations}

\cite{Guney2016} developed the network proximity method for in silico drug efficacy screening.  It
quantifies the topological proximity $d$ in the interactome of the set $T$ of a drug's target genes
to the set $S$ of disease genes: 

\begin{equation}
  d(S, T) = \frac{1}{|T|}\sum_{t \in T} \min_{s \in S} d(s, t)
\end{equation}
where $d(s,t)$ is the shortest path length between disease gene $s$ and drug
target $t$.  The equation can be interpreted as the average length of shortest
paths taken from all targets of a drug to all the disease genes.

We used \href{https://github.com/attilagk/guney_code}{our modified clone} of
the original, now obsolete, implementation of \cite{Guney2016}, which allows
running the code under Python~3.

\subsection{Rediscovery rate of drugs for a disease indication}

Let $T_k$ be the set of top-$k$ ranked drugs and $B_l$ that of the bottom-$l$
ranked drugs.  We define $\bar{\phi}_{T_k}(i)$, the average clinical phase of the top-$k$ drugs for
disease indication $i$, as follows:

\begin{equation}
  \bar{\phi}_{T_k}(i) = \frac{1}{k} \sum_{d \in T_k} \phi(d, i).
  \label{eq:phase-top-k}
\end{equation}

Moreover, we define $\bar{\phi}_{B_l}(i)$, the average clinical phase of the bottom-$l$
drugs for $i$ analogously.  Given some disease indication $i$, we then define the rediscovery rate of the top-$k$
drugs relative to that of the bottom-$l$ drugs as:
\begin{equation}
  r_l(k, i) = \frac{\bar{\phi}_{T_k}(i)}{\bar{\phi}_{B_l}(i)}
  \label{eq:rediscovery-rate}
\end{equation}

The $r_l(k, i)$ notation emphasizes the facts that
\begin{enumerate}
  \item rediscovery rate depends on the disease indication $i$ at hand, and
    that
  \item for
    Fig.~\ref{fig:ad-drug-rediscovery-multi}
    we held $l=1808$ for the bottom-$l$ drugs constant but varied $k$ for the
    top-$k$ drugs.
\end{enumerate}


\subsection{Rank aggregation}

As Fig.~\ref{fig:workflow} shows, we aggregated 8 lists of network
proximity-ranked drugs into a single list with rank aggregation algorithms.
We performed rank aggregation with the TopKLists R package
%\citep{Schimek2015}
several times, each time with different algorithm.  Then we assessed those
algorithm's performance with Modified Kendall Distance and found the MC3
algorithm to perform the best (Fig.~S\ref{fig:kendall-dist}).  Finally, we
used the MC3-aggregated list in our workflow.

\subsection{Cell-based assays}

We tested whether our selected drugs can rescue molecular phenotypes relevant
to Alzheimer's disease including tau phosphorylation, A$\beta$1-42 clearance,
A$\beta$ secretion, A$\beta$ toxicity, lipopolysaccharide (LPS)-induced
neuroinflammation, cell death due to trophic factor withdrawal and neurite
outgrowth and neurogenesis~\citep{Varma2020,Desai2022a}. In all assays
described below, cells treated with drugs were compared to either the vehicle
control (VC) or lesion control.

Statistical analysis was performed in GraphPad Prism. Group differences
were evaluated for each test item separately by one-way analysis of variance
(ANOVA) followed by Dunnett's multiple comparison test versus vehicle or
lesion control.

\subsubsection{A$\beta$ clearance}

For A$\beta$ clearance assay, 20,000 BV2 cells per well (uncoated 96-well
plates) were plated out. After changing cells to treatment medium, drug
compounds were administered 1 hour before A$\beta$ stimulation (Bachem,
4061966; final concentration in well, 200 ng/ml; dilutions in medium). Cells
treated with vehicle and cells treated with A$\beta$ alone served as controls.
After 3 hours of A$\beta$ stimulation, cell supernatants were collected for
the A$\beta$ measurement and cells were carefully washed twice with PBS and
thereafter lysed in 35 $\mu$l of cell lysis buffer (50 mM tris-HCl, pH 7.4,
150 mM NaCl, 5 mM EDTA, and 1\% SDS) supplemented with protease inhibitors.
Supernatants and cell lysates were analyzed for human A$\beta$42 with the MSD
V-PLEX Human A$\beta$42 Peptide (6E10) Kit (K151LBE, Mesoscale Discovery). The
immune assay was carried out according to the manual, and plates were read on
MESO QuickPlex SQ 120.

\subsubsection{A$\beta$ secretion}

H4-hAPP cells were cultivated in Opti-MEM supplemented with 10\% FCS, 1\%
penicillin/streptomycin, hygromycin B (200 $\mu$g/ml), and blasticidin S (2.5
$\mu$g/ml).  H4-hAPP cells were seeded into 96-well plates (2 $\times$ 104
cells per well). On the next day, cells in 96-well plates were treated with
compounds, reference item [400 nM
N-[N-(3,5-difluorophenacetyl-l-alanyl)]-S-phenylglycine t-butyl ester (DAPT)],
or vehicle. Twenty-four hours later, supernatants were collected for further
A$\beta$ measurements by MSD [V-PLEX A$\beta$ Peptide Panel 1 (6E10) Kit,
K15200E, Mesoscale Discovery].

\subsubsection{Tau phosphorylation}

SH-SY5Y-hTau441(V337M/R406W) cells were maintained in culture medium [DMEM
medium, 10\% FCS, 1\% nonessential amino acids (NEAA), 1\% l-glutamine,
gentamycin (100 $\mu$g/ml), and geneticin G-418 (300 $\mu$g/ml)] and
differentiated with 10 $\mu$M retinoic acid for 5 days changing medium every 2
to 3 days. Before the treatment, cells were seeded onto 24-well plates at a
cell density of 2 $\times$ 105 cells per well on day one of in vitro culture
(DIV1). Drug compounds were applied on DIV2. After 24 hours of incubation
(DIV3), cells on 24-well plates were harvested in 60 $\mu$l of RIPA buffer [50
mM tris (pH 7.4), 1\% NP-40, 0.25\% Na-deoxycholate, 150 mM NaCl, 1 mM EDTA
supplemented with freshly added 1 $\mu$M NaF, 0.2 mM Na-orthovanadate, 80
$\mu$M glycerophosphate, protease (Calbiochem), and phosphatase
(Sigma-Aldrich) inhibitor cocktail]. Protein concentration was determined by
BCA assay (Pierce, Thermo Fisher Scientific), and samples were adjusted to a
uniform total protein concentration. Total tau and phosphorylated tau were
determined by immunosorbent assay from Mesoscale Discovery
[Phospho(Thr231)/Total Tau Kit K15121D, Mesoscale Discovery].

\subsubsection{LPS-induced neuroinflammation}

The murine microglial cell line BV2 was cultivated in Dulbecco’s modified
Eagle's medium (DMEM) medium supplemented with 10\% fetal calf serum (FCS),
1\% penicillin/streptomycin, and 2 mM l-glutamine (culture medium). For LPS
stimulation assay, 5000 BV2 cells per well (uncoated 96-well plates) were
plated out and the medium was changed to treatment medium (DMEM, 5\% FCS, and
2 mM l-glutamine). After changing cells to treatment medium, drug compounds
were administered 1 hour before LPS stimulation [Sigma-Aldrich; L6529; 1 mg/ml
stock in ddH2O; final concentration in well, 100 ng/ml (dilutions in medium)].
Cells treated with vehicle, cells treated with LPS alone, and cells treated
with LPS plus reference item (dexamethasone, 10 $\mu$M; Sigma-Aldrich, D4902)
served as controls. After 24 hours of stimulation, cell supernatants were
collected for the cytokine measurement (V-PLEX Proinflammatory Panel 1 Mouse
Kit, K15048D, Mesoscale) and cells were subjected to
3-(4,5-dimethylthiazol-2-yl)-2,5-diphenyltetrazolium bromide (MTT) assay.

\subsubsection{Neurite outgrowth and neurogenesis}

Primary hippocampal neurons were prepared from E18.5 timed pregnant
C57BL/6JRccHsd mice as previously described. Cells were seeded in
poly-D-lysine pre-coated 96-well plates at a density of 2.6 $\times$ 104
cells/well in medium (Neurobasal,
2\% B-27, 0.5 mM glutamine, 25 $\mu$M glutamate,
1\% Penicillin–Streptomycin). Directly on DIV1, the drug of interest or VC was
applied. On DIV2, 10 $\mu$M Bromodeuxyuridine (BrdU; B5002 Sigma Aldrich) was
added and cells were fixed after additional 24h. Cells were permeabilized with
0.1\% Triton-X and incubated with primary Beta Tubulin Isotype III (T8660,
Sigma Aldrich) and BrdU antibodies (MAS25$\circ$C, AHrlan-Sera Lab) overnight
at 4$\circ$C. Afterwards, cells were washed two times with PBS and incubated
with fluorescently labelled secondary antibodies and DAPI for 1.5 h at room
temperature (RT) in the dark. Cells were rinsed three times with PBS and
imaged with the Cytation 5 Multimode reader (BioTek) at 10 $\times$
magnification (six images per well). BrdU-positive cells were counted as a
marker of neurogenesis and Beta Tubulin Isotype III signal was used for
macro-based quantification of neurite outgrowth.

\subsubsection{Trophic factor withdrawal}

Primary cortical neurons from embryonic day 18 (E18) C57Bl/6 mice were
prepared as previously described. On the day of preparation (DIV1), cortical
neurons were seeded on poly-d-lysine precoated 96-well plates at a density of
3 $\times$ 104 cells per well. Every 4 to 6 days, a half medium exchange using full
medium (Neurobasal, 2\% B-27, 0.5 mM glutamine, and 1\% penicillin-streptomycin)
was carried out. On DIV8, a full medium exchange to B-27 free medium
(Neurobasal, 0.5 mM glutamine, and 1\% penicillin-streptomycin) was performed
and drug compounds were applied thereafter. The experiment was carried out
with six technical replicates per condition, and vehicle-treated cells
served as control. After 28 hours on B-27–free medium, cells were subjected to
YO-PRO/propidium iodide (PI) and MTT as well as lactate dehydrogenase (LDH)
assay.


\bibliography{repurposing-ms}

\newpage

\section*{Supplementary Material}

\setcounter{table}{0}
\makeatletter 
\renewcommand{\tablename}{Table S} % nice
%\renewcommand{\figurename}{Table S} % nice
\makeatother

\setcounter{figure}{0}
\makeatletter 
\renewcommand{\figurename}{Figure S} % nice
\makeatother

\begin{table}[p]
  See file Table-S1.xlsx
\caption{
  Genes of the AD risk gene sets used as inputs to the present computational drug screen.
}
\label{tab:genes-in-genesets}
\end{table}

\begin{table}[p]
  See file Table-S2.csv
\caption{
  The 2413 drugs ranked according to their network proximity to each of the eight
  AD risk gene sets used as input.  The drugs' final, aggregate rank is also shown
  as well as their ChEMBL ID, standard InChI, indication class, and
  blood-brain-barrier permeability taken (if available) from the BBB
  database~\citep{Meng2021}.  Moreover, the UniProt name of each drug's
  targets is also indicated.
}
\label{tab:ranked-drugs}
\end{table}

\begin{table}[p]
\begin{tabular}{ll|rrrrrrrr}
\toprule
       &       &   \multicolumn{8}{c}{AD risk gene set$^\ast$} \\
       &       &   A &   B &   C &    D &    E &   F &   G &   H \\
target & mediator &     &     &     &      &      &     &     &     \\
\midrule
RGS4 & EGFR &   4 &   2 &   0 &    0 &    0 &   0 &   8 &   6 \\
       & CALM1 &   3 &   0 &   0 &    0 &    0 &   0 &   0 &   3 \\
       & GNAO1 &   1 &   0 &   0 &    0 &    0 &   0 &   1 &   1 \\
       & ERBB3 &   0 &   1 &   0 &    0 &    0 &   0 &   1 &   0 \\
       & GNAI1 &   0 &   0 &   0 &    0 &    0 &   0 &   1 &   1 \\
       & GNAQ &   0 &   0 &   0 &    0 &    0 &   0 &   2 &   0 \\
       & COPB1 &   1 &   0 &   0 &    0 &    0 &   0 &   0 &   0 \\
       & GNAI2 &   0 &   0 &   0 &    0 &    0 &   0 &   1 &   0 \\
\midrule
SLC10A2 & UBC &  16 &  19 &  39 &  157 &  162 &  34 &  60 &  97 \\
\midrule
CYP3A4 & UBC &  16 &  19 &  39 &  157 &  162 &  34 &  60 &  97 \\
       & PRKACA &   4 &   1 &   6 &   16 &   17 &   3 &   7 &   5 \\
       & PRKCA &   4 &   0 &   4 &   15 &   16 &   3 &   7 &   5 \\
       & STUB1 &   3 &   2 &   2 &    5 &    5 &   0 &   2 &   1 \\
       & AMFR &   0 &   0 &   1 &    2 &    2 &   0 &   0 &   0 \\
       & GK &   1 &   0 &   0 &    0 &    0 &   0 &   0 &   1 \\
       & CYB5A &   0 &   0 &   0 &    0 &    1 &   0 &   0 &   0 \\
\bottomrule
\end{tabular}
\caption{
  Number of AD risk genes $g$ interacting with a given mediator gene and forming
  network paths of the form: $\mathrm{drug} \rightarrow \mathrm{target}
  \rightarrow \mathrm{mediator} \rightarrow g$.  $^\ast$Code for AD risk gene sets:
  A: knowledge, B: TWAS2+, C: agora2+, D: AD DE APOE3-APOE3, E: AD DE
  APOE4-APOE4, F: APOE3-4 DE neuron, G: APOE3-4 DE astrocyte, H: APOE3-4 DE
  microglia.
}
\label{tab:mediators}
\end{table}

\begin{table}[p]
\begin{tabular}{cc||c|c|c}
\toprule
                       &                     &  \multicolumn{3}{c}{target}        \\
\midrule
                       &   {}                &  CYP3A4 &          RGS4 &  SLC10A2 \\
\midrule
shortest path length   & AD risk gene set         &  \multicolumn{3}{c}{AD risk genes}        \\
\midrule
                    0  &   AD DE APOE3-APOE3 &    -    &          RGS4 &    -     \\
                    0  &   AD DE APOE4-APOE4 &    -    &          RGS4 &    -     \\
\midrule
                    1  &   agora2+           &    -    &         ERBB3 &    -     \\
                    1  &   AD DE APOE3-APOE3 &    -    &         GNAI2 &    -     \\
                    1  &   AD DE APOE4-APOE4 &    -    &  CALM1; GNAI2 &    -     \\
                    1  &   APOE3-4 DE neuron &    -    &         PLCB1 &    -     \\
\bottomrule
\end{tabular}
\caption{
  AD risk genes $g$ that are either targets ($\mathrm{target} = g$; shortest path length $=0$) or directly
  interact with a target ($\mathrm{target} \rightarrow g$; shortest path length $=1$).
  Nearest AD risk gene(s) to a given target.  For paths of the form $\mathrm{target}
  \rightarrow \mathrm{mediator} \rightarrow
  \mathrm{mediator}_n \rightarrow g$ (shortest path length $= 2$) see Table
  S\ref{tab:mediators}.
}
\label{tab:nearest-ADgenes}
\end{table}

\begin{table}[p]
\begin{tabular}{llr}
\toprule
                              &                   &  ideal effect: increase (1) or decrease (-1)\\
class & assay &                                                                      \\
\midrule
A)  A$\beta$ clearance & Abeta in Lysate &                                                  1 \\
                              & Abeta in SN &                                                 -1 \\
                              & ratio &                                                 -1 \\
B)  A$\beta$ secretion & Ab1-38 &                                                 -1 \\
                              & Ab1-40 &                                                 -1 \\
                              & Ab1-42 &                                                 -1 \\
C)  Tau phosphorylation & pTau231 &                                                 -1 \\
                              & ratio &                                                 -1 \\
                              & total Tau &                                                  1 \\
D)  LPS neuroinflammation & IL-10 &                                                 -1 \\
                              & IL-1b &                                                 -1 \\
                              & IL-6 &                                                 -1 \\
                              & KC/GRO &                                                 -1 \\
                              & MTT &                                                  1 \\
                              & TNF-a &                                                 -1 \\
E)  Neurogenesis, neurite outgrowth & BrdU positive neurons &                                                  1 \\
                              & average longest neurite &                                                  1 \\
                              & number of branches &                                                  1 \\
                              & total neurite area &                                                  1 \\
F)  Trophic factor withdrawal & LDH (cell death) &                                                 -1 \\
                              & MTT (viability) &                                                  1 \\
                              & PI (cell death) &                                                 -1 \\
                              & YOPRO (apoptosis) &                                                 -1 \\
\bottomrule
\end{tabular}
\caption{
  Ideal, neuroprotective, effect for each assay.  Ideal effect is the
  direction of effect of an ideal, hypothetical, neuroprotective drug on a
  given assay.
}
\label{tab:protect-sign}
\end{table}

\begin{figure}[p]
\includegraphics[scale=0.4]{../../notebooks/2022-01-14-top-drugs/named-figure/rank-diff.pdf}
\includegraphics[scale=0.4]{../../notebooks/2022-01-14-top-drugs/named-figure/rank-diff-100.pdf}
\caption{
Difference of drug rank between the final, aggregated list and the list
resulting from each network proximity based drug screen.
}
\label{fig:rank-diff}
\end{figure}

%\begin{figure}[p]
%\includegraphics[scale=0.28]{../../notebooks/2021-12-02-proximity-various-ADgenesets/named-figure/prox-z-scatterplot_matrix.png}
%\caption{
%Z score of 2413 drugs (points) based on various input AD risk gene sets (rows and
%columns).
%}
%\label{fig:prox-z-scatterplot-m}
%\end{figure}

\begin{figure}
\includegraphics[scale=0.6]{../../notebooks/2021-07-01-high-conf-ADgenes/named-figure/cluster-experiments-genes.pdf}
\caption{
Similarity of TWA/PWA studies on AD in terms of shared genes.
}
\label{fig:twas-clustermap}
\end{figure}


\begin{figure}[p]
\includegraphics[scale=0.6]{../../notebooks/2021-12-02-proximity-various-ADgenesets/named-figure/jaccard-input-AD-sets.pdf}
\caption{
Eight AD risk gene sets were input to the workflow.
  Jaccard indices quantifying shared genes show that the sets are highly
  dissimilar to each other.
}
\label{fig:gset-jaccard}
\end{figure}


\begin{figure}[p]
\includegraphics[scale=0.6]{../../notebooks/2021-12-02-proximity-various-ADgenesets/named-figure/corr-coef-input-AD-sets.pdf}
\caption{
The network proximity scores of all
  screened drugs are fairly similar to each other across the different input
  AD risk gene sets, in spite of the marked dissimilarity among those.
}
\label{fig:gset-corr}
\end{figure}


%\begin{figure}[p]
%\includegraphics[scale=0.6]{../../notebooks/2021-11-30-proximity-cmap/named-figure/cmap-proxim-apoe4apoe4.png}
%\caption{
%Drug repurposing scores obtained by network proximity approach and CMap
%}
%\label{fig:proxim-cmap}
%\end{figure}

%\begin{figure}[p]
%\includegraphics[scale=0.4]{../../notebooks/2021-11-30-proximity-cmap/named-figure/cmap-proxim-celltype-apoe4-apoe4.png}
%\caption{
%Drug repurposing scores obtained by network proximity approach and CMap in
%given the cell type of the CMap expression profiling.
%}
%\label{fig:proxim-cmap-celltype}
%\end{figure}

%\begin{figure}[p]
%\includegraphics[scale=0.6]{../../notebooks/2021-10-04-CMap-discussion/named-figure/cmap-tools-consistency.pdf}
%\caption{
%Results under different implementations of CMap.
%TODO: remove Sudhir's?
%}
%\label{fig:cmap-cmap}
%\end{figure}

\begin{figure}[p]
\includegraphics[scale=0.4]{../../results/2022-01-14-rank-aggregation/kendall-distance.png}
\caption{
Several rank aggregation algorithms' performance assessed by modified Kendall
distance.
}
\label{fig:kendall-dist}
\end{figure}

%\begin{figure}[p]
%\includegraphics[scale=0.6]{../../notebooks/2022-01-14-top-drugs/named-figure/top-bottom-ratio-top-k.pdf}
%\caption{
%  Computational validation. }
%\label{fig:ad-drug-rediscovery}
%\end{figure}

\begin{figure}[p]
\includegraphics[scale=0.4]{../../notebooks/2022-09-26-selected-drugs/named-figure/shortest-path-lengths-gsets-hist_noHCl.pdf}
\caption{
  Shortest path length distribution from a given drug target (columns) to a given AD risk gene
  set (rows) across all genes in the set.
}
\label{fig:shortest-path-lengths}
\end{figure}


\end{document}
