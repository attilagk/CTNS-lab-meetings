%Version 3 October 2023
% See section 11 of the User Manual for version history
%
%%%%%%%%%%%%%%%%%%%%%%%%%%%%%%%%%%%%%%%%%%%%%%%%%%%%%%%%%%%%%%%%%%%%%%
%%                                                                 %%
%% Please do not use \input{...} to include other tex files.       %%
%% Submit your LaTeX manuscript as one .tex document.              %%
%%                                                                 %%
%% All additional figures and files should be attached             %%
%% separately and not embedded in the \TeX\ document itself.       %%
%%                                                                 %%
%%%%%%%%%%%%%%%%%%%%%%%%%%%%%%%%%%%%%%%%%%%%%%%%%%%%%%%%%%%%%%%%%%%%%

%%\documentclass[referee,sn-basic]{sn-jnl}% referee option is meant for double line spacing

%%=======================================================%%
%% to print line numbers in the margin use lineno option %%
%%=======================================================%%

%%\documentclass[lineno,sn-basic]{sn-jnl}% Basic Springer Nature Reference Style/Chemistry Reference Style

%%======================================================%%
%% to compile with pdflatex/xelatex use pdflatex option %%
%%======================================================%%

%%\documentclass[pdflatex,sn-basic]{sn-jnl}% Basic Springer Nature Reference Style/Chemistry Reference Style


%%Note: the following reference styles support Namedate and Numbered referencing. By default the style follows the most common style. To switch between the options you can add or remove �Numbered� in the optional parenthesis. 
%%The option is available for: sn-basic.bst, sn-vancouver.bst, sn-chicago.bst%  
 
%%\documentclass[sn-nature]{sn-jnl}% Style for submissions to Nature Portfolio journals
%%\documentclass[sn-basic]{sn-jnl}% Basic Springer Nature Reference Style/Chemistry Reference Style
\documentclass[sn-mathphys-num]{sn-jnl}% Math and Physical Sciences Numbered Reference Style 
%%\documentclass[sn-mathphys-ay]{sn-jnl}% Math and Physical Sciences Author Year Reference Style
%%\documentclass[sn-aps]{sn-jnl}% American Physical Society (APS) Reference Style
%%\documentclass[sn-vancouver,Numbered]{sn-jnl}% Vancouver Reference Style
%%\documentclass[sn-apa]{sn-jnl}% APA Reference Style 
%%\documentclass[sn-chicago]{sn-jnl}% Chicago-based Humanities Reference Style

%%%% Standard Packages
%%<additional latex packages if required can be included here>

\usepackage{graphicx}%
\usepackage{multirow}%
\usepackage{amsmath,amssymb,amsfonts}%
\usepackage{amsthm}%
\usepackage{mathrsfs}%
\usepackage[title]{appendix}%
\usepackage{xcolor}%
\usepackage{textcomp}%
\usepackage{manyfoot}%
\usepackage{booktabs}%
\usepackage{algorithm}%
\usepackage{algorithmicx}%
\usepackage{algpseudocode}%
\usepackage{listings}%
%%%%

%%%%%=============================================================================%%%%
%%%%  Remarks: This template is provided to aid authors with the preparation
%%%%  of original research articles intended for submission to journals published 
%%%%  by Springer Nature. The guidance has been prepared in partnership with 
%%%%  production teams to conform to Springer Nature technical requirements. 
%%%%  Editorial and presentation requirements differ among journal portfolios and 
%%%%  research disciplines. You may find sections in this template are irrelevant 
%%%%  to your work and are empowered to omit any such section if allowed by the 
%%%%  journal you intend to submit to. The submission guidelines and policies 
%%%%  of the journal take precedence. A detailed User Manual is available in the 
%%%%  template package for technical guidance.
%%%%%=============================================================================%%%%

%% as per the requirement new theorem styles can be included as shown below
\theoremstyle{thmstyleone}%
\newtheorem{theorem}{Theorem}%  meant for continuous numbers
%%\newtheorem{theorem}{Theorem}[section]% meant for sectionwise numbers
%% optional argument [theorem] produces theorem numbering sequence instead of independent numbers for Proposition
\newtheorem{proposition}[theorem]{Proposition}% 
%%\newtheorem{proposition}{Proposition}% to get separate numbers for theorem and proposition etc.

\theoremstyle{thmstyletwo}%
\newtheorem{example}{Example}%
\newtheorem{remark}{Remark}%

\theoremstyle{thmstylethree}%
\newtheorem{definition}{Definition}%

\raggedbottom
%%\unnumbered% uncomment this for unnumbered level heads

\UseRawInputEncoding

\begin{document}

\title[Article Title]{Network pharmacology-based discovery and experimental validation of novel drug repurposing candidates in Alzheimer's Disease}

%%=============================================================%%
%% GivenName	-> \fnm{Joergen W.}
%% Particle	-> \spfx{van der} -> surname prefix
%% FamilyName	-> \sur{Ploeg}
%% Suffix	-> \sfx{IV}
%% \author*[1,2]{\fnm{Joergen W.} \spfx{van der} \sur{Ploeg} 
%%  \sfx{IV}}\email{iauthor@gmail.com}
%%=============================================================%%
\author*[1]{\fnm{Attila} \sur{Jones}}\email{attila.g.jones@gmail.com}

\author[2]{\fnm{Tina} \sur{Loeffler}}\email{tlo@scantox.com}
%\equalcont{These authors contributed equally to this work.}

\author[3]{\fnm{Evan} \sur{Wu}}\email{evanwu@uchicago.edu}
%\equalcont{These authors contributed equally to this work.}

\author[1]{\fnm{Vijay R.} \sur{Varma}}\email{vijay.varma01@gmail.com}

\author[3]{\fnm{Hae Kyung} \sur{Im}}\email{haky@uchicago.edu}

\author*[1]{\fnm{Madhav} \sur{Thambisetty}}\email{madhavtr71@gmail.com}

%\affil*[1]{\orgdiv{Department}, \orgname{Organization}, \orgaddress{\street{Street}, \city{City}, \postcode{100190}, \state{State}, \country{Country}}}

%\affil[2]{\orgdiv{Department}, \orgname{Organization}, \orgaddress{\street{Street}, \city{City}, \postcode{10587}, \state{State}, \country{Country}}}

%\affil[3]{\orgdiv{Department}, \orgname{Organization}, \orgaddress{\street{Street}, \city{City}, \postcode{610101}, \state{State}, \country{Country}}}


\clearpage

%%==================================%%
%% Sample for unstructured abstract %%
%%==================================%%

%%================================%%
%% Sample for structured abstract %%
%%================================%%

% \abstract{\textbf{Purpose:} The abstract serves both as a general introduction to the topic and as a brief, non-technical summary of the main results and their implications. The abstract must not include subheadings (unless expressly permitted in the journal's Instructions to Authors), equations or citations. As a guide the abstract should not exceed 200 words. Most journals do not set a hard limit however authors are advised to check the author instructions for the journal they are submitting to.
% 
% \textbf{Methods:} The abstract serves both as a general introduction to the topic and as a brief, non-technical summary of the main results and their implications. The abstract must not include subheadings (unless expressly permitted in the journal's Instructions to Authors), equations or citations. As a guide the abstract should not exceed 200 words. Most journals do not set a hard limit however authors are advised to check the author instructions for the journal they are submitting to.
% 
% \textbf{Results:} The abstract serves both as a general introduction to the topic and as a brief, non-technical summary of the main results and their implications. The abstract must not include subheadings (unless expressly permitted in the journal's Instructions to Authors), equations or citations. As a guide the abstract should not exceed 200 words. Most journals do not set a hard limit however authors are advised to check the author instructions for the journal they are submitting to.
% 
% \textbf{Conclusion:} The abstract serves both as a general introduction to the topic and as a brief, non-technical summary of the main results and their implications. The abstract must not include subheadings (unless expressly permitted in the journal's Instructions to Authors), equations or citations. As a guide the abstract should not exceed 200 words. Most journals do not set a hard limit however authors are advised to check the author instructions for the journal they are submitting to.}

%\keywords{keyword1, Keyword2, Keyword3, Keyword4}

%%\pacs[JEL Classification]{D8, H51}

%%\pacs[MSC Classification]{35A01, 65L10, 65L12, 65L20, 65L70}

\maketitle

\section*{Supplemental Information}

\setcounter{table}{0}
\makeatletter 
\renewcommand{\tablename}{Table S} % nice
%\renewcommand{\figurename}{Table S} % nice
\makeatother

\setcounter{figure}{0}
\makeatletter 
\renewcommand{\figurename}{Figure S} % nice
\makeatother

\begin{table}[h]
  \begin{tabular}{p{0.8\textwidth}}
  See file Table-S1.xlsx \\
  \end{tabular}
\caption{
  Genes of the AD risk gene sets used as inputs to the present computational drug screen.
}
\label{tab:genes-in-genesets}
\end{table}

\begin{table}[h]
  \begin{tabular}{p{0.8\textwidth}}
  See file Table-S2.csv \\
  \end{tabular}
\caption{
  The 2413 drugs ranked according to their network proximity to each of the eight
  AD risk gene sets used as input.  The drugs' final, aggregate rank is also shown
  as well as their ChEMBL ID, standard InChI, indication class, and
  blood-brain-barrier permeability taken (if available) from the BBB
  database~\cite{Meng2021}.  Moreover, the UniProt name of each drug's
  targets is also indicated.
}
\label{tab:ranked-drugs}
\end{table}

\begin{table}[h]
\begin{tabular}{cc||c|c|c}
\toprule
                       &                     &  \multicolumn{3}{c}{target}        \\
\midrule
                       &   {}                &  CYP3A4 &          RGS4 &  SLC10A2 \\
\midrule
shortest path length   & AD risk gene set         &  \multicolumn{3}{c}{AD risk genes}        \\
\midrule
                    1  &   AD DE APOE3-APOE3 &    -    &          RGS4 &    -     \\
                    1  &   AD DE APOE4-APOE4 &    -    &          RGS4 &    -     \\
\midrule
                    2  &   agora2+           &    -    &         ERBB3 &    -     \\
                    2  &   AD DE APOE3-APOE3 &    -    &         GNAI2 &    -     \\
                    2  &   AD DE APOE4-APOE4 &    -    &  CALM1; GNAI2 &    -     \\
                    2  &   APOE3-4 DE neuron &    -    &         PLCB1 &    -     \\
\bottomrule
\end{tabular}
\caption{
Targets of the drugs Arundine, Cysteamine or Chenodiol: CYP3A4, RGS4 and
SLC10A2 and network paths going from the drugs through the targets to AD genes
in some AD gene set. When the target is itself an AD gene, the shortest path
length is 1.� When the target directly interacts with an AD gene, the path
length is 2.� We see that RGS4 is the only target that fulfils either
criterion.
}
\label{tab:nearest-ADgenes}
\end{table}

\begin{table}[h]
\begin{tabular}{lll}
\toprule
experiment & assay & desired effect  \\
\midrule
%\multirow[t]{10}{*}{LPS neuroinflammation (BV2 cells)} & IFN-$\gamma$ & decrease \\
LPS neuroinflammation (BV2 cells) & IFN-$\gamma$ & decrease \\
 & IL-10 & decrease \\
 & IL-12p70 & decrease \\
 & IL-1$\beta$ & decrease \\
 & IL-2 & decrease \\
 & IL-4 & decrease \\
 & IL-5 & decrease \\
 & IL-6 & decrease \\
 & KC/GRO & decrease \\
 & TNF-$\alpha$ & decrease \\
\hline
A$\beta$ toxicity (primary neurons) & MTT & increase \\
\hline
A$\beta$ release (H4 cells) & A$\beta$38 & decrease \\
 & A$\beta$40 & decrease \\
 & A$\beta$42 & decrease \\
\hline
A$\beta$ clearance (BV2 cells) & A$\beta$42 SN & decrease \\
 & A$\beta$42 Ly & increase \\
\hline
Trophic factor withdrawal (primary neurons) & PI & decrease \\
 & YOPRO & decrease \\
 & MTT & increase \\
 & LDH & decrease \\
\hline
Tau phosphorylation & Tau & increase \\
 & pTau (T231) & decrease \\
 & pT/T ratio & decrease \\
\hline
Neurite outgrowth (primary neurons) & $\sum$ neurite area & increase \\
 & branch points & increase \\
 & neurogenesis & increase \\
 & longest neurite & increase \\
\hline
A$\beta$ clearance (iPSC) & pHrodo-4h & increase \\
 & supernatant & decrease \\
\hline
LPS neuroinflammation (iPSC) & IL-1$\beta$ & decrease \\
 & IL-6 & decrease \\
 & IL-8 & decrease \\
 & MTT & decrease \\
 & TNF-$\alpha$ & decrease \\
\hline
\bottomrule
\end{tabular}
\caption{
  Desired, neuroprotective, effect for each assay.  Desired effect is the
  direction of effect of an ideal, hypothetical, neuroprotective drug on a
  given assay.
}
\label{tab:protect-sign}
\end{table}

\clearpage

\begin{figure}[h]
\includegraphics[scale=0.4]{../../notebooks/2022-01-14-top-drugs/named-figure/rank-diff.pdf}
\includegraphics[scale=0.4]{../../notebooks/2022-01-14-top-drugs/named-figure/rank-diff-100.pdf}
\caption{
Difference of drug rank between the final, aggregated list and the list
resulting from each network proximity based drug screen.
}
\label{fig:rank-diff}
\end{figure}

\begin{figure}[h]
\includegraphics[scale=0.6]{../../notebooks/2021-07-01-high-conf-ADgenes/named-figure/cluster-experiments-genes.pdf}
\caption{
Similarity of TWA/PWA studies on AD in terms of shared genes.
}
\label{fig:twas-clustermap}
\end{figure}

\begin{figure}[h]
\includegraphics[scale=0.6]{../../notebooks/2021-12-02-proximity-various-ADgenesets/named-figure/jaccard-input-AD-sets.pdf}
\caption{
Eight AD risk gene sets were input to the workflow.
  Jaccard indices quantifying shared genes show that the sets are highly
  dissimilar to each other.
}
\label{fig:gset-jaccard}
\end{figure}

\begin{figure}[h]
\includegraphics[scale=0.6]{../../notebooks/2021-12-02-proximity-various-ADgenesets/named-figure/corr-coef-input-AD-sets.pdf}
\caption{
The network proximity scores of all
  screened drugs are fairly similar to each other across the different input
  AD risk gene sets, in spite of the marked dissimilarity among those.
}
\label{fig:gset-corr}
\end{figure}

\begin{figure}[h]
\includegraphics[scale=0.4]{../../results/2022-01-14-rank-aggregation/kendall-distance.png}
\caption{
Several rank aggregation algorithms' performance assessed by modified Kendall
distance.
}
\label{fig:kendall-dist}
\end{figure}

\begin{figure}[h]
\includegraphics[scale=0.4]{../../notebooks/2022-01-14-top-drugs/named-figure/top-bottom-ratio-top-k-multi-1.pdf}
\caption{
Network proximity-based rediscovery of drugs in phase 1--4 clinical trials for
AD and 35 other disease indications.  Disease indications written on top of
individual plots.  Orange ticks: drugs that are in phase 1 or more advanced
clinical study for the given indication. Blue dots: rediscovery rate
(Eq.~3, Methods}) for the top-$k$ drugs, ranked by network
proximity, relative to that for the bottom-$1808$ drugs.  Rediscovery rate of
$>1$ means that top-ranked drugs tend to be those that are in phase 1--4
clinical trials for the given indication; these drugs are marked by orange
symbols above the $x$ axes.
}
\label{fig:ad-drug-rediscovery-multi}
\end{figure}

\begin{figure}[h]
\includegraphics[scale=0.4]{../../notebooks/2022-09-26-selected-drugs/named-figure/shortest-path-lengths-gsets-hist_noHCl.pdf}
\caption{
  Shortest path length distribution from a given drug target (columns) to a given AD risk gene
  set (rows) across all genes in the set.
}
\label{fig:shortest-path-lengths}
\end{figure}

\begin{figure}[h]
\includegraphics[height=0.5\textheight]{../../notebooks/2023-09-13-cell-based-assays-bayes/named-figure/sigmoid-2.png}
\caption{Dependency graph of the Bayesian nonlinear regression model used to
  fit dose-response ($x$--$y$) data from cell-based assays.
  For more details see~Eq.4 in Methods.  This model is named ``sigmoid 2'' in
  \url{https://github.com/attilagk/CTNS-notebook/blob/main/src/cellbayesassay.py}
  (see the definition of the sample\_sigmoid\_2 function therein).
}
\label{fig:model-2-plate}
\end{figure}

\begin{figure}[h]
\includegraphics[]{figures/prior-posterior-fit-H102.pdf}
\caption{Estimating the prior and posterior probability of the protective, neutral
  and adverse effect of a drug in a cell-based assay.  (A) Sigmoid regression
  curves sampled from the prior or posterior distribution of model parameters.
  The desired direction of drug effect (decrease in this particular assay) and
  thresholds $t_1<1$ and $t_2>1$ together define the three hypotheses of
  interest: protective ($H_1$), neutral ($H_0$) and adverse ($H_2$) drug
  effect.  The average fold change of expected bioactivity $\mathrm{FC}_y$ at
  saturating concentration, $y_1=\mathrm{FC}_y y_0$, relative to the expected activity
  at zero drug concentration, $y_0$, is marked by a dashed horizontal line
  given a sample from the prior or posterior distribution.  (B) Prior and
  posterior probability density of fold change for the same drug and assay as
  in (A).  Observe how the posterior density is shifted towards the desired
  direction and has a decreased standard deviation (vertical error bar)
  relative to the prior density.  Note that the prior density (as well as
  $t_1, t_2$) was chosen such
  that the mean fold change is 1 and such that the prior probability of
  protective ($H_1$) and adverse ($H_2$) effect is both 0.2, whereas that of
  neutral ($H_0$) effect is 0.6.
}
\label{fig:prior-posterior-fit-H102}
\end{figure}

\begin{figure}[h]
\includegraphics[scale=0.5]{../../notebooks/2023-09-26-cell-bayes-assays/named-figure/violin-posterior-pdf-legend.pdf}
\caption{Posterior probability density of drug-induced fold change in
  bioactivity across three candidate
  drugs and 33 cell-based assays. Blank (white) slots indicate either that the experiment was not
  performed on the drug or that model fit was poor.  One or more assays constitute an experiment
  (see parenthesized one-letter codes and the right center legend).  Each
  posterior density has a green, gray and red component corresponding to
  protective, neutral and adverse drug effect given the direction of the
  desired drug effect and fold-change thresholds $t_1, t_2$ (see
  Fig.~S\ref{fig:prior-posterior-fit-H102}).  The posterior mean (and standard
  deviation) of fold change is shown as a yellow circle (and yellow error
  bars).
}
\label{fig:violin-posterior-assays}
\end{figure}

\begin{figure}[h]
\includegraphics[scale=0.5]{../../notebooks/2023-09-26-cell-bayes-assays/named-figure/H102_posteriors-barchart-e2l_textbox.pdf}
\caption{Posterior probability of protective, neutral and adverse effect
  for Arundine, Cysteamine
  and TUDCA.
}
\label{fig:bar-posterior-assays}
\end{figure}

\end{appendices}
%%===========================================================================================%%
%% If you are submitting to one of the Nature Portfolio journals, using the eJP submission   %%
%% system, please include the references within the manuscript file itself. You may do this  %%
%% by copying the reference list from your .bbl file, paste it into the main manuscript .tex %%
%% file, and delete the associated \verb+\bibliography+ commands.                            %%
%%===========================================================================================%%

%\bibliography{sn-bibliography}% common bib file
\bibliography{repurposing-ms}
%% if required, the content of .bbl file can be included here once bbl is generated
%%\input sn-article.bbl

\end{document}
