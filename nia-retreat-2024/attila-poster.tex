\documentclass{tikzposter} %Options for format can be included here
%\documentclass[a1paper,16pt]{tikzposter} %Options for format can be included here

\geometry{paperwidth=48in,paperheight=48in}
\makeatletter
\setlength{\TP@visibletextwidth}{\textwidth-2\TP@innermargin}
\setlength{\TP@visibletextheight}{\textheight-2\TP@innermargin}
\makeatother

\usepackage{tikz}
\usepackage{multicol}
\setlength{\columnsep}{0.7cm}
\usepackage{natbib}
\bibliographystyle{plainnat}
%\bibliographystyle{unsrtnat}
\author{Attila Jones, Madhav Thambisetty}
\title{Drug Repurposing to Alzheimer’s from Genomics}

\institute{Clinical \& Translational Neuroscience Section, LBN}

 %Choose Layout
\usetheme{default}
%\usecolortheme{seagull}

\begin{document}
\maketitle

\begin{columns}
\column{0.5}

\block{1. Network proximity for computational drug repurposing}{
\begin{center}
\includegraphics[width=0.2\textwidth]{../../../figures/from-others/drug-repurposing-valli-et-al-2020.png}
\includegraphics[width=0.2\textwidth]{../../../figures/by-me/drug-target-disgene-network/drug-target-disgene-network-scores.pdf}
\end{center}

Top left: Drug repurposing.  Top right: scoring approved or investigational
drugs according to their topological network proximity to AD genes.  Bottom:
three high-scoring drugs, A-C, their targets, mediators and AD genes.

\begin{center}
\includegraphics[width=0.35\textwidth]{../../results/2022-09-26-selected-drugs/knowledge-dtn-anonymized.sif.png}
\end{center}
}

\block{2. Bayesian dose-response analysis}{
\begin{center}
\includegraphics[width=0.25\columnwidth]{../../../figures/from-others/BV2_DC_10PS_20X_2-20211206160943680-wide.png}
\end{center}

Top: AD relevant assays in BV2 and other cell lines.

Bottom: comparison of
frequentist hypothesis test at each drug dose with Bayesian regression based
hypothesis test.

\includegraphics[width=0.225\textwidth]{../../notebooks/2024-02-14-cell-bayes/named-figure/feq-bayes-dose-response-TI1.pdf}
\includegraphics[width=0.225\textwidth]{../../notebooks/2024-02-14-cell-bayes/named-figure/feq-bayes-dose-response-TI6.pdf}

\emph{Acknowledgements}:
We thank Vijay Varma, Yang An, Jackson Roberts for feedback and the NIH HPC
group for excellent service.
}

\column{0.5}

%\block{5. Computational validation}{
%\begin{center}
%\includegraphics[width=0.35\columnwidth]{../../../CTNS/notebooks/2022-01-14-top-drugs/named-figure/top-bottom-ratio-top-k.pdf}
%\end{center}
%
%Here we validate our network proximity based drug ranking by showing that the
%top-$t$ drugs are enriched in drugs developed for Alzheimer's.
%}

%\bibliography{attila-poster}

\block{3. Protective effects, cell-based assays }{

\begin{center}
\includegraphics[width=0.4\textwidth]{../../notebooks/2023-09-26-cell-bayes-assays/named-figure/H10-bayes-factors-anonym_drugs.pdf}
\end{center}

Protective effect of drug A and C in experiments a), c), f) as evidenced by
the Bayes factor (BF). The BF is analogous to the $p$ value but, unlike the $p$
value, the BF also expresses effect size.
}

\block{4. Protective effect in mice }{
\begin{center}
\includegraphics[width=0.4\textwidth]{../../notebooks/2023-11-15-5xfad-nfl-gfap/named-figure/NF-L-in-vivo-treatments-tg-TUDCA-anonym.pdf}
\end{center}

Drug C was given to 5xFAD mice and was found to protect against axonal
degeneration as measured by plasma Nfl \emph{in vivo}.
}

%\block{7. Acknowledgements}{
%We thank Vijay Varma, Yang An, Jackson Roberts for feedback and the NIH HPC
%group for excellent service.}

\end{columns}


\end{document}
