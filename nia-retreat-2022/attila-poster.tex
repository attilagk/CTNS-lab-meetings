\documentclass[17pt,a1paper]{tikzposter} %Options for format can be included here
\usepackage{tikz}
\usepackage{multicol}
\setlength{\columnsep}{0.7cm}
\usepackage{natbib}
\bibliographystyle{plainnat}
%\bibliographystyle{unsrtnat}
\author{Attila Jones$^1$, Evan Wu$^2$, Sudhir Varma, Hae Kyung Im$^2$, Madhav Thambisetty$^1$}
\title{Drug Repurposing Opportunities for Alzheimer’s Disease from
	Multiomics}

\institute{$1$: Clinical \& Translational Neuroscience Section, LBN, $2$: University
	of~Chicago}

 %Choose Layout
\usetheme{Default}

\begin{document}
\maketitle

\block{1. Introduction}{
\begin{multicols}{2}
We recently announced the Drug Repurposing for Effective
Alzheimer's Medicines (DREAM) study
%\citep{Desai2020}
, in which we plan to
find repurposable drugs based on:

\begin{enumerate}
\item our novel proteomics study
	%\citep{JacksonA.Roberts2021}
	that describes
	the \emph{incipient AD proteomic signature}
\item transcriptome-wide association studies, TWAS, on AD
	%\citep{Baird2021,Gerring2020,Jansen2019a,Kunkle2019a}
\item curated knowledge on established AD genes %\citep{PletscherFrankild2015}
\item a network based drug repurposing approach %\citep{Guney2016,Cheng2018}
\end{enumerate}

Below we provide an overview on the approach and on the input data sets.
\end{multicols}
}

\begin{columns}
\column{0.55}
\block{2. Network proximity predicts efficacy}{
\begin{multicols}{2}
	Network proximity (Guney et al Nat Commun Vol 7, No 26831545)
%\citep{Cheng2018}
is a quantity defined in terms of gene interaction network
topology between some drug's targets (\tikz \node[square, fill=blue]{};)  and
disease genes (\tikz \node[circle, fill=red!80!black]{}; or \tikz \node[circle,
fill=purple!70!black]{};)
We choose the approach to drug repurposing developed by
based on their earlier work
%\citep{Guney2016}.

showed that:
%\begin{itemize}
%\item cardiovascular (CV) disease genes form clusters, so-called \emph{disease
%	modules}, in the human protein-protein interactome (PPI)
%\item a few drugs approved for non-CV indications, like Hydroxychloroquine,
%	are proximal to the CAD (coronary artery disease) module
%\item Hydroxychloroquine's protective effect on CAD is supported by
%	pharmaco-epidemiology and an \emph{in vitro} assay
%%\item these evidence suggests Hydroxychloroquine might be repurposed for CAD
%\end{itemize}

\end{multicols}
\includegraphics[width=0.3\columnwidth]{../../../figures/from-others/cheng-desai-2018-Nat-comm-Fig4a.png}
}

\block{3. Workflow}{

\includegraphics[width=0.5\columnwidth]{../../../figures/by-me/repurposing-study-desing/repurposing-study-design.pdf}
}

\column{0.45}

\block{4. Top selected drugs}{
}

\block{5. Validation}{
%\begin{tabular}{rl}
%Knowledge & established AD genes in knowledge bases complied by~\cite{PletscherFrankild2015} \\
%TWAS & AD genes whose expression is inferred causal to AD by Mendelian randomization and similar techniques \\
%Incipient proteo. & the \emph{incipient AD proteomic signature} described by us~\citep{JacksonA.Roberts2021} \\
%\end{tabular}
\begin{center}
\includegraphics[width=0.35\columnwidth]{../../../CTNS/notebooks/2022-01-14-top-drugs/named-figure/top-bottom-ratio-top-k.pdf}
\end{center}
}

\block{}{
\small
\bibliography{library}
}

\end{columns}

\end{document}
