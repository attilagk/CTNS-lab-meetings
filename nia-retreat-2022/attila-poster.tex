\documentclass[a1paper]{tikzposter} %Options for format can be included here
\usepackage{tikz}
\usepackage{multicol}
\setlength{\columnsep}{0.7cm}
\usepackage{natbib}
\bibliographystyle{plainnat}
%\bibliographystyle{unsrtnat}
\author{Attila Jones$^1$, Evan Wu$^2$, Sudhir Varma, Hae Kyung Im$^2$, Madhav Thambisetty$^1$}
\title{Drug Repurposing to Alzheimer’s from Multiomics}

\institute{$1$: Clinical \& Translational Neuroscience Section, LBN, $2$: University
	of~Chicago}

 %Choose Layout
\usetheme{Default}

\begin{document}
\maketitle

\block{1. Introduction}{
	%\Large
	Alzheimer's is a polygenic disease without efficient treatment.
Repurposing drugs from other disease indications to Alzheimer's is an
attractive approach but it comes with the challenge of finding 
repurposeable drugs.  In this work we score ca.~2,400 drugs for their
repurposeability using multiomic information sources and a systems-based,
mechanistic network model.
}

\begin{columns}
\column{0.55}
\block{2. Network proximity predicts efficacy}{
	Network proximity
is a quantity defined between some drug's targets (\tikz \node[square,
fill=blue]{}; below)  and
disease genes (\tikz \node[circle, fill=red!80!black]{}; or \tikz \node[circle,
fill=purple!70!black]{};) in terms of gene interaction network
topology.
Previous work suggested that network proximity can
predict drug efficacy and thus can be used to screen drugs for repurposing
oportunities.

{\small
	Nature Comm Vol 7 26831545, Nature Comm Vol 9 30002366
}

\includegraphics[width=0.3\columnwidth]{../../../figures/from-others/cheng-desai-2018-Nat-comm-Fig4a.png}
}

\block{3. Workflow}{
\includegraphics[width=0.45\columnwidth]{../../../figures/by-me/repurposing-study-desing/repurposing-study-design.pdf}

Our workflow, centered at network proximity calculations, uses a rich set of
information from multiple omic modalities: genome and transcriptome-wide
association studies (GWAS, TWAS), transcriptomics (differentially expressed
genes), and interactomics (gene-gene and drug-target networks).
}

\column{0.45}

\block{4. Selected top drugs}{
\begin{tabular}{rll}
	rank & indication class    & targets \\
	\hline
	4    &                     & RGS4, GPR84 \\
	28   & Anticholelithogenic & RGS4, NTCP2 \\
	51   & Anti-Urolithic      & CP3A4, RGS4 \\
	322  & Anti-Urolithic      & GEMI, LMNA, PLK1, ... \\
	\hline
\end{tabular}
\vfill
\small
\emph{anticholelithogenics}: prevents gallstones, \emph{anti-urolithics}
prevents kidney stones, \emph{RGS4}: Regulator of G-protein signaling 4,
\emph{GPR84}: G-protein coupled receptor 84, \emph{NTCP2}: Ileal bile acid
transporter, \emph{CP3A4}: Cytochrome P450 3A4, \emph{GEMI}: Geminin,
\emph{LMNA}: Prelamin-A/C, \emph{PLK1}: Serine/threonine-protein kinase PLK1
}

\block{5. Computational validation}{
\begin{center}
\includegraphics[width=0.35\columnwidth]{../../../CTNS/notebooks/2022-01-14-top-drugs/named-figure/top-bottom-ratio-top-k.pdf}
\end{center}

Here we validate our network proximity based drug ranking by showing that the
top-$t$ drugs are enriched in ones developed for Alzheimer's.
}

%\bibliography{attila-poster}

\block{6. Experimental validation}{
In a collaboration we are currently testing the four selected top drugs in cell
based assays for their ability to rescue Alzheimer's-like cellular phenotypes.
}

\block{}{
We thank Vijay Varma, Yang An, Jackson Roberts for feedback and the NIH HPC
group for excellent service.}

\end{columns}

\end{document}
