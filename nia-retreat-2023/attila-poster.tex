\documentclass[a1paper,20pt]{tikzposter} %Options for format can be included here
\usepackage{tikz}
\usepackage{multicol}
\setlength{\columnsep}{0.7cm}
\usepackage{natbib}
\bibliographystyle{plainnat}
%\bibliographystyle{unsrtnat}
\author{Attila Jones$^1$, Evan Wu$^2$, Sudhir Varma, Hae Kyung Im$^2$, Madhav Thambisetty$^1$}
\title{Drug Repurposing to Alzheimer’s from Multiomics}

\institute{$1$: Clinical \& Translational Neuroscience Section, LBN, $2$: University
	of~Chicago}

 %Choose Layout
\usetheme{Default}

\begin{document}
\maketitle

\begin{columns}
\column{0.4}

\block{1. Why repurpose drugs to Alzheimer's disease (AD)?}{
\begin{center}
\includegraphics[width=0.2\textwidth]{../../../figures/from-others/drug-repurposing-valli-et-al-2020.png}
\end{center}

Drug repurposing (also called drug repositioning, reprofiling or re-tasking)
is a strategy for identifying new uses for approved or investigational drugs
that are outside the scope of the original medical indication.  Repurposing is
considered a cost- and time-efficient approach to finding potent drugs
for Alzheimer's disease (AD), for which earlier \emph{de novo} drug development
efforts have largely failed.
}

\block{2. Computational drug repurposing using network pharmacology}{
\includegraphics[width=0.3\textwidth]{../../../figures/by-me/drug-target-disgene-network/drug-target-disgene-network-scores.pdf}

In our present work we looked for drug repurposing opportunities towards AD by
combining computational network pharmacology with known AD risk genes from the
latest genomic fine mapping studies.  Our network pharmacology procedure
scored each of >2,000 drugs approved for non-AD indications.  The score
quantifies the topological proximity $d$ in the interactome of the set $T$ of a drug's target genes
to the set $S$ of disease genes: 

\begin{equation}
  d(S, T) = \frac{1}{|T|}\sum_{t \in T} \min_{s \in S} d(s, t)
\end{equation}
where $d(s,t)$ is the shortest path length between disease gene $s$ and drug
target $t$.  The equation can be interpreted as the average length of shortest
paths taken from all targets of a drug to all the disease genes.
}

\column{0.6}

%\block{5. Computational validation}{
%\begin{center}
%\includegraphics[width=0.35\columnwidth]{../../../CTNS/notebooks/2022-01-14-top-drugs/named-figure/top-bottom-ratio-top-k.pdf}
%\end{center}
%
%Here we validate our network proximity based drug ranking by showing that the
%top-$t$ drugs are enriched in drugs developed for Alzheimer's.
%}

%\bibliography{attila-poster}

\block{3. Three top-scoring drugs and their hypothetical mechanism of action
	in AD }{
\includegraphics[width=0.5\textwidth]{../../results/2022-09-26-selected-drugs/knowledge-dtn-anonymized.sif.png}

We selected three drugs with high proximity score that have been found to
exert neuroprotective actions by previous studies.  We performed further
network analyses, which revealed that the three drugs are
connected to long-established AD risk genes like APP, tau, APOE, PSEN1-2,
through mediator genes, some of which were implicated in AD by recent genomic
studies.
}

\block{4. Experimental validation}{
\begin{center}
\includegraphics[width=0.25\columnwidth]{../../../figures/from-others/CTNS-cell-culture-screen.png}
\includegraphics[height=0.3\textwidth]{../../notebooks/2022-09-21-cell-based-assays/named-figure/cell-based-assays-simplified-anonimized.pdf}
\end{center}

We next performed experimental validation of the three selected drugs.  We
found that two of the drugs had neuroprotective properties in cell based
assays for amyloid-beta clearance and secretion.  In future research we are
planning to subject these drugs to further experimental validation.
}

\block{7. Acknowledgements}{
We thank Vijay Varma, Yang An, Jackson Roberts for feedback and the NIH HPC
group for excellent service.}

\end{columns}

\end{document}
