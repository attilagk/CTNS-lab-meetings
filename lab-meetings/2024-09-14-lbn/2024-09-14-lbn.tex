\documentclass[aspectratio=169]{beamer}
%\documentclass[handout]{beamer}

% language settings
%\usepackage{fontspec, polyglossia}
%\setdefaultlanguage{magyar}

% common packages
\usepackage{amsmath, multimedia, hyperref, color, multirow}
%\usepackage{graphicx}
\usepackage{xcolor}
\definecolor{BlueGreen}{cmyk}{0.85,0,0.33,0}
\usepackage{url}
\usepackage[export]{adjustbox}
\usepackage[normalem]{ulem}

% TikZ
\usepackage{tikz}
%\usetikzlibrary{arrows.meta, decorations.pathmorphing, decorations.pathreplacing, shapes.geometric,mindmap}
%\usetikzlibrary{shapes.geometric,fadings,bayesnet}

% beamer styles
\mode<presentation>{
%\usetheme{Madrid}
\usetheme{default}
%\usetheme{Antibes}
%\usecolortheme{beaver}
\usecolortheme{default}
%\usecolortheme{dove}
%\usefonttheme{structureitalicserif}
\setbeamercovered{transparent}
}
\setbeamertemplate{blocks}[rounded][shadow=true]
\AtBeginSubsection[]{
  \begin{frame}<beamer>{Contents}
    \tableofcontents[currentsection,currentsubsection]
  \end{frame}
}
%\useoutertheme[]{tree}

% title, etc
\title{Bayesian Drug Discovery for Alzheimer's Disease}
\subtitle{}
\author{Attila Jones, Vijay Varma, Scantox, (Yang An), Madhav Thambisetty}
\date{}

\begin{document}

\titlepage

\begin{frame}{Outline}
  \begin{enumerate}
    \item Frequentist (classical) and Bayesian statistics
    \item Application I.: Drug dose-response analysis in cell-based assays
    \item Application II.: Testing drugs in cognitively impaired transgenic (TG) mice
  \end{enumerate}
\end{frame}

\begin{frame}{Frequentist and Bayesian statistics}
  \footnotesize
\begin{columns}[t]
\begin{column}{0.45\textwidth}

\includegraphics[width=1.0\columnwidth]{~/figures/from-others/Fisher-Bayes-cartoon.jpeg}

  \tiny{\url{https://agostontorok.github.io/2017/03/26/bayes_vs_frequentist/}}
\end{column}

\begin{column}{0.55\textwidth}

  \begin{tabular}{c|c|c}
    & Frequentist & Bayesian \\
    \hline
    key quantity & likelihood & posterior distribution \\
                 & $P(\mathrm{data} | \mathrm{model})$ & $P(\mathrm{model} | \mathrm{data})$ \\
    \hline
    parameters & fixed & random \\
    \hline
    estimation & max.~likelihood & posterior summary \\
    \hline
    computation & least squares & MCMC \\
  \end{tabular}

\end{column}
\end{columns}
\end{frame}

\begin{frame}{Application I.: Drug dose-response analysis in cell-based assays}
\begin{columns}[t]
\begin{column}{0.5\textwidth}

\includegraphics<1>[width=\columnwidth]{~/figures/from-others/BV2_DC_10PS_20X_2-20211206160943680.png}

\only<2>{Model as depenency graph}

\includegraphics<2>[height=0.7\textheight]{../../../notebooks/2023-09-13-cell-based-assays-bayes/named-figure/sigmoid-2.png}

\end{column}

\begin{column}{0.5\textwidth}

\includegraphics[width=\columnwidth]{~/figures/from-others/dose-response-curve-stimulation-wikipedia.jpg}

\tiny{\url{https://en.wikipedia.org/wiki/Dose-response\_relationship}}
\end{column}
\end{columns}
\end{frame}

\begin{frame}{Approximate posterior distribution with Markov Chain Monte-Carlo}
\begin{columns}%[t]
\begin{column}{0.3\textwidth}

\includegraphics[height=0.7\textheight]{../../../notebooks/2023-09-13-cell-based-assays-bayes/named-figure/sigmoid-2.png}
\end{column}

\begin{column}{0.7\textwidth}

\includegraphics[width=1.0\columnwidth]{../../../notebooks/2023-09-13-cell-based-assays-bayes/named-figure/mcmc-trace-sigmoid-2.pdf}
\end{column}
\end{columns}
\end{frame}

\begin{frame}{\only<1>{Curves from the posterior.  Hypotheses}\only<2>{Bayesian hypothesis testing}}
  \small
\begin{columns}[t]
\begin{column}{0.5\textwidth}

  \includegraphics<1>[scale=0.5]{../../../notebooks/2023-09-26-cell-bayes-assays/named-figure/prior-posterior-curves-sigmoid-2.pdf}

  \only<2>{
      \begin{itemize}
        \item $p$-value: evidence against $H_0$
        \item Bayes factor: evidence for $H_1$
      \end{itemize}
    
    \begin{eqnarray*}
      \mathrm{BF} &=& \frac{P(H_1|\mathrm{data})}{P(H_0|\mathrm{data}) + P(H_2|\mathrm{data})} \\
       %&=& \frac{0.9}{0.1 + 0} = 9 \\
      2 \times \log \mathrm{BF} &\approx& 4.4
    \end{eqnarray*}

    \begin{tabular}{c|l}
      $2 \times \log \mathrm{BF}$ & Strength of evidence \\
      \hline
      $ < 0 $ & nonexistent \\
      0 to 2 & negligible \\
      2 to 6 & moderate \\
      6 to 10 & strong \\
      $ 10 \ge$ & very strong \\
    \end{tabular}
  }
\end{column}

\begin{column}{0.5\textwidth}

  \includegraphics[scale=0.5]{../../../notebooks/2023-09-26-cell-bayes-assays/named-figure/prior-posterior-density-violin-vertical.pdf}
\end{column}
\end{columns}
\end{frame}

\begin{frame}{}
\begin{columns}[t]
\begin{column}{0.5\textwidth}

  \includegraphics[height=1.0\textheight]{../../../notebooks/2023-09-26-cell-bayes-assays/named-figure/H10-bayes-factors-anonym_drugs.pdf}
\end{column}

\begin{column}{0.5\textwidth}
  \only<2>{Aggregation: Bayesian model averaging}

  \includegraphics<2>[scale=0.4]{../../../notebooks/2024-04-09-qps-results-summary/named-figure/exper-barplot-drug-A.pdf}

  \includegraphics<2>[scale=0.4]{../../../notebooks/2024-04-09-qps-results-summary/named-figure/exper-barplot-drug-B.pdf}

  \includegraphics<2>[scale=0.4]{../../../notebooks/2024-04-09-qps-results-summary/named-figure/exper-barplot-drug-C.pdf}
\end{column}
\end{columns}
\end{frame}

\begin{frame}{Drug prioritization with Bayesian model averaging}
\begin{columns}[t]
\begin{column}{0.5\textwidth}

  \includegraphics[scale=0.4]{../../../notebooks/2024-04-09-qps-results-summary/named-figure/exper-barplot-Aβ-release-H4-cells-TI.pdf}
\end{column}
\begin{column}{0.5\textwidth}

  \includegraphics<2>[scale=0.4]{../../../notebooks/2024-04-09-qps-results-summary/named-figure/exper-barplot-LPS-neuroinflammation-BV2-cells-TI.pdf}
\end{column}
\end{columns}
\end{frame}

\begin{frame}{Application II.: Testing drugs in cognitively impaired transgenic (TG) mice}
  \small
\begin{columns}[t]
\begin{column}{0.5\textwidth}
\begin{enumerate}
  \item WT (wild type) and TG (5xFAD) mice
  \item confirm cognitive impairment of TG (5xFAD) w.r.t WT (wild type) mice
  \item Morris Water Maze
  \item training probe: \alert{escape latency} is a sensitive outcome
  \item test drug A and drug B together or separately in TG mice
\end{enumerate}
\end{column}

\begin{column}{0.5\textwidth}
  Morris water maze

\includegraphics[width=\columnwidth]{../../../../figures/from-others/MorrisWaterMaze.svg.png}
\end{column}
\end{columns}
\end{frame}

\begin{frame}{Bayesian regression of escape latency}

  \begin{center}
  \includegraphics[scale=0.5]{../../../notebooks/2024-06-04-5xfad-behavior-CO28154/named-figure/model-latency-CO28154-TUDCA_HCQ-TG-anonym.pdf}
  \end{center}
\end{frame}

\begin{frame}{Testing drug-induced rescue of TG phenotype}

  \begin{center}
  \includegraphics[scale=0.5]{../../../notebooks/2024-06-04-5xfad-behavior-CO28154/named-figure/CI-BF-TUDCA-HCQ-anonym.pdf}
  \end{center}
\end{frame}

\begin{frame}{Conclusion: the benefits of Bayesian statistics}
\begin{itemize}
  \item hypothesis testing with biologically meaningful effect sizes
  \item aggregation of results using Bayesian model averaging
\end{itemize}
\end{frame}

\end{document}

\begin{columns}[t]
\begin{column}{0.5\textwidth}
\end{column}

\begin{column}{0.5\textwidth}

\end{column}
\end{columns}
