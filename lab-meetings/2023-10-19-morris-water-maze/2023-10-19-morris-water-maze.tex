\documentclass[aspectratio=169]{beamer}
%\documentclass[handout]{beamer}

% language settings
%\usepackage{fontspec, polyglossia}
%\setdefaultlanguage{magyar}

% common packages
\usepackage{amsmath, multimedia, hyperref, color, multirow}
%\usepackage{graphicx}
\usepackage{xcolor}
\definecolor{BlueGreen}{cmyk}{0.85,0,0.33,0}
\usepackage{url}

% TikZ
\usepackage{tikz}
%\usetikzlibrary{arrows.meta, decorations.pathmorphing, decorations.pathreplacing, shapes.geometric,mindmap}
%\usetikzlibrary{shapes.geometric,fadings,bayesnet}

% beamer styles
\mode<presentation>{
%\usetheme{Madrid}
\usetheme{default}
%\usetheme{Antibes}
%\usecolortheme{beaver}
\usecolortheme{default}
%\usecolortheme{dove}
%\usefonttheme{structureitalicserif}
\setbeamercovered{transparent}
}
\setbeamertemplate{blocks}[rounded][shadow=true]
\AtBeginSubsection[]{
  \begin{frame}<beamer>{Contents}
    \tableofcontents[currentsection,currentsubsection]
  \end{frame}
}
%\useoutertheme[]{tree}

% title, etc
\title{Analyzing Morris Water Maze Experiments}
\subtitle{}
\author{Attila Jones}
\date{}

\begin{document}

\titlepage

\section{Our data}

\begin{frame}{Introduction}
\begin{columns}[t]
\begin{column}{0.5\textwidth}
\begin{itemize}
  \item WT and 5xFAD mice $\pm$ drug $X$
  \item Morris water maze 
  \begin{itemize}
    \item day 1-4 training probes (5): learning
    \item day 5 test probes (2): memory
    \item WT--5xFAD difference is variable
    \item the best statistical approaches?
  \end{itemize}
  \item Y maze
\end{itemize}
\end{column}

\begin{column}{0.5\textwidth}

\includegraphics[width=\columnwidth]{../../../../figures/from-others/MorrisWaterMaze.svg.png}
\end{column}
\end{columns}
\end{frame}

\begin{frame}{Training trials}
\begin{center}
\includegraphics<1>[scale=0.5]{../../../notebooks/2023-10-13-5xfad-maze/named-figure/training-Latency-s.pdf}
\includegraphics<2>[scale=0.5]{../../../notebooks/2023-10-13-5xfad-maze/named-figure/training-Distance-m.pdf}
\includegraphics<3>[scale=0.5]{../../../notebooks/2023-10-13-5xfad-maze/named-figure/training-Floating.pdf}
\includegraphics<4>[scale=0.5]{../../../notebooks/2023-10-13-5xfad-maze/named-figure/training-Thigmotaxis.pdf}
\includegraphics<5>[scale=0.5]{../../../notebooks/2023-10-13-5xfad-maze/named-figure/training-Velocity-m-s.pdf}
\end{center}
\end{frame}

\begin{frame}{Probe trial}
\begin{columns}[t]
\begin{column}{0.5\textwidth}

\includegraphics[scale=0.5]{../../../notebooks/2023-10-13-5xfad-maze/named-figure/target-quadrant-barplot.pdf}
\end{column}

\begin{column}{0.5\textwidth}

\begin{itemize}
  \item alternative outcome $y$: target Q crossings
  \item alternative experiment: Y-maze
\end{itemize}
\end{column}
\end{columns}
\end{frame}

\section{Similar studies}

\begin{frame}{5xFAD learning and memory are age- and sex-dependent}
\begin{columns}[t]
\begin{column}{0.5\textwidth}

\includegraphics[width=\columnwidth]{../../../../figures/from-others/mwm-5xFAD-Fig6A-F.png}
\end{column}

\begin{column}{0.5\textwidth}

\includegraphics[width=\columnwidth]{../../../../figures/from-others/mwm-5xFAD-Fig6G-J.png}
\end{column}
\end{columns}
\begin{center}
\tiny \url{https://doi.org/10.1111/gbb.12794}
\end{center}
\end{frame}

\begin{frame}{Sensitivity analysis}
\begin{columns}[t]
\begin{column}{0.4\textwidth}

%\includegraphics[scale=0.5]{../../../notebooks/2023-09-26-cell-bayes-assays/named-figure/prior-posterior-curves-sigmoid-2.pdf}
\includegraphics[width=1.0\columnwidth]{../../../../figures/from-others/mwm-measure-sensitivity-Fig5A.png}
\end{column}
\begin{column}{0.6\textwidth}

%\includegraphics<1>[scale=0.5]{../../../notebooks/2023-09-26-cell-bayes-assays/named-figure/prior-posterior-density-complex-sigmoid-2.pdf}
\includegraphics[width=1.0\columnwidth]{../../../../figures/from-others/mwm-measure-sensitivity-Fig5B.png}
\end{column}
\end{columns}
\begin{center}
\tiny \url{https://doi.org/10.3389/neuro.07.004.2009}
\end{center}
\end{frame}

\section{Preliminary analysis}

\begin{frame}{A better way of modeling fraction in quadrants}
\begin{center}
\includegraphics<1>[height=0.7\textheight]{../../../../figures/from-others/mwm-dirichlet-Fig1.png}
\only<2>{
\begin{eqnarray}
  \Lambda &=& 2 \times \left( \sup_{\alpha \in H_1} \log f(y | \alpha) -
    \sup_{\alpha' \in H_0} \log f(y | \alpha') \right) \\
    \alpha &=& g(x) \\ 
    g &=& \mathrm{?}
\end{eqnarray}
}
\includegraphics<3>[height=0.7\textheight]{../../../../figures/from-others/mwm-dirichlet-Fig5.png}

Limitation: the method ``cannot compare memory abilities between several
groups of animals''

\tiny \url{https://doi.org/10.12688/f1000research.20072.2}
\end{center}

\end{frame}
\begin{frame}{Application to our data}
\begin{columns}[t]
\begin{column}{0.33\textwidth}

\includegraphics[width=1.0\columnwidth]{../../../notebooks/2023-10-13-5xfad-maze/named-figure/dirichlet-plot-group-A.png}
\end{column}

\begin{column}{0.33\textwidth}

\includegraphics[width=1.0\columnwidth]{../../../notebooks/2023-10-13-5xfad-maze/named-figure/dirichlet-plot-group-B.png}
\end{column}

\begin{column}{0.33\textwidth}

\includegraphics[width=1.0\columnwidth]{../../../notebooks/2023-10-13-5xfad-maze/named-figure/dirichlet-plot-group-C.png}
\end{column}
\end{columns}
\end{frame}


\againframe{summary}


\end{document}

\begin{columns}[t]
\begin{column}{0.5\textwidth}
\end{column}

\begin{column}{0.5\textwidth}

\end{column}
\end{columns}
