\documentclass[aspectratio=169]{beamer}
%\documentclass[handout]{beamer}

% language settings
%\usepackage{fontspec, polyglossia}
%\setdefaultlanguage{magyar}

% common packages
\usepackage{amsmath, multimedia, hyperref, color, multirow}
%\usepackage{graphicx}

% TikZ
\usepackage{tikz}
%\usetikzlibrary{arrows.meta, decorations.pathmorphing, decorations.pathreplacing, shapes.geometric,mindmap}
%\usetikzlibrary{shapes.geometric,fadings,bayesnet}

% beamer styles
\mode<presentation>{
%\usetheme{Warsaw}
%\usetheme{Antibes}
\usecolortheme{beaver}
%\usecolortheme{seahorse}
%\usefonttheme{structureitalicserif}
\setbeamercovered{transparent}
}
\setbeamertemplate{blocks}[rounded][shadow=true]
\AtBeginSubsection[]{
  \begin{frame}<beamer>{Contents}
    \tableofcontents[currentsection,currentsubsection]
  \end{frame}
}
%\useoutertheme[]{tree}

% title, etc
\title{The Main Title Naming Key Concepts}
\subtitle{A subtitle may be shorter and more technical}
\author{Attila Jones}
\date{}

\begin{document}

\begin{frame}{An inspiring study by Peter Antal's group using Bayesian
  networks}
\begin{center}
\includegraphics[height=0.7\textheight]{figures/bn-bmla-2014-pnas-fig3.jpeg}
\end{center}

\begin{center}
\tiny{Juhasz, ..., Antal et al PNAS 2014}
\end{center}
\end{frame}


\begin{frame}{A Bayesian network}
\begin{columns}[t]
\begin{column}{0.6\textwidth}

\includegraphics[width=1.0\columnwidth]{figures/bn-intro-sprinkler.png}

\begin{center}
\tiny{Murphy 1998}
\end{center}
\end{column}

\begin{column}{0.4\textwidth}
\begin{itemize}
  \item Bayesian network \(\neq\) Bayesian statistics
  \item graph: dependency structure
  \begin{description}
    \item[nodes] random variables
    \item[edges/verteces] dependencies (causality)
  \end{description}
  \item conditional independence
  \begin{itemize}
    \item WetGrass \(\perp\) Cloudy \(|\) Sprinkler, Rain
  \end{itemize}
\end{itemize}

\end{column}
\end{columns}
\end{frame}


\begin{frame}{The Markov blanket of variable $A$}
\begin{columns}[t]
\begin{column}{0.5\textwidth}

\begin{center}
\includegraphics[width=0.6\columnwidth]{figures/markov-blanket-wikipedia.png}
\end{center}

\begin{center}
\tiny{Wikipedia}
\end{center}
\end{column}

\begin{column}{0.5\textwidth}
\begin{enumerate}
  \item<1> Markov blanket interpretations
  \begin{itemize}
    \item conditional~independence
    \item information
    \item strong/weak relevance
  \end{itemize}
  \item<2> application examples if \(A = \{ control, MCI, AD \} \)
  \begin{itemize}
    \item find the most informative set of biomarkers
    \item find strongly relevant target genes and environmental variables
  \end{itemize}
  \item<3> the task
  \begin{itemize}
    \item learn dependency graph from observations of $A$ and other
    variables
  \end{itemize}
\end{enumerate}

\end{column}
\end{columns}
\end{frame}


\begin{frame}{Bayesian inference of the Bayesian network's dependency structure}
  \[
  \mathrm{data} \overset{MCMC}{\to} P_M(\mathrm{full.Bayesian.network} \, M \, | \, \mathrm{data})
  \overset{\text{avg.}}{\to} P_v(\mathrm{single.dependency} \, v \, | \, \mathrm{data})
  \]

\begin{center}
  Example: coronary heart disease

  \tiny{Madigan et al 1996}
\end{center}

\begin{columns}[t]
\begin{column}{0.6\textwidth}

\begin{center}
\includegraphics[width=1.0\columnwidth]{figures/bma-dag-madigan-1996-coronary-heart-dis.png}
\end{center}

\vfill

{\small
\begin{tabular}{r|cc}
  & complexity & informative $P$ \\
  \hline
  \hline
  full Bayesian network $M$ & \checkmark\checkmark &  \\
  ??? & \checkmark & \checkmark \\
  single dependency $v$ &  & \checkmark\checkmark \\
  \hline
\end{tabular}
}

\end{column}

\begin{column}{0.4\textwidth}

\begin{center}
\includegraphics[width=1.0\columnwidth]{figures/bma-dag-madigan-1996-coronary-heart-dis-table.png}
\end{center}

\end{column}
\end{columns}
\end{frame}


\begin{frame}{Optimal compromise between complexity \& informativeness}
  {BN-BMLA: \emph{B}ayesian \emph{n}etwork based \emph{B}ayesian \emph{m}ultilevel
  \emph{a}nalsys}

\begin{columns}[t]
\begin{column}{0.5\textwidth}
  $k$-sub-Markov blanket sets ($k$-subMBS)

\includegraphics[width=1.0\columnwidth]{figures/antal-book-fig3.png}
\end{column}

\begin{column}{0.5\textwidth}
  informativeness decreases with $k$

\includegraphics[width=1.0\columnwidth]{figures/antal-book-fig11.png}
\end{column}
\end{columns}
\begin{center}
  \tiny{Antal et al 2014}
\end{center}
\end{frame}


\begin{frame}%{Multivariate, complex phenotypes}
\begin{columns}[t]
\begin{column}{0.5\textwidth}

\includegraphics[height=0.85\textheight]{figures/bn-bmla-pharmaceuticals-2021.png}

\begin{center}
\tiny{Eszlari,..., Antal et al 2021}
\end{center}
\end{column}

\begin{column}{0.5\textwidth}

Categories of variables: complex phenotypes
\begin{description}
  \item[Fram] Framingham: cardiovascular risk
  \item[BSI] Brief Symptom Inventory: psychiatric outcome
  \item[RRS] Ruminative Response Scale: recycling of negative thoughts
\end{description}

\end{column}
\end{columns}
\end{frame}

\begin{frame}{Summary}
\begin{enumerate}
  \item Bayesian networks can represent any dependency structure of variables
  \begin{itemize}
    \item genetic variants, expressionlevels, environmental exposures, phenotypes
    \item can dissect direct effects from mediated ones
  \end{itemize}
  \item Learning dependency structures from observations...
  \begin{itemize}
    \item ...typically leaves much uncertainty
  \end{itemize}
  \item BN-BMLA technique (Antal et al)...
  \begin{itemize}
    \item increase local certainty by decreasing the complexity of hypotheses \emph{as much as needed}!
  \end{itemize}
  \item<2-> since BN-BMLA is Bayesian statistics...
  \begin{itemize}
    \item posterior prob.:~simple interpretation
    \item built-in correction for multiple
      hypothesis testing
    \item causes of multivariate phenotypes (complex mental tests) can be investigated
    \item adjustable granularity of effects: SNPs $\to$ genes $\to$ pathways (ontology)
    \item difficult: MCMC needed
  \end{itemize}
\end{enumerate}
\end{frame}


\end{document}

\begin{columns}[t]
\begin{column}{0.5\textwidth}

\end{column}

\begin{column}{0.5\textwidth}

\end{column}
\end{columns}
